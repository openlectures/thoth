\RequirePackage[l2tabu,orthodox]{nag} % Checks for incorrect or obsolete LaTeX packages

\documentclass[DIV=calc,11pt,parskip,numbers=noenddot]{scrartcl} % Uses the KOMA-Script package with customizations

% Universal Fixes
\usepackage{fixltx2e} % Corrects LaTeX2e bugs and quirks
\usepackage{ifxetex} % Checks if XeTeX/XeLaTeX is the compiler-of-choice

% Math
\usepackage{amsmath} % Swiss-knife math package

% Graphics
\usepackage{tikz} % Engine that produces vector graphics from geometric/algebraic descriptions
\usepackage{graphicx} % Allows inserting of graphics and images
\usepackage{epstopdf} % Converts .eps files to .pdf for easy manipulation

% Layout and Format
\usepackage{parallel} % Aligns paragraphs across columns
\usepackage{lineno} % Appends line-numbers in the margin with \linenumbers
\usepackage{showkeys} % Shows label references where they are defined
\usepackage{csquotes} % Fixes inline and display quotations
\usepackage{paralist} % Allows inline enumeration in a paragraph and compact enums/lists

% Tables and Figures
\usepackage{booktabs} % For drawing nice tables with proper line weights
\usepackage{flafter} % To ensure that figures float only after they are defined/referenced
\usepackage{subfig} % Figure-ception - Allows figures within figures
\usepackage{array}  % Extends options for column formats, alignments and layouts
\usepackage{tabu} % Provides better control for tables and column widths
\usepackage{longtable} % Allows tables to span across pages - integrates with tabu
\usepackage{multicol} % Allows spanning columns in tables
\usepackage{caption} % Allows greater customization of captions and captions outside floats

% Referencing
\usepackage[longnamesfirst]{natbib} % Reimplements \cite to work with author-year and numerical citations
\usepackage{cleveref} % Adds semantic naming when referencing figures
\usepackage{varioref} % Introduces referencing by page instead of figure number

% Fonts and Typography
\usepackage{microtype} % Tweaks smallish fonts and kernings
\usepackage{textcomp} % Supports the Text Companion font, which provides additional text symbols
\usepackage{siunitx} % Provides support for typesetting SI units
\usepackage{ellipsis} % Fixes space uneveness around ellipses
\usepackage{url} % Allows encapsulated URLs to break across lines
\usepackage[colorlinks,hypertexnames=false,plainpages=false]{hyperref} % Converts \url references to valid hyperlinks in PDF documents

\usepackage{euler}

\ifxetex
\usepackage{fontspec} % Allows usage of system opentype/truetype fonts
\setsansfont[BoldFont={* Bold}]{Miso} % Sets the Header font via system font name
\fi
\usepackage[T1]{fontenc} % Sets output font encoding to support accented characters and Type 1 fonts
\usepackage{concrete} % Swaps out ancient Computer Modern font for Latin Modern

\setkomafont{title}{\fontencoding{EU1}\sffamily}
\setkomafont{section}{\fontencoding{EU1}\sffamily\Large\centering}

% Title Information
\title{2 SAT Essay Writing Tips}
\author{\large by Su Hang}
\date{\small correct as of \today}
\begin{document}
\maketitle
\section{Meet Your Graders}
It is a bit of a hack, but knowing the psyche of whoever is marking your essay can actually help you curry favour with them, and help you score better.

Now. Your essay’s graders are immensely bored American high school teachers (and rarely, college professors) looking to earn a little spare cash.

Your essay will be scored by 2 graders. Each will have, on average, only a hundred and twenty seconds (that's two minutes) to scan through and give you a score ranging from 0 to 6. If the two graders disagree by more than 2 or so points, a third will be called in to arbitrate. Now that we know the target audience that we will be writing for, we shall next learn about how to use this to our advantage while writing the essay in the next checkpoint.
\section{What This Means For You}
Now that you know that your essay’s graders have to read, think about, and judge your essay, all in 2 minutes, you must make your first impressions count.

First and foremost, fill up the full 2 pages. The claim that how much you write doesn’t matter is a lie, because when your graders have to judge you on so few indications, essay length becomes one of the more prominent flags of a student who is proficient at writing. Actual studies by an MIT Writing Director have found that there is a high level of correlation between the length of an essay, and the score it was given on the exam! Use large handwriting, leave more space between words, smoke your way through, whatever. Just fill it all up and you would have scored that first impression.

Talking about impressions, you should use most of your bombastic vocabulary at the Introduction, and the Conclusion. Chances are, the reader’s attention will be at the peak when they start (New essay to grade!), and when they end (On to the next one! What score should I give this essay?) Therefore, make full use of this fact, and demonstrate your fantabulous command of the English language when they are paying the most attention. That is not to say that you need not use any in the paragraph texts.

Also, do remember to keep your sentences succinct. Sometimes, it’s nice to have long and flowery descriptions of scenery and everything, but this is not the time to do it. Those same descriptions slow the pace of your essay and make it doubly hard to read and figure out what you’re saying. Make your sleepy grader see immediately that your essay is worthy of a high mark.

Very importantly, write legibly. Your graders are having a hard time trying to get through thousands and thousands of essays over many days as it is. Why make it even harder for them to give you the mark? If they can’t read it, they’ll knock a few points off your score. So write neatly.

Lastly, when you’re writing, do not, I repeat, do not veer off topic. Know what the topic is, and stay there. You might be knowledgeable about something close to it, but if it is deemed off topic, then you literally get a big fat zero, according to College Board guidelines.
\section{Examples}
When writing your essay itself, you should think of your examples, and then decide on your stand. You might think that you have a fixed stand on the issue you want to write about, but if you’re unlucky, you might just find that you’ll run out of steam halfway through the essay and cannot think up any more examples. In the interests of time, think up all your examples, and then write the essay.

For your examples, avoid contentious topics in America. If you have been reading, you’d know of some of the more sensitive topics that Americans face today: separation of church and state, their war on terrorism (and Islam), gay rights, healthcare reform, evolution, et cetera, et cetera. If you write in praise of Charles Darwin, and your grader just happens to be a Bible-wielding fundamentalist, then there goes your score. Theoretically, graders are not supposed to penalise a response just because they disagree with it, but you never know what subconscious psychological effects may do to your score. Write to please everybody.

A good rule is to choose the examples that you know most about! If you have the perfect example, but don’t know much about it and just try to smoke your way through, chances are that you won’t be able to write anything immensely coherent or detailed within the short time you have. Do yourself a favour and write about things you’re familiar with.

And although it may seem unethical, it is totally okay to make stuff up. Sometimes, a well placed name-drop can help your case along. If you know that Sir Jonathan Ive is Senior Vice President of Industrial Design at Apple, bonus points to you, because wow, you know your world affairs really well. But if you can’t remember the poor bugger’s name, just make one up. Like Sir Jack Irving or something. The grader will probably not know who you’re talking about either way because there really is no way to check within that short timespan that they have to grade your essay, but they will still be impressed by your knowledge. Just don’t try to claim that Hitler had like, 3 moustaches, or something.

Now that we know how to please our graders, we will look at how to polish and perfect those shotgun essays in the next checkpoint.
\section{How To Practice}
When you practice writing the SAT essay, whether as an individual component, or as part of a complete 5 hour marathon, the most important thing to do, is to always, always, time yourself. Timing is usually the most tricky thing to manage during the essay, regardless of your writing skill level. During the actual SAT, you will only be given 25 minutes to think, plan, and write your essay. These 25 minutes will fly by even faster when you are under stress on test day. So, when you are practicing, give yourself 20 minutes instead of 25, and try and force yourself to stop at that time, so that you have time to check on the actual day.

After writing the essay, take a photo or a scan of it, and send it to your grammar nazi of a friend to read and grade, on a scale of 0-6. Painful as it may be, ask them why they gave you a certain grade, and, more importantly, ask for advice on how you can improve. Keep those things in mind and apply them as you write your next essay, or else you will just be stuck grinding away practicing the same wrong things over and over again without correcting them!
\section{Stockpiling}
Prepare some versatile examples to use in your essay. If, hypothetically, you are a huge fan of Apple and Steve Jobs, and know of everything that goes on regarding them, you may choose to write about Apple, regardless of what the prompt is. Let’s take this idea for a spin with prompts that have appeared on the test before.

If the prompt is, “Do all people need to be creative?” you can answer, “Yes! As a tech company, Apple has practically single handedly built the smartphone market from the ground up in the past few years with their revolutionary, best selling iPhone.” If the prompt is, “Do people truly benefit from hardship and misfortune?” you can answer, “Yes. Steve Jobs was ousted from Apple after a power struggle, but he went on to found NeXT and acquired Pixar. Through his trials and tribulations, he gained experience as a leader, and when he eventually returned to Apple again, he saved the company from the brink of bankruptcy, and turned it into the world’s most profitable tech company.” And so on. The moral of the story is, as long as you are familiar with a certain topic, you can usually turn what you know around to fit what the prompt is asking for. Choose about 2 or 3 topics that you are interested in, and are willing to read about, to prepare for test day.
\section{Other Things That May Work}
You can choose to listen to this section, and you can choose to not. Just like some people choose to burn incense and drink the ashes for good test scores, these strategies may or may not work for you.

You can use supporting points familiar to the American public. Or not. Such as Dickens. Or the Emancipation Proclamation. Or the Romney "Rmoney" gaffe. Your graders, as true blue hegemonic Americans, may not want to hear about the opposition politics in Hougang, Singapore. On the other hand, if you are not familiar with American affairs, it is perfectly okay to stick to what English speaking people all over the world should know. Such as Harry Potter. And the Greek debt crisis. On the other hand, if you cite examples that every single American high-schooler would cite, your scorer would likely already be bored with those examples and would not like reading your essay.

Or you can start your essay with the prompt’s keyword. Some people may find it boring. Some may find it to be easier that way. Either way, it’s a no-brainer way to tell your grader that your essay is on track.

Another strategy is, don’t use ‘I’ in your essay. In an effort to appear learned and distinguished, scientists have devised a way to objectivise their scientific findings such that their results seem to speak simple truths. That is, to avoid the self when talking. Using ‘I’ makes your essay sound like something an elementary school kid would write. 

Keep these things in mind during practice, and during the test itself. In the next lesson, we will be looking at exactly how to go about writing the SAT essay.
\end{document}
