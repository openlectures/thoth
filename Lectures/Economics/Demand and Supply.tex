\RequirePackage[l2tabu,orthodox]{nag}
\documentclass[DIV=classic,11pt,numbers=noenddot,listof=totoc,bibliography=totoc,parskip]{scrartcl}
\usepackage{fixltx2e,multicol,graphicx,url,caption,csquotes,amsmath,paralist,ellipsis,subfig,array,microtype,flafter,siunitx,cleveref,booktabs,textcomp}
\usepackage[hypertexnames=false,plainpages=false]{hyperref}
\usepackage{fontspec}
\setsansfont[BoldFont={* Bold}]{Miso}
\setmainfont[BoldFont={* Bold},ItalicFont={* Italic},BoldItalicFont={* Bold Italic}]{Courier Prime}
\title{Demand and Supply}
\author{Linan Qiu}
\date{}
\begin{document}
\maketitle
\tableofcontents
\newpage
\section{You Demand}
\subsection{Visualizing Demand}
Before I bore you with the things like demand curve, supply curve, consumer surplus and producer surplus, let's go into something simpler. Everyone knows cookies. I have a pack of cookise here. Let's say that the pack of cookies cost five dollars. You think that the cookies are expensive. I can't afford that many cookies. I'm just going to buy one packet of cookies. Now imagine a graph. The vertical axis is the price. The horizontal axis is the quantity. At 5 dollars, I buy one pack of cookies. If I drop the price to 4 dollars, you can buy more cookies - you buy two packs. If I lower the price a little bit more, to 3 dollars, I'll buy even more cookies - three packets. If I lower the price to 2 dollars, you might buy 4 packets. To 1 dollar, 5 packets. What you see happening is that on the graph, you have 5 dots. And these dots show the relationship between the price and the quantity. So in the next checkpoint, you will see how these points can be joined together to form the demand curve.
\subsection{Connecting the Dots}
Cookies might be a simple example because if you are a single person, you can buy 5 packets of cookies, then you arrange them in a line and you get a curve. But what if I were to try and construct a demand curve for Apple computers? Each apple computer costs 3000 dollars. Even if I lower the price to 2900, you are not going to just buy another Apple computer the price lowered to 2900. How am I going to construct a downward sloping demand curve? Then don't think of it as a single person because the demand curve is never meant for a single person! The demand curve is meant for a market. So let's put this in perspective. Imagine that you are in a school. The school has a population of 1000. So if I sell the Apple computer at 3000 dollars, then maybe 100 people will buy it and each of them buys one. If I lower it to 2900, some more people might want to buy it. 120 people will buy it! If you lower the price of Apple computers more and more and you chart this progress, again you can form a line. So it's not just about a single person, but a market as a whole. When we connect all these lines, it becomes a demand curve for the market. So this is how you construct a demand curve for every single thing in the world.
\subsection{Consumer Surplus}
Remember the school example? Let's say we are selling Macbook Pros for 2500 dollars and there are 300 students who are wiling to buy it? Even when we sell it at 3000 dollars, there are 100 students who are willing to buy it. So those 100 students say "hey I'm getting it at a cheaper price! I'm willing and able to pay 3000 dollars for it but right now I'm only paying 2500." So what happens to the additional 500 dollars? To the student, he'll say "I got something at a discount. So if we map out all the discounts even for the students who are willing to pay 2900 and the students who are paying 2800, 2700 and so on, all the way until we hit the 2500 price, we have something called the consumer surplus. So this consumer surplus can be viewed as something extra that the consumers got because they are willing and able to pay for something at a certain price but they got it for something cheaper. So that is what we call the consumer surplus or what we call the consumer welfare. 
\subsection{Demand vs Quantity Demanded}
So there's a difference between Demand and Quantity Demanded because there is a difference. One shifts the curve and the other one is a shift along the curve. Newspapers use this term interchangeably. That is wrong. Here is the difference. Imagine your demand curve, Y axis price, X axis quantity and you draw the demand curve. Think of the price as a slider. When you change the price, the quantity changes as well, according to how much you change the price. And that makes the price your independent variable and the quantity your dependent variable. Let's ignore that for now. When you change the price, the quantity changes. The quantity demanded changes. Again, when you change the price, the quantity demanded changes. But when something other than your price changes, for example something like consumption pattern. Your demand drops. In that case, your entire demand curve shifts. You have a new demand curve. So again, if it is a price factor, then you get a shift along the curve and you still keep the same demand curve. If it is a non price factor, you move the entire demand curve. And that is the difference between demand and quantity demanded.
\newpage
\section{I Supply}
\subsection{Visualizing Supply}
So now that we are done with consumers. Let's go on to producers. So the producers look at the supply curve. Now what is the supply curve? Let's say I'm a producer of marker pens. If I can sell marker pens at 3 dollars and there is a great amount of profit there, I'll be willing to produce a thousand market pens. If, however, the price of marker pens drop to 2 dollars, then I will be less willing to produce that many pens because I will not be making that much profits. I will only be selling 500 pens. So again, if we connect a lot of these dots together, we get a preliminary supply curve.
\subsection{Connecting the Dots (Again)}
In your demand curve, we looked at a thousand consumers together? The same applies for producers and the supply curve. If you are to analyze the marker pen market as a whole and we talk about numbers in the millions, there's realistic way that a single small producer can produce a million pens. We can look a hundred or a thousand producers and ask them "If I price the pen at 3 dollars, how many pens are you willing to produce? What about other prices?" This is the supply curve of the market.
\subsection{Producer Surplus}
So let's say the producer is willing to sell a 100 pens at 3 dollars, and 75 pens at 2 dollars, now eventually if the price goes for 3 dollars and he's actually able to sell pens at 2 dollars or 1 dollars, then he must that he has been able to earn an additional amount of profit on top of what the producer's cost is. He is able to supply at 1 dollars or 2 dollars, but he is selling at 3 dollars. It is the additional amount that the producer earns. And this concept is related to your consumers surplus idea. It is the producer's surplus. It is the amount of welfare that the producer gets because he is selling above the minimum price he will sell at. 
\subsection{Supply vs Quantity Supplied}
In the supply curve as well, there is a difference between changing supply and changing quantity supplied. Again the difference is in price. If your price changes, and the only factor that changes is your price, then it's probably a shift along the supply curve. But if the supply changes, for example an oil producer finds a new oil field in UAE, or let's say you build a new cracker in Singapore, then you have an increase in the supply of oil. That will shift the supply curve entirely. That is a non price factor and it affects your supply. So again, change in supply curve and change in quantity supplied are different.
\newpage
\section{The Markets}
\subsection{Attaining Zen (or Equilibrium)}
So we've talked about demand and supply. When we put these two curves on the same graph, there will be an intersection. The equilibrium price and quantity, the ending point, doesn't just happen because the two lines intersect. That is just a graphical phenomenon. What actually happens in the background is this. If we choose a price that is not the equilibrium price, for example a price higher than the equilibrium price, is that the quantity supplied is higher than the quantity demanded. That means, for example, your suppliers are willing to supply 500 units of a good, but your consumers are only willing and able to buy 300 units at that price. Only 300 units are sold, and there's a surplus of 200 units. The producers will cut back on production and lower their price to cut down on their surplus. So instead, they lower to 400 units of production. At 400 units, that is when the quantity demanded equals the quantity supplied. So when the price is higher than the equilibrium price, the suppliers will lower the price of the goods so that consumers can consume more, quantity demanded increases and eventually you get to the equilibrium. If I choose a price that is lower than the equilibrium price, the quantity demanded will be higher than the quantity supplied. Your producers are not willing a lot because the price is too low. But the consumers want to consume a lot more due to the low price. So eventually the producers get the signal and increase the price and increase the quantity produced. At the same time, the consumers will bid up the price because they will be competing to get the products in limited supply. Eventually the price increases and they settle at the equilibrium quantity and price. 
\subsection{The Efficient Equilibrium}
What's so special about the equilibrium quantity? It produces the eventual end state in a market. Eventually, this is how your market will be. There will be an equilibrium price and an equilibrium quantity for almost every good in the world. But what's special about the ending quantity and price? This is the quantity and the price that maximizes welfare. At that price and quantity, the consumer and producer's surplus which represents the welfare of the consumers and producers are maximized. At a quantity that is lower than the equilibrium quantity, the consumer surplus becomes less because the consumers don't get to consume as much as they want to. The triangle representing surplus actually shrinks to a small portion. At the same time, the producers don't get to produce as much either because the producers don't consume as much as well. Then producers don't get to sell that much and they lose a bit of their producers' surplus. What happens if the quantity is set to be more than the equilibrium quantity? Producers will produce a lot more, but the producers will only be able to sell at a higher price. That higher price is too much for the consumers to take so the consumers will only consume according to their demand curve - at a lower quantity. So eventually it goes back to the same scenario as when we chose a lower quantity than the equilibrium quantity. It leads to a suboptimal distribution of surplus. You can think of this in the same way as prices. If you set a price that is higher than the equilibrium price, the consumer surplus becomes lower because the consumers are not willing to buy at that high price. The producers don't get to sell as much as well because the consumers don't buy that much. The producer's surplus gets affected as well. If you set a lower price than the equilibrium quantity, the same thing happens. The producers are not willing to produce
 so much at the lower price, so they produce a smaller amount and get a smaller surplus. Same for the consumers because they don't get to consume so much. So the equilibrium price and quantity is the only price and quantity at which the consumers and the producers surplus is maximized.
\subsection{The Invisible Hand}
Let's talk about your most important lesson in economics. Imagine a country where you have to plan everything. You have to plan the price of markers, the price of laptops, the price of everything that goes into the production of laptops. That will be too much for a government to plan. That's why most socialist countries fail and most big governments fail because they just can't plan so much. So much goes on in an economy. Think, on the other hand, we have the market mechanism. Through the interaction of demand and supply, prices and quantity get determined. It does not happen through intervention, for example when the government dictates the price and quantity of rice. Instead the consumer talks to the producer and through the market, they interact and arrive at an equilibrium price. That's all that happens! And it happens for every single thing - things that go into your production chain and your final product that affects consumers. Without planning, the price and quantity get determined. Do remember that your equilibrium price and quantity - those are the optimal values that consumers and producers should consume at to maximize their welfare. Everybody's welfare is maximized without anyone having to do anything. The price tells everybody how much to consume, so it's almost like doing everything by doing nothing. That is one of the most important lessons you should take away from economics - if not the most important one. That is what Adam Smith meant by the invisible hand. It is like as if an invisible hand is guiding the economy, making sure that things are consumed at the right quantity and at the right price. 
\newpage
\section{Shifts}
\subsection{Demand Moves}
In real life, the demand curve and the supply curve never stays constant. It moves all the time. For example, oil. The demand for oil has increased. You should picture the demand curve moving up or to your right. Let's say that demand for oil has increased. The demand curve goes up, and they form a new intersection. What is the explanation then? A lot of students say that because the demand curve moved up, there is a new intersection and that gives a new equilibrium price and quantity. They are merely describing the movement. They are not explaining how the new equilibrium quantity and price came about. That will not suffice. What you should do instead is this. When the demand curve moves up, at the original equilibrium price and quantity, the quantity demanded is more than the quantity supplied because of the ne w demand curve. When the new quantity demanded is higher than the quantity supplied, the consumers bid up the price due to the shortage. They push up the price and when the producers see the higher price, they produce more in response. Eventually, we arrive at a higher price and a higher equilibrium quantity. This is how you should explain the movement of a demand curve.
\subsection{Supply Moves}
Let's talk about supply. Let's use the example of oil again and we found a new oil field. This increases the supply of oil. Then the supply curve moves to the right. At the original equilibrium price and quantity, you find that the quantity supplied is much higher than the quantity demanded. What then happens is that the suppliers notice the huge surplus because the quantity demanded is much lower than the quantity supplied. The producers will produce less and the price goes down according to the supply curves and eventually they reach a new equilibrium where there is absolutely no shortage. This is the new equilibrium price and quantity.
\subsection{Demand Moves More}
Most of the time, the demand and supply curves move together. Let's imagine the case where the demand curve moves more than the supply curve. What happens? The increasing supply curve will lower the price and increase the quantity supplied, but the demand curve increasing will increase the price. What happens eventually? When the demand curve increases more than the supply curve, then at the original equilibrium price, the quantity is demanded is much higher than the quantity supplied. Your consumers will outbid each other and increase the price. The producers, upon seeing this increase in price, will produce more. You end up with a new equilibrium price that is much higher and a equilibrium quantity that is higher than your original equilibrium quantity and price. This is because the demand curve moved more than the supply curve.
\subsection{Supply Moves More}
What if two curves move but the supply curve moves than the demand curve? In this case we look at your original equilibrium price and quantity. The quantity supplied will be higher than the quantity demanded. Producers will sell less at a lower price because they want to get rid of their surplus. The consumers buy more at a lower price. Even though the consumers buy more and the producers sell less, the new equilibrium quantity is still higher than the original equilibrium quantity because the supply curve moved more than the demand curve. So to illustrate this again, here is the original demand and supply curve. Here is the new demand and supply curve. The equilibrium price is lower and the quantity is higher. 
\subsection{What About Shape?}
Does the shape of the of these graphs affect the outcome? Of course they do! For example, if you have a supply curve moving and a demand curve that is almost entirely vertical. If I push the supply curve all the way to the right, what happens? The price dropped a lot but the quantity didn't move much. But if I change the demand curve to be somewhat flat. And I shift the supply curve all the way to the right again. What we then observe is that the price didn't change much but the quantity changed a lot. This is the concept that we will be exploring in the next topic. 
\end{document}