\RequirePackage{../../dominatrix}

\title{Market Structure}
\author{\large by Linan Qiu}
\date{\small correct as of \today}
\begin{document}
\maketitle
\tableofcontents
\section{Features of the Market Structures}
\subsection{Structure Conduct Performance}
When we look at different market structures, we use the structure conduct performance approach. Let’s use an analogy. When you want to decide what car to buy, there are some things you look out for. First, you look at the physical characteristics. What kind of motor does it have? How many seaters? That is what we call structure. The physical structure of the car causes it to behave in a certain way. We call that conduct. For example, having a large motor might increase fuel consumption and speed. Eventually you make a judgement on whether the car is good or not based on several factors - is it economical? Is it comfortable enough. That is what we call performance. So again, physical characteristics, the structure, causes it to behave in a certain way, which is what we call conduct. Then we evaluate the conduct. That’s what we call performance. Same for market structures. We’ll talk about structure first. We’ll leave conduct and performance for the last lesson in this topic. So we divide industries into categories according to the degree of competition that exists between the firms in the industry. There are 4 such categories. At one extreme is perfect competition, where there are many firms competing. Each firm is so small relative to the whole industry that it has no power to influence price. It is a price taker. At one other extreme is monopoly, where there is just one firm in the industry. You should be familiar with the term from the board game! There is no competition from within the industry, because well, you are the industry all by yourself! In the middle comes monopolistic competition, which involves quite a lot of firms competing and where there is freedom for new firms to enter the industry and oligopoly which involves only a few firms and where entry of new firms is restricted. To distinguish them more precisely, we use three indicators - Number of firms, freedom of entry and the nature of the product. These will be explored in the next segment.
\subsection{Number of Firms}
From the spectrum of perfect competition to monopoly, there is a clear decrease in the number of firms. So it is a very important indicator of the type of market. Evidently, to measure the number of firms in the market, we well simply count the number of firms. But does that always work? Let’s say I have an industry where one company takes up 95\% of the market share, and 100 other companies 5\%. Sure I have 101 companies in the market, but does that mean it is perfectly competitive? If anything it should be monopoly! So we look at something called the 4 firm concentration ratio. We take the 4 firms in the industry with the highest market share, and we add them up. Say we look at internet browsers. Google Chrome takes up 32\%. Internet Explorer takes up another 32\%. Firefox takes up 24\% and Safari takes up 7\%. Total, the 4 firm concentration ratio is 95\%. You can see that 4 systems dominate the browser market, and it is pretty oligopolistic. We can look at 5 firm ratios or 3 firm ratios as well, depending on the amount of data you have. Right now we are simply adding up concentration ratios. There are many more complex ways of measuring concentration ratios, but you’ll only look at them if you go into economics and econometrics in university. 
\subsection{Freedom of Entry}
Here we investigate how free is a firm able to enter the industry or exit the industry. Some schools might term this factor Barriers to entry. Same thing. If your barrier to entry is high, you don’t have much freedom of entry. What barriers are there? There are real and artificial barriers to entry. Real barrier to entry can be a huge sunk cost. For example, in order to enter the petrochemicals industry, or the aviation industry, one has to invest a lot in capital equipment. In the case of petrochemicals, it is the huge crackers downstream and the drills upstream. For aviation, you have to buy many planes! And that is not cheap. So not everyone can just walk into the aviation industry and declare himself a player. Artificial barriers to entry are things like patents or legislation. For example, if a pharmaceutical company holds a patent on a certain kind of medicine, even if you are able to invest in a plant and get researchers, you will not be able to enter the market for that particular medicine because you’ll be violating the patent. That is unless you buy the patent from the original pharmaceutical company. In some countries, some state owned company might be the only one allowed to operate in a particular industry. For example, for a long time, Singtel was the only Telecoms operator in Singapore because for the early part of Singapore’s history, Singtel was the only authorized Telco operator! that is a barrier to entry as well. These barriers affect the freedom of entry into the industry. What about getting out of the industry? Well if you have a huge sunk cost, that will deter you from exiting the industry as and when you like. You have a huge factory there! You can’t just quit! These barriers affect the freedom of entry in and out of an industry.
\subsection{Nature of Product}
Nature of product is pretty straightforward. Do all the firms produce an identical product, or do firms produce their own particular brand or model or variety? In perfectly competitive markets, all the firms produce the same item. In monopolistic competition, we allow for a little variation. In oligopolies, things are a little more complicated and we might have differentiated or undifferentiated products. You’ll understand why in a while. Of course for monopolies, by definition, you’re the only guy in the market! So of course your product is unique.
\subsection{Market Power}
The combination of these 3 factors give rise to varying degrees of market power. What does this mean? Well, this refers to the firm’s degree of control over price. Is the firm a price taker or can it choose its price, and if so how will changing its price affects its profits? Sounds counter intuitive to you? Well think of it this way. If you’re in a perfectly competitive market, you’re just a small firm, you cannot set your own price! If you set your price above the market price, people will just go to someone else to buy their goods. And since you sell the same good as everyone else, and there are so many of everyone else, you won’t get anything! You won’t want to set your price below the market price either, for why will you want to earn less? So you take the market price. But if you’re a monopoly, you get to set your price! Because there isn’t this “market” dictating you. So Market power refers to the amount of power you have over prices. Market power occurs when there’s less competition, so perfectly competitive firms have zero market power, whereas monopolies have lots of market power. A firm with high market power has a very price inelastic demand curve, because, for example for a monopoly, we have no choice but to buy things from that monopoly! There are no substitutes. Compare that to a firm which is in a perfectly competitive market. There are so many substitutes all over the place! It’s demand curve is hence highly price elastic.
\newpage
\section{Perfect Competition}
\subsection{What is perfectly competitive}
First let’s look at the structure of perfectly competitive firms. There is complete freedom of entry into the industry for new firms. Existing firms are unable to stop new firms from setting up businesses. Setting up a business takes time however, hence exit and entry of firms only applies in the long run. All firms produce an identical product, hence the product is homogenous. There is therefore no branding, no advertising. No innovation as well. There are many many firms in the market, so each single firm produces an insignificantly small portion of the total industry supply. What does this result in for conduct? Since there are so many firms in the industry, and each individual firm is insignificant, each individual firm has no power whatsoever to affect the price of the product! Say among a group of 1000 small producers, 1 producer decides to sell at a price higher than the rest. What will happen? The consumers will just go to the other 999 producers who are selling the EXACT SAME things at the original price. The 1 producer will get nothing. So he can’t even raise his price by a single cent. He also has no incentive to lower his price because he can sell at the market price! So actually, the demand curve of the single competitive firm is a horizontal demand curve. At every quantity, it can only sell at one single price. That is the demand curve of the firm, it’s average revenue too. This price is determined by the interaction of demand and supply in the whole market. Here we need to differentiate between the curve of a firm and that of a market. The firm faces a horizontal demand curve, but the industry as a whole, all 1000 producers, still face a downward sloping demand curve. That’s because you’re looking at the industry as a whole. To an individual small producer, the amount of demand he can meet is very little. He can’t sell that much stuff. But when you look at the entire industry and all of the consumers, it is actually a downward sloping demand curve. It is just that the producer is so infinitely small and insignificant that he faces a horizontal demand curve. The interaction of the market demand and supply sets a price, and the competitive firm takes that price, and makes it its horizontal demand curve. Since the firm only earns that amount determined by the market for every additional unit of good sold, this horizontal line is also its MR.
\subsection{Review: Short Run vs Long Run}
Remember our previous lesson on short run vs long run? Here we revisit it. In the short run, the number of firms is fixed. Depending on its cost and revenue, a firm might be making supernormal profits, normal profits or subnormal profits. I hope you still remember what these terms mean. In the short run, it may continue to do so. In the long run however, the level of profits affects the entry and exit from the industry. Firms are allowed to exit and enter the industry as well. If there are supernormal profits in the industry, new firms will be attracted into the industry. Whereas if losses are being made, firms will leave. 
\subsection{Short Run Equilibrium of the Firm}
Let’s start by drawing two diagrams. We have here the short run equilibrium for both industry and firm under perfect competition. Both parts of the diagram have the same scale on the vertical axis - that of price. However, on the horizontal axes, they have totally different scales. For example, if the horizontal axis for the firm were measured in, say thousands of units, the horizontal axis for the whole industry might be measured in millions or tens of millions of units, depending on the number of firms in the industry. Let’s them examine the determination of price output and profit. The price is determined in the industry by the intersection of the demand and supply. Remember how the equilibrium is obtained in the industry? At a price higher than this, there will be a surplus, and prices will be pushed downwards because producers want to get rid of the surplus hence lower prices. At a price lower than this, consumers bid up the price due to a shortage, and hence prices only stop changing at the equilibrium price. the firm faces a horizontal demand curve at this price. It can sell all it can produce at the market price P, but nothing at a price above P. The firm will maximize profit where marginal cost equals marginal revenue, at an output of Q. Since the firm has only one price at every quantity, the marginal revenue will equal price. Hence, the firm produces at MC = MR = AR = DD = P. But what about Profit? There are three possible scenarios we can illustrate in the short run. If the firm’s AC dips below the AR curve at the profit maximizing point Q, the firm will earn supernormal profit. Supernormal profit per unit at Q is the vertical difference between AR and AC at Q. And the total supernormal profit is simply the vertical distance times the quantity produced, which gives you this rectangle here. What happens if the firm cannot make a profit at any level of output? This situation would occur if the AC is above the AR at all points. This is illustrated here. In this case, the firm still produces at the profit maximizing price where MC = MR. In this case however, since profit is negative, we are producing at a loss minizming point. The loss is represented by this rectangle, and the firm is earning subnormal profits. Of course, the firm can make normal profits. That’s when it’s AC is equal to its AR at the profit maximizing point. So there are three scenarios in the short run. In the short run, the firm’s supply curve, not the industry’s, will be its marginal cost curve. Why? A supply curve shows how much will be supplied at each price. It relates quantity to price. The marginal cost curve relates quantity to marginal cost, but under perfect competition, P = MR and MR = MC, hence if the firm is profit maximizing, which we usually assume it does, then P must always equal MC. Hence the supply curve and the MC curve will always follow the same line. What about the industry? The industry supply curve is simply the sum of the individual firm supply curves. Take for example this firm. It is able to supply 30 units of apples at \$1 each. If the industry has a thousand such firms, then the industry supply curve, at \$1, will be able to supply 30,000 apples. This is called a horizontal sum.
\subsection{Long Run Equilibrium of the Firm}
Imagine you are a guy who is thinking of setting up a company, but you don’t know which industry to go. Let’s say you saw an industry where there is a firm which makes supernormal profits. You’d think to yourself, “hmm how about I go there as well!” And because the market is perfectly competitive, there are no barriers to entry hence you can just step into the industry as and when you like. And that’s pretty much what happens when you have perfect competition in the long run, where firms can exit and enter. If firms are making supernormal profits, new firms will be attracted into the industry. What then happens can be observed on the industry demand and supply curve. Supply increases due to new entrants, and we settle on a lower equilibrium price. When will it stop? Well people will stop entering when there are no longer any supernormal profits. And that happens when the P is equal to the AC of a firm in the perfectly competitive market. That’s how we arrive at the new price in the market. What about subnormal profits? Almost the same thing. When there are subnormal profits, firms which are losing money exit the market. Since there is no impediments to entry and exit, this is done very easily! These firms exit and the supply curve shifts to the left. At every price, a lower quantity is supplied. When does this stop? Again, when profits become normal again. And that happens when P = AC, and here we arrive at our new industry equilibrium. 
\subsection{Behavior of Perfectly Competitive Firms}
Perfectly competitive firms are hence price takers. They take the price from the market demand and supply curves, themselves having no say over their price. This causes them to have zero market power. This is the most significant behavior of a perfectly competitive firm - a price taker. Other characteristics include the fact that perfectly competitive firms rarely have any significant economies of scale, unless the firm’s optimum size - the Minimum Efficient Scale - is very low. There is no innovation in a perfectly competitive firm as well - everyone sells the same stuff. So is there ever any real life example of a perfectly competitive firm? Many markets come close, but the only very real approximation is that of the stock market. Every stock is the same, and as a small shareholder you cannot influence the price in the market! Then again, there are people like Warren Buffet. 
\subsection{Perfect is the best?}
Perfect in perfect competition doesn’t mean the best! In fact, you’ll see later on that there are many disadvantages of being a perfectly competitive firm, and many advantages of being a monopoly. Counter intuitive isn’t it? Here perfect simply refers to being absolutely, completely competitive. That’s all! It’s not a judgement or in structure conduct performance terms, a performance evaluation.
\newpage
\section{Monopoly}
\subsection{What is a Monopoly}
You have my permission to go “What the hell Linan? Why are you asking me what is a monopoly? A monopoly exists when there is only one firm in the industry!” That’s a valid question, but things are not always as simple as they look. A textile company may have monopoly on certain types of fabric, but it does not have a monopoly on fabrics in general! The consumer can buy alternative fabrics to those supplied by the company. A pharmaceutical company may have a monopoly of a certain drug, but there may be alternative drugs for treating a particular illness. So it depends on how narrowly you define the industry! Is there a guideline? Not really. It is arbitrary how you define an industry. But what’s more important is the amount of monopoly power it has and that depends on the closeness of substitutes produced by rival industries. Singpost has a monopoly over delivery of letters, at least for normal delivery, in Singapore. But it faces competition in communications from phones, faxes and emails! Monopolies have very high barriers to entry, There is only one firm in a monopoly, and since there is only one firm, the product is usually unique. This causes monopolies to be able to set prices, instead of being a price taker like a perfectly competitive firm. Don’t worry we’ll take you through this in the next few checkpoints.
\subsection{Barrier to Entry}
How do monopolies continue being monopolies and keep earning supernormal profits? Remember in the case of a perfectly competitive firm, if an external party sees supernormal profits, it will enter the industry and reduce the supernormal profits! So retaining supernormal profits depends on the barrier to entry of new firm. There are real barriers to entry, and there are artificial barriers to entry. Of real barriers to entry, there are certain types. For example, if you need to be the size of ExxonMobil to gain significantly economies of scale in the petrochemicals industry, then not many people can attain that size without a significant amount (hundreds of billions) of capital. What about artificial barriers to entry? Well for example, Apple might seem to have a monopoly over Macs. Is there really that much of a difference between a Windows and a Mac? Some people might scream yes, but at the end of the day, they’re both computers! But consumers associates the product with the brand, and it will be very difficult for another new firm to break into that market. At the same time, Apple has ownership over patents that prevents people from coming into the market. It will take legal action against other people that threaten to copy some of its products, like it did to Samsung in 2012. In other cases, if a firm governs the supply of vital inputs, for example by owning the sole supplier of a component part, it can deny access to these inputs to potential rivals. On a world scale, the de Beers company has a monopoly in fine diamonds because pretty much all diamond producers market their diamonds through de Beers. An established monopolist can probably sustain losses for longer than a new entrant. Thus it can start a price war, mount massive advertising campaigns and offer an attractive after sales service, introduce new brands to compete with the new entrants and so on. These barriers to entry prevents the easy entry of new firms into the industry, and allows the monopoly to continue making profits even in the long run!
\subsection{Equilibrium Price and Output}
Hence in the short run and in the long run, the monopoly’s curves are the same! Let’s start drawing. The firm’s demand curve is also the industry demand curve, because well you only have that one firm in the industry! Compared with other market structures, demand under the monopoly can be rather price inelastic. Remember we were talking about market power and how it translates into price elasticity? The more market power you have, the more price inelastic your demand curve is because there should not be any substitutes for your product if you’re a powerful monopoly. In this case, the monopolist an raise its price and consumers have no alternative firm in the industry to go to. They either pay the higher price or go without the good altogether, hence explaining the lower quantity demanded. Hence unlike the firm in perfect competition, the monopoly firm is a price setter. It can choose what price to charge. Nevertheless, it is still constrained by its demand curve. Hence it can only choose a price, or a quantity, because choosing one will determine the other. It can’t choose both at the same time! Say an arbitrary quantity and price here. The MR of the monopolist is always double the gradient of the AR. This is a mathematical relationship that can be proved rather easily with a line of a downward sloping gradient. The more inelastic your demand curve is, the larger the gap between your MR and your AR. For example, this is a monopoly with a more price inelastic demand curve. A monopolist will maximize profits where MR = MC. So profits is maximized at Q. then we draw in your AC this time showing that the firm is earning supernormal profits, as a monopoly usually does. The supernormal profit obtained is the gap the AR has over the AC, times the quantity. Hence giving you this shaded rectangle These profits tend to be larger when the curve is more inelastic. Let me show you. here we have a monopoly with a more inelastic demand curve, and you can see that because the gap between the MR and the AR are larger, the firm actually earns a higher profit! But common sensically when does that happen? Well when the firm has a more price inelastic demand curve, it has less substitutes! the firm has more market power! So due to the lack of substitutes, less consumers will switch away from the monopolist when a price increase happens. Now in the long run, these supernormal profits will not be competed away. Is it possible for monopolies to make losses? Of course, when you draw the graph like this. Of course, good monopolies don’t allow that to happen. This monopoly probably needs to lower its costs a little. Try drawing a monopoly graph where there is only normal profit.
\subsection{Behavior of Monopolies}
The high barriers to entry gives the monopoly large market power, which in turn gives the firm the ability to set prices. The price setting ability of a monopoly is the hallmark of a monopoly, hence we name it as the conduct of a monopoly . So the structure of being the only firm, having a unique product and having very low freedom of entry in the industry gives rise to this conduct here - price setting behavior.
\subsection{Natural Monopolies}
So we’ve been painting monopolies in a bad light right? But not all monopolies WANT to be monopolies! There is an interesting case of the natural monopoly. There are some firms with a very very large minimum efficient scale. So if the monopoly experience substantial economies of scale only at a very large scale, the industry may not be able to support more than one producer. Let’s take a look at this diagram Let's say that the industry faces a demand curve, or the AR curve to a monopolist, as AR1 or DD1. Then we draw in your LRAC. The monopolies can gain supernormal profit at any output between Q0 and Q1. If there were two firms however, each charging same price and supplying half the industry output, they will each have a demand curve DD2 or AR2. In this case, each of them won’t be able to make even a normal profit at any quantity! There is no price that would allow them to cover costs! This case is known as the Natural monopoly. It is particularly likely if the market is small. For example, electricity transmission grids in cities are usually operated by only one company, because there is no need to have two sets of electricity grids leading to your house and asking you to choose one. There is wasteful duplication, and both firms will make losses! 
\newpage
\section{Monopolistic Competition}
\subsection{What is Monopolistic Competition?}
Now let’s look at something in the middle, more towards the side of perfect competition. It is best understood as the situation where there are lots of firms competing, but each firm does nevertheless have some degree of market power. Hence the term “monopolistic”competition. Each firm has some choice over what price to charge for its products. So let’s look at the the structure of the market. There are many firms. As a result, each firm has an insignificantly small share of the market and therefore its actions are unlikely to affect rivals to any great extent. This also means that when each firm makes its decisions it does not have to worry about how its rivals will react. It assumes that what its rivals choose to do will not be influenced by what it does. This is known as the assumption of independence. This is a great difference between oligopolies and monopolistic competition. In oligopolies, the action of rivals do affect them. There is also freedom of entry of new firms into the industry. If any firm wants to set up business in this market, it is very free to do so. Hence in these two structures, it is very similar to a perfectly competitive market. However, unlike perfect competition, each firm produces a product or service that is in some small way different from its rivals. As a result, it can raise its price without losing all its customers. IT’s demand curve is usually elastic, but still downward sloping. It is still elastic because there is only a small differentiation, not large. So there is still a certain amount of substitutes. this is known as product differentiation, where the firm innovates a little and provides goods that are a little different. Places like restaurants, hairdressers, builders are all examples of monopolistic competition. They sell largely the same product, but there are small differences.
\subsection{Short Run Equilibrium of Monopolistic Competition}
As with other market structures, profits are maximized at output where MC = MR. So the diagram will be the same as for the monopolist, except that the AR and MR curves will be a lot more elastic. As with perfect competition, it is possible for the monopolistically competitive firm to make supernormal profit in the short run. So you can see how it is shaded here where we find the profit maximizing quantity at MC = MR, we extrapolate up, and find the distance between the AR and the AC. The AR length times the quantity gives us the revenue. The AC length times the quantity gives us the total cost. We take total revenue minus total cost we see that we now have the profit of the firm. Just how much profit the firm will make in the short run depends on the strength of demand: the position and elasticity of the demand curve. The further to the right the demand curve is relative to the average cost curve, the less elastic the demand curve is and the more the firm’s short run profit. Thus a firm facing little competition and whose product is considerably differentiated from that of its rivals may be able to earn considerable short run profits.
\subsection{Long Run Equilibrium of Monopolistic Competition}
Now do you remember what happens when you consider the long run in perfect competition? If firms are making supernormal profits, new firm will enter the industry in the long run. similarly in monopolistic competition, if typical firms are making supernormal profit, new firms will enter the industry in the long run because there are almost no barriers to entry. As they do, they will take some of the customers away from established firms. The demand for the established firms will therefore fall. The AR will shift to the left, and will continue doing so as long as supernormal profits remain and thus new firms continue entering. Long run equilibrium is reached only when normal profits remain, when there is no further incentive for new firms to enter. This is shown here . It will only happen when at the profit maximizing point, the LRAC equals to the firm’s demand curve, hence making sure that the firm only earns normal profits.
\subsection{Behavior of the Firm - Non Price Competition}
Monopolistic competitive firms tend not to compete based on price, but instead uses non price competition. It involves two major elements - product development and advertising. The major aims of product development are to produce a product that will sell well. One that is high in demand and that is different from rival’s demand. This will give the monopolistic firm a demand curve that is both high, and highly inelastic. Some shops choose to do this via providing personal service, late opening, certain lines stocked and so on. The major aim of advertising is to sell the product. This can be achieved not only by informing the consumer of the product’s existence and availability but also by deliberately trying to persuade consumers to purchase the good. Successfully advertising also increases demand and eventually cause a more inelastic demand curve. This is a characteristic conduct of the monopolistic competitive firm.
\newpage
\section{Oligopoly}
\subsection{What is an oligopoly?}
Oligopoly happens when there are just a few firms between them sharing a huge proportion of the industry. A good way to identify them is via the four or five firm concentration ratio that we talked about at the start of this topic. So let’s talk about structure . There are few firms in the industry like I’ve said . There are also very high barriers to entry, or very low freedom of entry. These are similar to the ones described in monopoly, both artificial and real barriers to entry. Some firms may produce an identical product, for example in the petroleum industry. Most oligopolists, however, produce differentiated products like car, soap powder, soft drinks and so on. Much of the competition between such oligopolists is in terms of marketing their particular brand. 
\subsection{Competition and Collusion}
These structural characteristics causes oligopolists to exhibit a rather interesting behavior. They are interdependent . Because there are only so few firms under oligopoly, each has to take account of the others. this means that they are mutually dependent. They are interdependent. Each firm is affected by its rivals’ actions. If a firm changes the price or the specification of its product, for example, or the amount of its advertising, the sales of its rivals will be affected! The rivals may then respond by changing their price, specification or advertising. No firm can afford to ignore the actions and reactions of other firms in the industry! And this causes two kinds of behavior to emerge - that if either competition of collusion. The interdependence of firms may make them wish to collude with each other. They want to gang together and act as if they were a monopoly and maximize industry profits. at the same time, they will be tempted to compete with each other, either in price or non price factors, to gain a bigger share of industry profits for themselves. It’s really like a bunch of kids playing on the playground. These two policies are incompatible! the more fiercely the firms compete to gain a bigger share of industry profits, the smaller these industry profits will become! For example, price competition will drive down the average industry price, while competition through advertising will raise industry costs. Either way, industry profits fall. 
\subsection{Equilibrium under Collusive Oligopoly}
When firms under oligopoly engage in collusion, they may agree on prices, marker share and advertising expenditure etc. Such collusion reduces the uncertainty they face. It reduces the fear of engaging in competitive price cutting or retaliation or advertising. So a formal collusion is called a cartel! The cartel will maximize profits like as if it’s a monopoly. Then in that case, we can analyze it exactly like a monopoly The total market demand curve becomes the collective market demand of the firms. MC curve is the horizontal sum of the MC curves of its members. Profits are maximized at Q, where MC = MR. The cartel must therefore set a price of P. Having agreed on the cartel price, the members may then compete against one another using non price competition. Or perhaps they may not compete at all . They might be given a quota. In many countries, cartels are illegal! This is seen by the government as a means of driving up prices and profits, and thereby as being against public interest. Where open collusion is illegal, firms may break the law. Alternatively, they do it tacitly. One form of tacit collusion is where firms keep to the price set by an established leader. This is called dominant firm price leadership. Or sometimes, it may not be the dominant firm, just the one that has proved to be the most reliable one to follow. That’s called Barometric firm price leadership. An alternative to following a leader is following simple rules of thumb. These are easier than price leadership, and prevents immediate outbreak, maintaining profits in the long term. For example they may use the cost plus method to price themselves. 
\subsection{Factors Favoring Collusion}
So when does the conditions in a market favor collusion? Well, when there are very few firms which all know each other well, they collude easily. When they are not very secretive about cost and production methods. When they have similar production methods and average costs, and thus likely to want to change their prices at the same time and by the same percentage. When they sell similar products, it’s easier to agree on things as well. It also helps when there is a dominant firm, and when there are significant barriers to entry therefore little fear of disruption by new firms. The market being stable is important as well, because it prevents oligopolists who hide their own price increases as “market conditions and fluctuation”. However, sometimes, oligopolists choose not to collude if these conditions are not really met. 
\subsection{Breakdown of Collusion}
In some oligopolies, there may be only a few factors met that favors collusion. In such cases, price competition becomes greater. Even if there is collusion, there will still be temptation for hte individual oligopolists to cheat by cutting prices or by selling more than their allotted quota. Now when this happens, other members in the cartel might retaliate, resulting in a price war. Prices will then fall, and the cartel can very well break up. THis is rather like a war game, where generals make allies or enemies, and tries to outwit their opponents. 
\subsection{Competitive Oligopoly - Game Theory}
Now the simplest way to analyse competing oligopolies is to use game theory. Game theory is the study of alternative strategies that players in the game may choose to adopt based on their assumption about other rivals’ behaviors. This is the perfect theory for oligopolists which respond based on the behavior of other firms! The interdependence of oligopolists allows us to use this. Now let’s look at some of the scenarios we can map out.
\subsection{Dominant Strategies}
The simplest case is where there are just two firms with identical costs, products and demand. They are both considering which two alternative prices to charge. So let’s look at this table. We have Coke and Pepsi Let’s assume that both firms are charging two dollars right now. So they belong to the box over here. And they are each making a profit of 10 million, giving a total industry profit of 20 million. Now assume they are both independently considering reducing their price to \$1.80. First, they must take into account what their rival might do and how this will affect them. Let’s consider Pepsi. In our example, we assume that there are just two things that its rival, Coke, might do. Either Coke can cut its price to \$1.80 as well, or it could leave its price at \$2. If Pepsi remains at \$2, but Coke drops to \$1.80, Pepsi will suffer a huge loss, making only \$5million. Coke will due to the lower price and enjoy higher revenue, making \$12m. If pepsi lowers to \$1.80, but Coke remains at \$2, then same thing, though this time Pepsi gets a larger revenue of \$12m, and Coke only gets \$5m. If both of them lowers to \$1.80, they both earn less, and earn only \$8m each. What should Pepsi do? One alternative is to go for the cautious approach and think of the worst thing that its rival can do. If Pepsi kept its price at \$2, the worst thing for Pepsi will be if its rival Coke cut its price. This is shown in the cell at the lower left hand corner. Pepsi then earns only \$5m. If however, pepsi cut its price to \$1.80 the worst thing that can happen to it is if Coke cuts as well, which makes Pepsi accept a revenue of \$8m. So Pepsi chooses to cut. If we make the same argument for Coke as well, then coke, in order to minimize losses, will cut its price to \$1.80 as well. Then both of them will end up in the lower right hand box, which allows each of them to earn \$8m. This is called a maximin strategy - maximizing its minimum possible loss. Another way is to go for the optimistic approach, and assume that your rivals will act in the way most favorable to you. In this case, this again means that Pepsi will cut its price because Pepsi will assume that Coke keeps its price the same. Remember it is going for the optimistic approach. So Pepsi will lower its price. Same for coke! So both of them will end up here again. Given that in this “game” both approaches, both pessimistic and optimistic, will lead to the same strategy, this is known as a dominant strategy game This equilibrium is called a Nash Equilibrium. 
\subsection{Prisoner’s Dilemma}
But you can see from the previous example that both Pepsi and coke will be tempted to lower prices, and they both end up earning a lower profit! \$8m instead of \$10m each. Thus a collusion, rather than competing each other, would have benefited both. Yet both will be tempted to cheat and cut prices. This is known as the prisoner’s dilemma, where two or more firms by attempting independently to choose the best strategy for whatever the others are likely to do, end up in a worse position if they had cooperated in the first place. You’ll see this prisoner’s dilemma occur a lot in real life! Whenever you go “damn we should have cooperated in the first place!”, this happens. For example, When people go to some public event such as a concert or a match, they often stand in order to get a better view. But once people start standing, everyone is likely to do so. AFter all, if they stayed sitting, they would not see at all. In this equilibrium, most people are worse off since except for tall people, their view is likely to be worse and they lose the comfort of sitting down. Another example is when firms spend a lot on advertising! If they are aggressive, they do so to get ahead of their rivals. If they are cautious, they do so in case their rivals increase advertising. Although in both cases it may be in the individual firm’s best interests to increase advertising, the resulting Nash equilibrium is likely to be one of excessive advertising: the total spend on advertising by all firms is not recouped in the additional sales. 
\subsection{No Dominant Strategy}
More complex games can be devised with more than two firms, many alternative prices, differentiated products and non price competition. Or sometimes, they might do funny things like threats and promises as well. An oligopolist will make a threat or promise that it will act in a certain way. As long as the threat or promise is credible, the firm can gain and it will influence the rival’s behavior. Take for example a large oil company, such as Shell, states that it will match the price charged by any competitor within a given radius. Assume that the competitors believe this price promise. Few other firms will lower their price to be lower than Shell’s, knowing that Shell will match their promise and eventually result in a price war and take away whatever benefits they have gained in lower prices. But the threat has to be credible! Here the threat is credible because Shell is a large company, and thus even if it lowers prices and suffers losses, it will be able to tank the loss longer than most companies can, driving the opponents out of the market before they can even recover. Sometimes, some firms may move first, earlier than others. In this case it has a first mover advantage. So, game theory is a rather complex branch that mathematicians spend lots of time on. But we can assess the usefulness of game theory. The advantage is that the firm does not need to know which response its rivals will make. But it needs to measure the effect of each possible response. This will be impossible in most situations! In real life, most oligopolists just compete, then realize they are not making a lot of money, so they collude. After a while, someone gets itchy and competes again, spoiling the collusion. Then they collude again. It’s very much like a bunch of kids playing on a playground. 
\end{document}