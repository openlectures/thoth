\RequirePackage{../../dominatrix}

\title{Supply Side Policy}
\author{\large by Lee Wen Qiang}
\date{\small correct as of \today}
\begin{document}
\maketitle
\tableofcontents
\section{Market-Oriented Supply Side Policies}
\subsection{Introduction to Supply Side Policy}
The types of macroeconomic policies that you have learnt so far deals with the aggregate demand of the economy. However, there is a different class of policies which deals with the aggregate supply of the economy, named supply side policy.

The primary aim of supply side policies is to improve the long term potential economic growth of an economy. This is to allow sustainable economic growth, which is one of the main macroaims of a government. Sustainable economic growth is where there is actual economic growth but the general price level in an economy is maintained at a low level. (draw graph) As can be seen from the graph, as the aggregate demand continually increases, the general price level rises exponentially with actual economic growth increasing at a slower pace near to the full employment level. However, with an outward shift of the aggregate supply, not only can there be real economic growth, but the general price level will also fall. To achieve such sustainable economic growth, supply side policies generally work by improving the efficiency and productivity of the economy.
\subsection{Introduction to Policies for Product Markets}
The first broad category of supply side policy would be the market oriented policy. This category can be further split into policies for product markets and for labour markets. In this section, I will focus on policies for the product markets.

As you may have guessed, product markets are simply where all types of goods and services are traded, be it iPhones or banking services. Policies targeting product markets tend to be pro-competition policies, which mean that they aim to encourage firms to compete. As you know, competition spurs firms to be less productively inefficient and this will help to increase output for the same unit of goods, thus increasing the long term potential economic growth of the economy. Furthermore, by reducing the monopoly power of firms, firms will be less allocatively inefficient and the economy will use scarce resources more efficiently.
\subsection{Privitization}
The first type of product market policy is privatisation. This refers to the sale of state-owned enterprises to the private sector. This is done in the belief that the private sector is subjected to greater market forces since they are not funded by the government and has to be efficient to survive. Therefore, firms will try to reduce productive inefficiency, which will help to increase aggregate supply by increasing output per unit of input. Also, the private sector is more responsive to the price mechanism since they need to adapt quickly to changes in market conditions in order to maximise profits as compared to state-owned enterprises which do not have profit maximizing motives. Such a characteristic will allow firms to be less allocatively inefficient since they will allocate resources more closely to changes in market conditions. An example of privatisation in Singapore would be the privatisation of POSB to DBS Bank.
\subsection{Deregulation and Competition Policies}
The second type of policy is deregulation and competition polices. Deregulation involves reducing the red tape involved in setting up businesses. By providing a conducive environment for the setting up of businesses, firms will be attracted to invest in Singapore, thus increasing aggregate supply as the productive capacity increases. Competition policies prevent the prevalence of monopolies either by preventing local firms from forming monopolies or by opening up the market to foreign competition. This will force firms to be less productively inefficient as they have to compete with each other. Such policies in Singapore include the Competition Act which prevents firms from merging if there is significant reduction in competition without resulting in greater efficiency, innovation or quality. Another example would be free trade agreements to promote foreign competition.
\subsection{Set-Up SMEs}
The third policy would be setting up of small and medium enterprises (SMEs) can be encouraged. In Singapore, SMEs are defined to be firms with an annual turnover of not more than \$100 million and an employment size of not more than 200 people. SMEs are seen in many economies to be the drivers of innovation. The effect of this policy is similar to deregulation by allowing more firms to enter the market to increase supply. To promote growth of SMEs, Singapore has grants to SMEs for R\&D and skills training of workers.
\section{Labor-Oriented Supply Side Policies}
\subsection{Introduction to Policies for Labour Markets}
The second part of market oriented policies is concerned with the labour market, where instead of goods, labour service is being traded. Supply side policies which target the labour market aim to manage wage increments so that they will be in line with labour productivity to curb wage-push inflation. Another aim is to increase the supply of labour so that the productive capacity of the economy is enhanced.
\subsection{Reducing the Power of Trade Unions}
The first of labour market policy would be to reduce the power of trade unions. This is because some countries have militant trade unions which may push for wage increases exceeding productivity growth, leading to increase in cost of production for firms. Thus, firms may have to cut back on employment and cost push inflation will set in. In addition, by reducing the power of trade unions, it would help prevent strikes from occurring, thus attracting more investment since the economy will be more stable. In Singapore, we have the tripartite policy, which ensures a harmonious relationship among the employers (represented by the Singapore Employers Federation), the workers (represented by NTUC) and the government (represented by the Ministry of Manpower). Representatives from all 3 parties come together to discuss wage changes in the National Wage Council. This is so that wages will move in line with productivity growth to curb wage-push inflation and attract foreign investment, thus increasing Singapore's aggregate supply.
\subsection{Income Policy}
The second labour market policy would be income policies, which actually includes a variety of policies such as wage guides, flexible wages and wage freeze. What all these policies aim to achieve is wage increases in line with productivity growth. This means that the growth in output per unit of labour is faster than the increases in wages, thus from the firms' perspective, cost of production has fallen. In Singapore, there was a 2-year wage freeze for civil servants as well as a 5\% reduction in employer CPF contributions during the 1985 recession so that companies will not fire workers. However, clear communication must be present between employers and workers so that everyone understands the rationale for the policy. Also, employers must be willing to reinstate or increase wages when the economy recovers so that workers be more receptive of such policies.
\subsection{Tax Reform}
The third labour market policy is tax reform, which mainly refers to the reduction in income or corporate tax rate. With lower income tax, working becomes more attractive since the rewards are higher. Thus, workers are willing to work longer and some may even come out of retirement. In addition, lower income tax will attract foreign talent. Lower corporate tax will increase after-tax profits for firms, so investment becomes more attractive for them. Thus, tax reform can increase both the quantity and quality of the labour force as well as attract more foreign investment. Singapore's corporate tax rate of 17\% is one of the lowest in the world, although other factors such as a stable political environment helps to attract investment as well.
\subsection{Cutting Social and Welfare Programmes}
Finally, another labour market policy is cutting social and welfare programmes. This is because such welfare programmes have eroded the incentives to work since the unemployed are able to receive monetary help. Thus, by cutting back on such programmes, it will encourage people to look harder for jobs and be less picky about the jobs they take up. This will help to increase the supply of labour in an economy. In Singapore, unemployment benefits are minimal so that reliance on the government is reduced.
\section{Interventionist Supply Side Policies}
\subsection{Difference from Market-Oriented Policies}
As you have learnt earlier, supply side policy has two broad categories. Now that you have learnt about market oriented policies, the second category would be interventionist supply-side policies. The difference between the two different broad categories of supply side policies is that market oriented policies aim to increase aggregate supply through market forces by providing incentives and rewarding work effort. The role of the government is thus reduced. On the other hand, interventionist supply side policies advocate more government intervention to encourage investment in R\&D and skills training.

The rationale for having interventionist policies is that the free market may fail to provide sufficient incentive for R\&D and skills training due to the presence of externalities. For example, R\&D in general science and technology may benefit many other firms not involved in the research, thus firms will engage in less than socially optimal level of R\&D as they do not consider the marginal external benefits. Similarly, skills training may yield benefit for society when workers leave their original firm to work in other firms, bringing with them their skill sets learnt from the skills training.
\subsection{Manpower Policies}
One interventionist supply side policy is manpower policies, which includes investment in human capital through education and training. Education and training helps to improve the skills and thus quality of the workforce, as well as help in re-structuring of the economy by targeting the training towards meeting the needs of the industry, thus reducing structural unemployment. In addition, a more educated workforce is usually more attractive to firms since people are more productive, thus more investment is attracted. Singapore has various skills training policies in place such as the Continuing Education and Training (CET) scheme to help workers remain relevant. There is also the Skills Programme for Upgrading and Resilience (SPUR) which provides vocational training for retrenched workers, especially during the recent recession. Finally, there is the Skills Development Fund (SDF) which subsidizes skills training for workers.
\subsection{Nationalization}
Nationalisation is a second interventionist policy. Nationalisation is the opposite of privatization and involves the government buying up private assets, be it in part or in full. This is usually implemented in strategic industries like transport and telecommunications. Nationalisation will provide the necessary government funds for investment which may not be available in the hands of the private sector. However, nationalization is not usually done in Singapore. In fact, the government is moving towards privatization gradually so that Singaporeans can have a greater stake in the economy.
\subsection{Grants for R\&D}
The government may intervene through grants or sponsorship of R\&D in industries which they consider to be of importance to the economy. Research and development is inevitable in order to find out novel production techniques ad create new products. Innovation and technology advancement can help to increase the productive capacity of the economy and thus increase aggregate supply of the economy. Singapore aims to raise R\&D to 3.5\% of its annual GDP by 2015 and invests heavily on R\&D in sectors like biomedical sciences through the creation of the Biopolis as well as in aerospace technology.
\section{Evaluation of Supply Side Policies}
\subsection{Effects on the Economy}
Supply side policies affect all 4 macroeconomic aims of the economy. By increasing the aggregate supply of the economy while reducing cost push inflation such as wage push inflation, sustainable economic growth can be achieved where there are real increases in national output with minimal inflationary pressures.

Skills training will help to reduce the level of structural unemployment in the economy while job fairs will reduce frictional unemployment by providing more information to potential workers on the jobs available.

Finally, by keeping inflationary pressures low, supply side policies can help the general price level of the economy to rise at a slower rate than other countries. In the long run, a lower inflation rate will cause the price of exports of the economy to be lower relative to other countries, thus increasing the export competitiveness of a country. This will help to maintain or improve the balance of payment through the current account. This is especially important to export-reliant countries like Singapore.
\subsection{Limitations}
Unlike the demand management policies that the government may implement, supply side policies have the advantage of not having significant conflicts between the different macroaims. However, a main limitation of supply-side policy is that it is long term in nature and takes a long time before the effect is felt. This is because education takes many years and creating a harmonious tripartite relationship build on trust such as in Singapore takes a long time too. Thus, it is not suitable in crisis like a recession.

In addition, the effectiveness of supply side policies may depend on the receptiveness. For example, even with the many schemes available for skills training, if workers do not take advantage of such schemes, there will be minimal effect on the quality of the workforce.
\end{document}
