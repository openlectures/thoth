\RequirePackage{../../dominatrix}

\title{Standard of Living}
\author{\large by Jeremy Tay}
\date{\small correct as of \today}
\begin{document}
\maketitle
\tableofcontents
\section{Material and Non-Material Wellbeing}
\subsection{Standard Of Living}
The idea of a standard of living is a normative concept. It may seem like a very smoky concept; but the topic of SOL is very easy to understand once you understand the big picture of SOL.

In this topic, there are almost no graphs to draw. Some of you may feel uncomfortable that there are no graphs to draw; but the truth is, this topic is really quite qualitative rather than quantitative.

There are 2 measures of SOL: material and non material welfare.
\subsection{Welfare Of The Materialistic Person - Material SOL}
In order to measure Economic welfare you must remember these four words: REAL GDP PER CAPITA. Every single word is important. Trust me. And this is all you need to know about how to measure material welfare

So if the real GDP per capita of a country rises, then we can say that the Material welfare of a country has increased. However, how do we explain this in the examination. This is very easy. In the examination, after establishing that the real GDP per capita of a country has increased, we then say that citizens of the country are able to purchase more goods and services and hence their material standard of living has increased. Simple? 
 
That is really all you need to say; nothing less, nothing more in the examination.
\subsection{The Long Term View On High Real GDP}
However, real GDP per capita does not give any indication of productivity in a country. Suppose the GDP per capita of country rises, but productivity remains stagnant, ceteris paribus, this indicates that people in the country have to work longer hours; this is against the idea of work-life balance.

Furthermore, an increase in the real GDP per capita can also mean that pollution levels are rising in a country, this will have detrimental effects on future generations of those in the country.

At the same time, a rising real GDP per capita may not indicate that the average citizen is better off especially if income inequality is rising. The rise in the GDP per capita may be simply because of an increase in incomes at those at the highest income brackets. In fact, in Singapore, even though the real GDP per capita has been rising over the past decade; the real GDP per capita of those at the bottom decile of the income range has been nearly stagnant. 
\subsection{Looking At The Finer Things - Non-Material SOL}
When we talk about the non-material standard of living, we usually bring that in as an antithesis to why an increase in the material standard of living may not necessarily lead to an increase in the quality of life of citizens in a country.

In order to measure the non-material standard of living, we usually use indices like the HDI or Gross Happiness Index (in Bhutan), which is an aggregation of various non-economic metrics like literacy rates and life expectancies.
\section{Question Analysis}
\subsection{Standard Of Living Question}
Now that we have covered the boring stuff, let’s go straight to the point of SOL. In J2, 99.9\% of questions that deal with SOL are mainly concerned about whether the SOL of a country has changed as a result of some macroeconomic change.

For example, let me give you a very common question in SOL: To what extent has the SOL of a country improved as a result of a 5\% GDP growth rate? 

This is usually a 15 mark question. And if it comes out you should do it because it is like a christmas present even though its not christmas.

So how do we approach this question?

As usual; as with all H2 Economics questions: the approach is the thesis-antithesis-synthesis approach. With these kinds of questions we always start with whether the material welfare of citizens in a country has improved. So recall that material welfare is measured by RGDP per capita.
\subsection{This House Believes That\ldots}
Since the preamble has given you the fact that GDP has increased by 5\%, we can conclude ceteris paribus, that real GDP per capita has improved, and that citizens in the country are able to afford more goods and services and are hence able to enjoy a higher material standard of living.

However this ceteris paribus assumption is dangerous.

Remember that nominal GDP is only a part of real GDP per capita. We have assumed that everything else has been held constant. For these kinds of questions, we usually state that it is unlikely that the population has grown more than the nominal GDP has GDP per capita usually increases ( unless there is a massive war or genocide than I’m sorry).

But then we have not dealt with the “REAL” part of the real GDP per capita.

So how do we address this “real” part. This part usually requires a bit of general knowledge.

Suppose food prices are rising rapidly (especially in China) and demand for commodities like oil is rising, then we may argue that inflation may out pace the increase in the nominal GDP: in this case we have our antithesis, that although nominal GDP per capita has risen, real GDP per capita may have fallen.

Of course we may argue the converse, how you choose to argue will depend on specific macroeconomic conditions. So what is the lesson learnt? Read more Economist or wall street journal.
\subsection{On The Other Hand\ldots}
The anti thesis for such questions is essentially a repetition of the shortcomings of how we measure SOL through the material wellbeing of citizens.

Remember - Real GDP per capita does not give any indication of productivity in a country. Suppose the GDP per capita of country rises, but productivity remains stagnant, ceteris paribus, this indicates that people in the country have to work longer hours; this is against the idea of work-life balance.

Furthermore, an increase in the real GDP per capita can also mean that pollution levels are rising in a country, this will have detrimental effects on future generations of those in the country.

At the same time, a rising real GDP per capita may not indicate that the average citizen is better off especially if income inequality is rising. The rise in the GDP per capita may be simply because of an increase in incomes at those at the highest income brackets. In fact, in Singapore, even though the real GDP per capita has been rising over the past decade; the real GDP per capita of those at the bottom decile of the income range has been nearly stagnant.
\subsection{Exception}
Finally, a caveat. Remember the although I said that 99.9\% of SOL questions are variations of the questions I gave you that does not give you the license to just copy wholesale what I gave you. The preamble can always be different; you must be versatile enough to adapt your answer to the question. Furthermore there is always the 0.1\% chance that the question may not be the same (especially if you come from a certain school in the Bishan area); thats why you should make sure you know the concepts well; rather than just regurgitate.
\end{document}
