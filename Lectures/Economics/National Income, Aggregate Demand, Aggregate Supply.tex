\RequirePackage{../../dominatrix}

\title{National Income, Aggregate Demand, Aggregate Supply}
\author{\large by Cynthia Gao}
\date{\small correct as of \today}
\begin{document}
\maketitle
\tableofcontents
\section{Introduction to the Circular Flow of Income}
\subsection{Flow of Income in a Simple Two-Sector Economy}
Hello everyone. Before we learn anything about the macroeconomics, we must first understand how the economy works. If money is the blood of the economy, then the circular flow of income is its circulatory system. So how exactly does money, and the goods and services it buys, move in an economy?

Let’s start with a simplified economy. On one side you have households, which are made up of people like you or I that spend the money they earn on things they want. On the other side you have domestic firms, just like the shop around the corner, these businesses employ people to sell you goods and services. 

The two arrows that link these entities are national income (Y) and consumption (C). When households consume goods and services from firms, they pay them money. This money is in turn used to pay the salaries of the firm’s employees and other factors of production. This is how money flows from households to firms.

There are two important principles that underlie the circular flow. The first is that in every economic exchange, the seller receives exactly the same amount that the buyer spends. When I buy bubble tea, all the money I use to purchase the drink goes to the shop. Money does not inexplicably disappear as it passes from my hands to the cashier. 

The second is that goods and services flow in an opposite direction relative to money payments. When households consume goods and services, the payment flows from the household to the firm while the purchased product moves from the firm to the household.
\subsection{Components of a Four-Sector Economy}
Now obviously things are more complicated in the real world. Not all money you earn goes into buying items domestically. Your government will take a portion for taxes, you will save some, and some of it will go into buying that fancy imported luxury good you’ve craved for so long. 

Besides households and firms, the 3 other sectors in the economy are Financial Markets, the Government and the Foreign Sector.

Not all the money you earn goes back to domestic firms through the purchase of goods and services. When part of your salary goes into these 3 other sectors, they are called withdrawals or leakages. So how exactly does money move into these sectors?

Financial markets are essentially banks, and money flows into them in the form of savings (S). Governments take a portion of your money in the form of taxes (T), and your money flows into the foreign sector when you import (M) goods from overseas. Because of these leakages, your total spending, or consumption, might not equal the national income.

Calling these flows leakages can be misleading, because even though they "leak" out of the simplified economy they are not completely lost. Simply put, your tax dollars don’t fall into some black hole never to be put to productive use. Rather, they are injected back into the economy and go to the firms. Injections, thus, refer to any payment of income given to domestic firms that do not arise from households.

When you put money into a savings account, the banks will use it to invest in promising new firms or products. This investment (I) is how money flows from financial markets to firms. Government expenditure (G), say on infrastructure projects, is how money flows from the government to firms. Lastly, foreign nations pump money back into our economy when they purchase our exports (X) and pay our domestic firms. 

Because there are no black holes in the economy, the sum of leakages or withdrawals must be equal to that of injections when an economy is at equilibrium. Do note that this does not mean that savings must equal to investment or taxes to government expenditure (some of the taxes our government collects goes into our reserves). 
\subsection{Determining National Income}
In this lesson we will learn how national income (Y) is calculated. We’ve mentioned before that a key principle of the circular flow is that what the seller receives must equal what the buyers spend. This essentially means that national income must equal national expenditure. 

The production of goods and services generates income for households in the form of wages, profits, rents and interests. Adding up all these incomes gives us the national income (Y) of the economy.

Alternatively, since national income and expenditure are equivalent, another way to determine Y is to add up all expenditure on final output, which are basically payments made to firms. This includes:
\begin{itemize}
\item Consumer expenditure by households
\item Investments 
\item Government expenditure on final goods and services as well as non-marketed services like education
\item Net exports (X-M)
\end{itemize}
Hence $Y = C + G + I + (X-M)$ 
\section{Aggregate Demand}
\subsection{Explaining the Process of Aggregation}
The difference between demand and supply, and aggregate demand and supply, clearly, is the word aggregate. But what exactly does this mean?

The difference between micro and macroeconomics is the level at which the economy is studied. In macroeconomics, we are concerned with the big picture, the economy as a whole. So we don’t care about fine distinctions between firms and markets. We don’t want individual demand curves; we want the total demand in the economy. And to do that we simply add up the demand curves for every market in the economy. This process of summing individual economic variables to obtain economy-wide totals is called aggregation. 

By aggregating individual demand and supply curves, we get the curves for aggregate demand and supply, which look like this. 
\subsection{Introduction to the AD Curve}
In this lesson we will learn about the AD curve. Aggregate demand refers to the total level of spending in an economy at each price level. It shows the amount of domestically produced goods and services that households, firms, the government and foreigners desire to buy at each level. 

Thus $AD = C + G + I + (X-M)$

Before we draw the AD curve, we first have to label the axes. On the Y-axis we have the general price level. Recall that in microeconomics, we would have put price here. However, since we are looking at the economy in general, we are not concerned about the price of individual goods, but rather the average or general price of all goods and services. 

On the X-axis we have real national output or income, since the two are equivalent, as we have previously explained. We add the word "real" to show that it has already been adjusted for inflation. This is an important step, and we will talk more about it later. 

The AD curve itself is simply a downward sloping line, and looks identical to the demand curve in microeconomics. The only way to distinguish the two is by the axes, so we cannot forget to label them in exams!

Ceteris paribus, or keeping all other factors constant, the higher the general price level, the lower the quantity of goods and services demanded. There are several reasons for this. The most obvious one is that when the sticker price of all goods and services in the economy increases, your ability to buy things, or your purchasing power, falls. Your \$100 now buys less, which discourages you from buying as much as you would have otherwise wanted. Since everyone else in the country feels the same, consumption (C) in the whole economy falls. Since C is a component of AD, national expenditure and thus output falls. This causes the AD curve to slope downwards.
\subsection{Factors that Cause a Shift in the AD Curve}
In the previous lesson, we’ve looked at how changes in the general price level affect the demand for goods and services in the economy. Notice, however, that these changes essentially involve a movement along the AD curve. What factors, then, cause the entire AD curve to shift?

When the AD curve shifts, the quantity of goods and services demanded at every price level changes. The factors that cause these shifts are thus called non-price factors, because their effects are not dependent on the general price level. These factors can be grouped into three main sets:
\begin{itemize}
\item Changes in Expectations
\item Changes in Government policies
\item Changes in the World Economy
\end{itemize}
So firstly, let’s talk about changes in expectations. Let’s say I’ve somehow managed to peer into the future with the help of a passing fortuneteller. Much to my surprise, I see that I managed to clinch that promotion I wanted, and the huge pay rise that comes with it. Even more delightfully, I see that the shares I have just bought have doubled in value, and the mall just built beside my house has caused it to increase dramatically in price. Believing my wealth and income will increase in the future, I naturally feel much more financially secure in the present, and thus opt to buy that car I was looking at. Assuming this passing fortuneteller managed to meet most of the population as well, which isn’t hard to believe since we live in a world where people can see into the future, consumption in the economy increases dramatically in the present. 

Now, being able to predict the future isn’t a prerequisite for this to happen. People just need to believe they can. In other words, as long as there is optimism in the economy, households are more likely to increase present consumption, which leads to a rightward shift in the AD curve as shown. This can also happen if people expect inflation to increase in the future, which causes goods and services to be more expensive. Hence households prefer to purchase more goods now before the price increases.

For more information on the other two sets of factors, you can refer to subsequent lectures.
\section{Aggregate Supply}
\subsection{Introduction to the AS Curve}
Where there is an aggregate demand curve, there must be an aggregate supply curve. Similar to it’s microeconomic counterpart, the AS curve shows the total output of goods and services that firms as a whole would like to produce and sell at each price level. 

There is quite a lot of debate among economists regarding what the AS curve should look like. When we put all these models together we get the AS curve as shown. As you can see, there are 3 distinct regions:
\begin{itemize}
\item On the extreme left we have the horizontal Keynesian range. The national output is very low over this range, meaning resources are largely unutilized or under-utilized. This is commonly seen during recessions when aggregate demand is very low, and there is significant unemployment in the economy. Should there be a rise in AD, national output easily increases without any increase in the general price level. In other words, AS is perfectly elastic over this range. 
\item On the extreme right we have the vertical classical range. At this point, the economy has reached full employment (Yf), meaning all available resources have been efficiently utilized. Output cannot increase any further because the economy has run out of raw materials. Instead, when AD increases the general price level rises. Supply is perfectly inelastic over this range.
\item The curve connecting these two regions is the intermediate range. Since the curve is upward sloping, increases in AD result in both increases in national output and general price level. 
\end{itemize}
\subsection{Movements in the AS Curve}
Since the AS curve has two extreme regions, it can shift in two very different ways. This lesson will explain what each of these movements mean, and when to shift the curve in that manner. 

So what are these two main movements? Lets start with the horizontal Keynesian range:
This horizontal range can either shift upwards or downwards as shown. It is mainly affected by changes in costs of production. If raw materials like oil become more expensive, unions negotiate for higher wages or governments reduce production subsidies, the AS will shift upwards as shown. The converse is also true. 

Essentially, in order to determine whether the AS curve will shift upwards, you just need to figure out whether the firm’s cost of production increases, and vice versa. 

Now on to the vertical classical range: This part of the curve can move either left or right. It is affected mainly by changes in the productive capacity of an economy, and thus moves in a similar manner to the production possibilities curve.

Since the classical range denotes to full employment level of output in the economy, it is affected by factors that impact the maximum amount of goods and services a country can produce. 

If a country has a sudden baby boom, or improves its level of technology, it will increase its production potential, and hence the AS will shift to the right. If a massive natural disaster destroys a nation’s factories, the greatest quantity of goods and services it can produce will fall, resulting in a leftward shift in the AS.
\subsection{Simultaneous Shifts}
Now that we’ve covered the rationale behind both movements, let’s go back to an example we’ve raised earlier. We’ve already explained that improvements in technology, like the advent of the internet or the computer chip, can increase the productive capacity of a country. But now that we think about it, the increases in worker productivity that automated processes generate also mean the cost of producing a single unit of good increases. Instead of one factory worker being able to process 10 boxes an hour, with the aid of machines he can now process 100. The cost of churning out each box falls. So how do we shift the AS?

In this case, the AS moves both rightwards and downwards. The entire curve shifts outwards. The point here is that whilst the reasons behind to two AS movements may be different, they are not mutually exclusive. The same factor can cause both movements. Hence, rather than trying to categorise every factor under the sun, the easier thing to do is to ask yourself how the factor affects cost of production and productive capacity, and shift the curve correspondingly.
\section{Equilibrium Level of Output and Price}
\subsection{Finding the Equilibrium Level of Ouptut and Price}
Putting the AD and AS curves together allows us to find the equilibrium level of output and price. I think most of you will understand intuitively that this is the point where the two curves intersect, much like the demand and supply model in microeconomics. The purpose of this lesson is to explain to you why this is so. 

What happens when the general price level is below the equilibrium price? Let’s say the price level is at 0P1. This line intersects the AS curve at point A and the AD curve and point B. We can clearly see that the output at B is greater than that at A. This means that the quantity demanded for goods and services exceeds the quantity supplied, creating a shortage. Some consumers will be willing to pay higher prices for the limited goods, just like how limited edition merchandise fetches higher prices. As prices rise, firms, seeing there is money to be made, increase production resulting in an increase in national output. The economy thus moves towards the equilibrium. 

Now let’s look at the opposite situation. Say the general price level is at 0P2, above the equilibrium level. Now A is greater than B, meaning the quantity of goods and services supplied exceeds the quantity demanded. This creates a surplus. Firms reduce prices to clear excess stock, causing consumers to purchase more goods and services as the prices fall. Just think clearance sale. The economy once again moves towards the equilibrium point. 
\subsection{Introduction to the Keynsian Model of Income Determination}
So far we’ve talked extensively about the AD/AS model of income determination. However, there is another model in the syllabus we’ve yet to discuss that helps us explain certain concepts that are more difficult to represent on the AD/AS model. We call this model the Keynesian or Aggregate Expenditure (AE) model on income determination. 

Let’s begin by going through the assumptions of this model. They are as follows:
\begin{itemize}
\item A constant level of technology
\item A constant productive capacity or potential output level (recall classical range of AS curve)
\item A fixed general price level i.e.\~no inflation
\end{itemize}
Recall that the x-axis of the AD/AS model is real national output. However, since one of the assumptions of the AE model is that there are no inflationary pressures, the x-axis of the AE model is nominal national output. Nominal means inflation is not taken into account. Additionally, since we’ve assumed the price level is fixed, the y-axis of the AE model must also change, hence we simply label it AE instead. 
\subsection{Planned vs. Actual Expenditure}
The key to understanding the AE model is the concept of planned versus actual expenditure. Planned expenditures describe how firms, consumers and governments choose to spend given what they expect to happen in the future. I think I will be hungry in an hour, so I plan to purchase a sandwich. 

Planned expenditures need not equal actual expenditure. Just because I plan to buy a sandwich doesn’t prevent me from choosing not to an hour later if I inexplicably no longer feel hungry. If buyers spend less on goods and services than firms anticipate, then actual expenditures are less than planned expenditures and firms will experience unplanned changes in their inventories. When both planned and actual expenditure are equal, the economy is said to be in equilibrium. 

Aggregate expenditure is defined as the planned expenditure of the various sectors of the economy, and is comprised of the same components as aggregate demand
$$AE = C + G + I + (X-M)$$
However, even though they are made up of the same components, by virtue of the different axes, the AE curve looks very different to the AD curve. As you can see, the AE curve is an upward slopping line that need not cut the graph at 0.
\subsection{The AE-Income Approach of Income Determination}
So now that we know how the AE curve looks like, what’s next is learning how the model works. 

We’ve explained in the previous checkpoint that the equilibrium level of national income occurs when planned expenditures equal actual expenditures. Firms correctly anticipate how much goods and services are demanded, and produce only up to this point, so there are unplanned increases or decreases in their spare stocks and inventories. Firms don’t need to decrease production in the next period, for example, to get rid of excess accumulated stock. 

But our graph only has one line on it at the moment; there are no intersection points, so how can we find this equilibrium level? We need to add one more line to the graph.

This line represents actual expenditure: the real amount of money spent. Recall that in Section 1.3 we’ve established that national income equals national expenditure. Hence the line that represents actual expenditure is an upward sloping line that cuts the graph at 0 and forms a 45 degree angle with the x or y axis. At any point on this line, the corresponding x and y values, in other words the expenditure and output, are exactly equal. We label this line $Y = AE$. The point at which this line intersects the line representing $AE = C + G + I + (X-M)$ is the equilibrium level of national output. 
\section{The Multiplier Effect}
\subsection{What is the Multiplier Effect?}
Over the course of your study of economics, you will learn to recite the multiplier effect in your sleep. It is one of the most important concepts in the syllabus, so let’s get started immediately. 

The multiplier effect occurs when one person’s spending becomes another person’s income. Let’s say I pay a grocer \$2 for some apples. The grocer, now \$2 richer, uses that money to buy some bubble tea. Now the bubble tea shop owner, feeling richer as well, goes on to purchase some stationery. At each transaction, one person’s spending becomes another’s income.

The multiplier effect is defined as the numerical coefficient by which an autonomous change in AE, in other words changes in C, G, I or (X-M), is multiplied by to derive its final impact on national output. This seems like quite a mouthful, so for now we’ll just look at the multiplier effect in action. 

Let’s say the government spends \$1 million to build a factory. The money does not disappear, but rather becomes wages to builders, revenue to suppliers etc. The builders will have higher disposable income, and may spend that on goods and services, so that aggregate demand will also rise. Suppose further those recipients of the new spending by the builders in turn spend their new income, this will raise demand and possibly consumption further, and so the process continues. However, not all of the increases in income will be spent. Some money ‘leaks’ away from the cycle in the form of savings, taxes or import expenditure. Every time the process repeats itself, some money leaks away. Much like friction on a moving object, this causes the multiplier process to slow down, and the cycle eventually stops when the total leakages equal the initial \$1 million injection.

The increase in the national income or output is the sum of the increases in net income of everyone affected. Since one person’s spending becomes another person’s income, the increase in national income is much greater than the initial \$1 million injection, and in fact is a multiple of it. 
\subsection{Calculating the Multiplier Effect}
In economics, the multiplier is denoted by k, where 
$$k=\frac{\delta Y}{\delta AE}=\frac{1}{\textrm{Marginal Propensity to Withdraw (MPW)}}$$
Recall that in checkpoint 1.2, we learnt that the main withdrawals or leakages in the economy are savings, taxes and imports. $\textrm{MPW} = \textrm{MPS} + \textrm{MPT} + \textrm{MPM}$. MPW is thus the increase in total withdrawals for every unit change in income. Since every \$1 increase in income must go to either consumption, savings, taxes or imports, as per the circular flow of income,
$$\textrm{MPC} + \textrm{MPS} + \textrm{MPT} + \textrm{MPM} = 1$$
When $MPW=0.4$, $k=\frac{1}{\textrm{MPW}}=\frac{1}{0.4}=2.5$. Hence an autonomous injection of \$1 million would increase national income by \$2.5 million. 
\subsection{Explaining the Multiplier Effect Using the AE Model}
In this lesson, we will explain the multiplier effect using a graphical illustration. This explanation is what you should be using in exams, and can be shortened or lengthened according to the mark weightage. Essentially, you need to show at least 2 rounds of the multiplier, and explain when it stops. 

We start with an autonomous increase in AE, as shown by the parallel upward shift of the planned AE curve from AE1 to AE2. 

At the original national output level $Y_0$, planned AE ($BY_0$) exceeds actual national output (point $AY_0$). Firms have failed to correctly anticipate demand, and thus have to draw upon their stocks and spare capacities to cater to excess demand. We call this an unplanned disinvestment, and it is equal to the value of the excess expenditure (AB).

Firms thus have an incentive to increase production in the next time period in order to restore they’re stocks to the desired levels. They thus increase production by BC, which equals to the unplanned disinvestment AB. In order to do this, they must engage more factors of production, which include labour, and pay more factor income. Hence national income increases by BC to the new output level $Y_1$.

When workers get paid more income, they are likely to spend part of that increase on goods and services. This causes an increase in induced consumption of CD. Note however, that not all the income is spent on consumption. Some of the money leaks away in the form of savings/taxes/imports. Hence we see that at output level $Y_1$, planned AE ($DY_1$), once again exceeds actual national output ($CY_1$), so firms draw upon their spare stocks and an unplanned disinvestment of CD occurs. They again have an incentive to increase production, and the process continues. 

At each new cycle, the additional induced consumption falls due to the leakages mentioned. When total leakages equals the initial autonomous increase, the multiplier effect stops, resulting in a new equilibrium level of national income, $Y_2$.
\subsection{The Multiplier in the Real World}
The multiplier effect is the cornerstone of Keynesian economics since it shows just how great the impact of that \$1 billion government-spending project is on kick-starting the economy. Like a boulder at the top of a hill, that initial push can lead to a monumental movement. 

In the real world however, things aren’t so peachy. One big limitation of the multiplier is that part of the increase in AE may manifest itself in higher prices instead of higher output. In the AE model, we’ve assumed a constant general price level. This means the economy is operating on the Keynesian range. This isn’t true for most economies in reality. When the economy is operating close to full employment level, increasing demand is just going to cause inflation. 

Additionally, it takes a lot of time for the multiplier process to work. Each round of spending may take months, so only the effects of the first few rounds of the multiplier may be observable in the short term.

Finally, every country has a different multiplier value due depending on its MPW. Forced savings schemes like CPF in Singapore contribute to our high MPS. This is exacerbated by the Asian mentality to save for a rainy day. Moreover, our lack of natural resources mean we are very dependent on imports, increasing our MPM. As a result, Singapore’s multiplier is very small, making it much more difficult to increase national income through demand management tools.
\end{document}
