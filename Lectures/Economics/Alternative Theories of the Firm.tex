\RequirePackage[l2tabu,orthodox]{nag} % Checks for incorrect or obsolete LaTeX packages

\documentclass[DIV=calc,11pt,parskip,numbers=noenddot]{scrartcl} % Uses the KOMA-Script package with customizations

% Universal Fixes
\usepackage{fixltx2e} % Corrects LaTeX2e bugs and quirks
\usepackage{ifxetex} % Checks if XeTeX/XeLaTeX is the compiler-of-choice

% Math
\usepackage{amsmath} % Swiss-knife math package

% Graphics
\usepackage{tikz} % Engine that produces vector graphics from geometric/algebraic descriptions
\usepackage{graphicx} % Allows inserting of graphics and images
\usepackage{epstopdf} % Converts .eps files to .pdf for easy manipulation

% Layout and Format
\usepackage{parallel} % Aligns paragraphs across columns
\usepackage{lineno} % Appends line-numbers in the margin with \linenumbers
\usepackage{showkeys} % Shows label references where they are defined
\usepackage{csquotes} % Fixes inline and display quotations
\usepackage{paralist} % Allows inline enumeration in a paragraph and compact enums/lists

% Tables and Figures
\usepackage{booktabs} % For drawing nice tables with proper line weights
\usepackage{flafter} % To ensure that figures float only after they are defined/referenced
\usepackage{subfig} % Figure-ception - Allows figures within figures
\usepackage{array}  % Extends options for column formats, alignments and layouts
\usepackage{tabu} % Provides better control for tables and column widths
\usepackage{longtable} % Allows tables to span across pages - integrates with tabu
\usepackage{multicol} % Allows spanning columns in tables
\usepackage{caption} % Allows greater customization of captions and captions outside floats

% Referencing
\usepackage[longnamesfirst]{natbib} % Reimplements \cite to work with author-year and numerical citations
\usepackage{cleveref} % Adds semantic naming when referencing figures
\usepackage{varioref} % Introduces referencing by page instead of figure number

% Fonts and Typography
\usepackage{microtype} % Tweaks smallish fonts and kernings
\usepackage{textcomp} % Supports the Text Companion font, which provides additional text symbols
\usepackage{siunitx} % Provides support for typesetting SI units
\usepackage{ellipsis} % Fixes space uneveness around ellipses
\usepackage{url} % Allows encapsulated URLs to break across lines
\usepackage[colorlinks,hypertexnames=false,plainpages=false]{hyperref} % Converts \url references to valid hyperlinks in PDF documents

\usepackage{euler}

\ifxetex
\usepackage{fontspec} % Allows usage of system opentype/truetype fonts
\setsansfont[BoldFont={* Bold}]{Miso} % Sets the Header font via system font name
\fi
\usepackage[T1]{fontenc} % Sets output font encoding to support accented characters and Type 1 fonts
\usepackage{concrete} % Swaps out ancient Computer Modern font for Latin Modern

\setkomafont{title}{\fontencoding{EU1}\sffamily}
\setkomafont{section}{\fontencoding{EU1}\sffamily\Large\centering}
\setkomafont{subsection}{\fontencoding{EU1}\sffamily\large}

% Title Information
\title{Alternative Theories of the Firm}
\author{\large by Linan Qiu}
\date{\small correct as of \today}
\begin{document}
\maketitle
\tableofcontents
\section{Problems with Traditional Theory}
\subsection{Maximizing Profits}
So in traditional economics, we always assume that firms maximize profit. That’s not too bad an assumption at all, because who doesn’t want profits? But do all people want to maximize profits? Let me tell a story that one of my economics teachers loves to tell. When I take taxis, I love talking to the taxi drivers. They usually have interesting stories to tell. Same for my economics teacher. So he asked the taxi drivers if they earned more on a rainy day. Now think. Since more people require taxis on a rainy day, taxi drivers should be earning a lot more right? That’s actually not the case! In fact, most taxi drivers have a mental quota on how much they should earn a day. Once they fulfil that quota, they go home! So on a rainy day, they just fulfil that quota early, and go home early! Is that irrational? According to traditional economists, yes! But if so many people are irrational, and I’m pretty sure you can find other people who do not purely work to maximize profits, then there’s something wrong with traditional economics isn’t it? Well, there is. There are two reasons why people don’t maximize profits all the time. Let’s look at the first. First, it is difficult to maximize profits. Most firms do not have their own MC, MR curves. They don’t find the quantity at which they meet, then use that quantity to produce. In fact, they do it through trial and error, and most of the time they don’t produce at a point where profit is maximum. Also, find me a firm that uses economic costs in accounting procedures. Their chief financial officer will tell you that you are crazy. There are theories like game theory in economics which can help predict behavior, but even that is limited because it assumes firms know the consequences of their own actions and that of their opponents. If Microsoft can so accurately predict the actions of Apple, would it still end up like it is today? The second reason that profits are not maximized is because some firms simply have alternative aims. This is the one we are going to explore in more detail.
\subsection{Alternative Aims of the Firm}
 In traditional economic analysis, we always assume that it is the owner of a firm that decides the output and the operations of a firm. Notice how I’ve always used the example “imagine you are the owner of Foxconn”. Well the assumption here is that the owner of Foxconn makes the decisions! But do they get to make all the decisions? However, many large companies nowadays are public limited companies. The shareholders are the owners, and profits so as to increase their dividends and the value of their shares. Shareholders elect directors. Directors in turn employ professional managers who are often given considerable discretion in making decisions. There is therefore a separation between the ownership and control of a firm. This causes many interesting things to happen, some of which we will explore in the next lesson. But first, let’s look at how businesses are organized to understand this separation between owners and managers. Small firms are often governed by their owners. Think about your food stall uncle. He’s the boss, the cook, the finance guy. He’s everything. In larger firms, however, things become more complicated. Medium size firms are usually broken into separate departments. Your school probably operates via this model! There are departments like marketing, finance, production. For the school, there’s the teachers, then there are the finance people, the admin people, admissions people and so on. The managers of each department are normally directly responsible to a chief executive, or your principal - though for some schools in Bishan there’s not much of a difference. This chief executive coordinates all their activities, relaying the firm’s overall strategy and being responsible for inter departmental communication. We call this the U-form. U for unitary. When a firm expands beyond a certain size however, a U form structure is likely to become way too inefficient. this inefficiency arises from difficulties in communication, coordination and control. It becomes too difficult to manage the whole organisation structure from the centre. Then comes the M form of organization. To overcome these organizational problems, the firm can adopt a M (multi divisional) form of management. This suits larger firms. The firm is divided into a number of divisions. Think Apple. It has Apple South East Asia, Apple North America and so on. Each division could be responsible for a particular product or group of products, or a particular market. the day to day running and even certain long term decisions of each division would be the responsibility of the divisional managers. So now you can see that in most firms, the owner of the firm isn’t actually the guy running the things. But let’s complicate things a little more. As many business expand their operations further, often on a global scale, much more complex forms of organizations appear. There’s the H form, or that of a holding company. A holding company (or a parent company) is one that owns a controlling interest in other subsidiary companies. these subsidiaries, in turn ma also have controlling interests in other companies. This is a complex web of ownerships. While the parent company has ultimate control over its various subsidiaries, it is likely that both tactical and strategic decision making is left to the individual companies within the organization. Many multinationals are organized along the lines of an international holding company, where overseas subsidiaries pursue their own independent strategy. For example, Sinopec is the largest oil and gas producer in China. It does not have a Singapore division. Rather, it has 3 to 4 Singapore subsidiary companies!
\subsection{Principal Agent Problem}
 So can the owners always make sure that the managers listen to them and act in the best interests of the company? This is an example of what is known in economics as the principal–agent problem. One of the features of a complex modern economy is that people (principals) have to employ others (agents) to carry out their wishes. If you want to go on holiday, it is easier to go to a travel agent to sort out the arrangements than to do it all yourself. Likewise, if you want to buy a house, it is more convenient to go to an estate agent. The point is that these agents have specialist knowledge and can save you, the principal, a great deal of time and effort. It is merely an example of the benefits of the specialisation and division of labour. It is the same with firms. They employ people with specialist knowledge and skills to carry out specific tasks. Companies frequently employ consultants to give them advice or engage the services of specialist firms such as an advertising agency. It is the same with the employees of the company. They can be seen as ‘agents’ of their employer. In the case of workers, they can be seen as the agents of management. Junior managers are the agents of senior management. Senior managers are the agents of the directors, who are themselves agents of the shareholders. Thus in large firms there is often a complex chain of principal– agent relationships. But these relationships have an inherent danger for the principal: there is asymmetric information between the two sides. The agent knows more about the situation than the principal – in fact this is part of the reason why the principal employs the agent in the first place. The danger is that the agent may well not act in the principal’s best interests, and may be able to get away with it because of the principal’s imperfect knowledge. The estate agent may try to convince the vendor that it is necessary to accept a lower price, while the real reason is to save the agent time, effort and expense. In firms too, agents frequently do not act in the best interests of their principals. For example, workers may be able to get away with not working very hard, preferring instead a quiet life. Similarly, given the divorce between the ownership and control of a company, managers (agents) may pursue goals different from those of shareholders (principals). They may just slack off, or in economics, cause what we call “x-inefficiency”.
\subsection{Attitude Towards Risk}
 In traditional economics, we don’t consider risk as well. I can hear you exclaiming “What? How can economics be a proper discipline of study if it is so full of loopholes?” Well be patient! We are talking about traditional economics here, theories that are devised more than 6 decades ago. But back to risks. Not all firms make profits the top priority, because they are more concerned about survival. they might be what we call risk averse. They might need a very very high amount of profit to justify taking the risk to do something. In that case, they might miss out on certain profit opportunities. For example, if there’s a project that the firm can take on. It will earn \$100 million dollars - that’s a huge amount of money. But the success rate of the project is only 10\%. So the expected payoff is 10\% times 100million, which is 10 million dollars. Now let’s say the cost is 5 million dollars. A rational firm should do it! But a firm might be risk averse and simply not do the project because of the huge amount of risk. So he’d need a much larger pay off to do the project. Not all firms, however, make survival the top priority. Some are adventurous and are prepared to take risks. Adventurous firms are most likely to be those dominated by a powerful and ambitious individual – an individual prepared to take gambles. The more dispersed the decision making power is in the firm, the more worried managers are about their own survival. and the more cautious their policies are. They will prefer to stick with products that are popular, and use tried and tested methods to earn money. Soon, the company will find that it is losing ground to more aggressive competitors.
\newpage
\section{Alternative Maximizing Theories}
\subsection{Long Run Profit Maximization}
 So in this section, we will explore some of the alternative aims that a firm can adopt. Traditional theory assumes that we always maximize profits in the short run. However, investing in a large piece of equipment can make a loss in the short run, but make profits in the long run. Same for investing in some other capital intensive products, or entering another market. Hence long run profit maximization should be a good alternative to the traditional short run profit maximization. However, this theory is not very useful. Why? Managers can just claim that they were attempting to maximize long run profits. They can use that as an excuse for virtually any policy! The manager can always justify anything by saying “ah yes, but in the long run it will pay off!” Well, in the long run we are all dead! This kind of excuses leads to jokes in economics like this. So there was this man who sits on a railway carriage throwing pieces of paper out of the window. Another man was curious and asked him why he kept doing this. The man said “It keeps the tigers away.” But the other man said “there are no tigers around here!” And the man throwing paper replied “I know! Isn’t that effective?” We can view this in another perspective. the boss of a company always does strange things. He would spend lots of money on advertising, then he will stop. Then he will give huge wage increases to keep workers happy. Then he will close the factory for two months to give everyone a break. He will move the business to a new location every now and then. His deputy asked him why he does these things. He says “I want to make the business profitable!” The deputy replied” But it’s already profitable!” So the boss said “I know! It just goes to show how effective my policies are.” The long run theory of profit maximization might be similar! It might be so impossible to refute, that it no longer becomes an useful theory. Some of you philosophers might relate this to the falsification criteria set out by Karl Popper. 
\subsection{Managerial Utility Maximization}
 So let’s say you’re a manager, not the boss, of Foxconn. Will you be very concerned about whether Foxconn makes money? Well maybe. But you’d definitely be more concerned about whether you make money, you get to keep your job, or you get your promotion! Unless those are in line with profit maximization, then you’re maximizing your own utility not that of the company! So an economist named Williamson in the 1960s said that once managers fulfilled a minimum criteria of profit earned, they often have the freedom to choose whether to continue pursuing profit at all costs or do something else in their own interests. So what are the managers’ interest? to maximize their own utility, said Williamson. Things like Salary, job security, dominance and professional excellence. Williamson then proceeded to try to align these with the overall aim of the company. Much of what we do in modern companies involves that as well! Giving managers stock options, bonuses and so on. 
\subsection{Sales Revenue Maximization}
 What if people are just trying to maximize revenue instead? That’s also possible, according to an economist named William Baumol in the 1950s. Some of the bonuses of managers are tied to their sales level, not profit level! Sales figures are also sometimes more noticeable than profit levels. If the manager manages to sell a lot of things, his prestige, and professional excellence might get a boost! Or perhaps you were a salesperson in an Apple store. You get commission from selling Apple products. Do you care about profits of Apple? Of course not. You just want to maximize Apple’s sales so that you earn the most amount of commissions! 
\subsection{Growth Maximization}
 Some other firms, rather than aiming for profits, aim for growth. They want the firm to grow bigger. Why? They want their company to be dynamic. When the firm grows as well, there will be more promotion prospects. Larger firms also pay higher salary. Growth is probably best measured in terms of a growth in sales revenue, since sales revenue (or ‘turnover’) is the simplest way of measuring the size of a business. An alternative would be to measure the capital value of a firm, but this will depend on the ups and downs of the stock market and is thus a rather unreliable method. If a firm is to maximise growth, it needs to be clear about the time period over which it is setting itself this objective. For example, maximum growth over the next two or three years might be obtained by running factories to absolute maximum capacity, cramming in as many machines and workers as possible, and backing this up with massive advertising campaigns and price cuts. Such policies, however, may not be sustainable in the longer run. The firm may simply not be able to finance them. A longer-term perspective (say, 5–10 years) may require the firm to ‘pace’ itself, and perhaps to direct resources away from current production and sales into the development of new products that have a potentially high and growing long-term demand. So what are some ways a firm can grow? We explore them here.
\subsection{Growth by Merger}
 Firms can grow by simply expanding their internal operations. They can build more factories, build more shops etc. But another interesting way that firms can grow is through merger. That is when they take over (in the case of a hostile takeover) or peacefully merge or simply buy over another company. Why do they do that? They can diversify their products. A good example of a highly diversified company is Virgin. Its interests include planes, trains, cars, finance, music, mobile phones, holidays, wine, cinemas, radio, cosmetics, publishing and even space travel. If the current market is saturated, stagnant or in decline, diversification might be the only avenue open to the business if it wishes to maintain a high growth performance. In other words, it is not only the level of profits that may be limited in the current market, but also the growth of sales. Diversification also has the advantage of spreading risks. So long as a business produces a single product in a single market, it is vulnerable to changes in that market’s conditions. If a farmer produces nothing but potatoes and the potato harvest fails, the farmer is ruined. If, however, the farmer produces a whole range of vegetable products, or even diversifies into livestock, then he or she is less subject to the forces of nature and the unpredictability of the market. They can also gain larger scale, and have more economies of scale. They can share factories, share shipping routes and so on. They can share monopoly power as well, and be more of a monopoly to well, rip consumers off. or sometimes, because of mismanagement, a company can become so cheap on the stock market that other people just decide to take it over for its equipment, patents and other assets. There are a few types of mergers that can occur. The first is a horizontal merger, where firms in the same industry and at the same stage of production merge. For example, two car manufacturers. then there is vertical merger, <10> where firms in the same industry but at different stages of the production of a good merge. For example, a car manufacturer with a car component parts producer. <11> Then there is a conglomerate merger where firms in different industries merge. <12> For example, when Siemens acquired some companies to go into the Water Technology sphere.
\subsection{Growth through Strategic Alliances}
 There are also strategic alliances that a formed between alliances. One very simple one is a joint venture when two or more firms pool resources to create a new organization or a new project. For example, in Malaysia, a few of the large oil and gas companies bunched together to build a storage facility rivalling that of Jurong Island in Singapore. Many companies can also form a consortium together. It is usually created for large projects, such as a large civil engineering work. Once the project is finished, the consortium dissolves. Franchising is another form of strategic alliances. A business agrees to franchise its operations to third parties. McDonalds and Coca Cola are good examples of businesses that uses franchises. A franchisee is responsible for manufacturing and selling, and the franchiser retains responsibility for branding and marketing. But why strategic alliances? Well, not everyone is an enemy! Not everyone who is an enemy is always an enemy. When there’s money to be made, and there’s more to be made together, people work together to earn them! It also allows people to share risks. For large construction projects, many companies work together so that risk can be spread and it won’t be too risky for anyone. They can also specialize and take less risky parts of the project for themselves. They can also pool capital! Projects that have way too high start up costs might encourage firms to cooperate and pool their capitals.
\newpage 
\section{Multiple Aims in a Firm}
\subsection{Key Performance Indicators}
 So now one concept you should be introduced to is that of KPIs - Key Performance Indicators. Large firms are often complex institutions with several departments. Each department is likely to have its own set of aims and objectives. these separate aims and objectives are set by the head of the company, and they are called Key Performance Indicators. Setting the KPIs is a tricky task, because you’d have to strike a balance between the person’s personal objectives, or the managerial utility, and the overall goals of the company, which is the owner’s utility. In many firms, numbers like production, sales, profit, stock prices and so on are used as measurements. Managers spend a ridiculous amount of time adjusting these. Often, the targets conflict as well.  Now think. If department’s target conflicts with that of another department’s, say saving electricity from the admin department, and working overtime and hence turning on the air conditioner at full blast for the operations department, whose target goes first? It simply depends on the political clout of the individual managers and their bargaining processes. 
\subsection{Organizational Slack}
 Most businesses actually set targets rather conservatively, because they need to account for changing market conditions. To avoid the need to change targets all the time, they set their targets not all the way up, but somewhere in the middle. This leads to what we call organisational slack. When the firm does better than planned, it will allow for slack to develop. For example, if your target was to earn \$20 million this year, and by three-quarters of the year you have already achieved 20 million, you don’t need to work that hard anymore! if you work too hard, the targets next year might be increased even further and you have to work your ass off. Instead, you just kind of slack off. thus keeping targets fairly low and allowing slack to develop allows all targets to be met with minimum conflict. However, organisational slack adds to a firm’s costs! If the firm was operating in a competitive environment, they may be forced to cut slack in order to survive. Hence organisational slack happens very often with monopolies. When the slack arises due to lack of competition, it is called "X-inefficiency".
\newpage
\section{In the Real World}
\subsection{Cost Based Pricing}
 So how does pricing actually work in real life? Well ask your food stall uncles. They’ll probably say I just take the cost and add something on top so that I’ll earn some money. This is what we call the mark up pricing. Is he earning profits? Of course! The maximum? That, you might not be very sure. Here look at this diagram. First we draw the AC curve. Then the AR curve. Where is the profit maximized? That’s where the distance between the two is maximum. Where the difference between the price and the cost is the most. However, the uncle might simply take the cost, and add some additional amount on top. In this case, he might produce Here or here. It might not be the maximum profit at all! But that’s how things work in real life - cost plus markups. 
\subsection{1995 BOE Survey}
 So in 1995, the Bank of England conducts a survey of price setting behavior of nearly 700 UK companies. Among other things, the survey sought to establish what factors influenced companies’ pricing decisions. The results are as follows The most common way of pricing was by market level. They were set at the highest level that the market could bear. Another 25\% said that they will look at something already in the market, and price it according to their competitors’ prices. The third most common way is to do the cost plus mark up. Now isn’t that interesting? When does it say profit maximizing? Now how’s that for traditional theory. 
\end{document}