\RequirePackage[l2tabu,orthodox]{nag} % Checks for incorrect or obsolete LaTeX packages

\documentclass[DIV=calc,11pt,parskip,numbers=noenddot]{scrartcl} % Uses the KOMA-Script package with customizations

% Universal Fixes
\usepackage{fixltx2e} % Corrects LaTeX2e bugs and quirks
\usepackage{ifxetex} % Checks if XeTeX/XeLaTeX is the compiler-of-choice

% Math
\usepackage{amsmath} % Swiss-knife math package

% Graphics
\usepackage{tikz} % Engine that produces vector graphics from geometric/algebraic descriptions
\usepackage{graphicx} % Allows inserting of graphics and images
\usepackage{epstopdf} % Converts .eps files to .pdf for easy manipulation

% Layout and Format
\usepackage{parallel} % Aligns paragraphs across columns
\usepackage{lineno} % Appends line-numbers in the margin with \linenumbers
\usepackage{showkeys} % Shows label references where they are defined
\usepackage{csquotes} % Fixes inline and display quotations
\usepackage{paralist} % Allows inline enumeration in a paragraph and compact enums/lists

% Tables and Figures
\usepackage{booktabs} % For drawing nice tables with proper line weights
\usepackage{flafter} % To ensure that figures float only after they are defined/referenced
\usepackage{subfig} % Figure-ception - Allows figures within figures
\usepackage{array}  % Extends options for column formats, alignments and layouts
\usepackage{tabu} % Provides better control for tables and column widths
\usepackage{longtable} % Allows tables to span across pages - integrates with tabu
\usepackage{multicol} % Allows spanning columns in tables
\usepackage{caption} % Allows greater customization of captions and captions outside floats

% Referencing
\usepackage[longnamesfirst]{natbib} % Reimplements \cite to work with author-year and numerical citations
\usepackage{cleveref} % Adds semantic naming when referencing figures
\usepackage{varioref} % Introduces referencing by page instead of figure number

% Fonts and Typography
\usepackage{microtype} % Tweaks smallish fonts and kernings
\usepackage{textcomp} % Supports the Text Companion font, which provides additional text symbols
\usepackage{siunitx} % Provides support for typesetting SI units
\usepackage{ellipsis} % Fixes space uneveness around ellipses
\usepackage{url} % Allows encapsulated URLs to break across lines
\usepackage[colorlinks,hypertexnames=false,plainpages=false]{hyperref} % Converts \url references to valid hyperlinks in PDF documents

\usepackage{euler}

\ifxetex
\usepackage{fontspec} % Allows usage of system opentype/truetype fonts
\setsansfont[BoldFont={* Bold}]{Miso} % Sets the Header font via system font name
\fi
\usepackage[T1]{fontenc} % Sets output font encoding to support accented characters and Type 1 fonts
\usepackage{concrete} % Swaps out ancient Computer Modern font for Latin Modern

\setkomafont{title}{\fontencoding{EU1}\sffamily}
\setkomafont{section}{\fontencoding{EU1}\sffamily\Large\centering}
\setkomafont{subsection}{\fontencoding{EU1}\sffamily\large}

% Title Information
\title{Exchange Rate Policy}
\author{\large by Linan Qiu}
\date{\small correct as of \today}
\begin{document}
\maketitle
\tableofcontents
\section{The Exchange Rate}
\subsection{Definition of Exchange Rate}
If you have ever travelled overseas, you’re probably already familiar with the concept of an exchange rate. It’s definition is pretty intuitive. Exchange rate is simply the value of one country’s currency in terms of another currency. Additionally, exchange rate policy is essentially a subset of monetary policy, though it differs quite a bit from traditional monetary policy.

H2 Econs is rather obsessed with graphical illustrations, so let’s start by showing how the equilibrium exchange rate is determined using a demand and supply diagram. The diagram shows the demand and supply of the Singapore dollar in the foreign exchange market. On the Y axis is “exchange rate”, which shows the price of S\$ in terms of another currency, in this case USD. The x axis is just quantity of S\$. By now, we should all know that the equilibrium exchange rate is the point where the demand and supply curves intersect, but how do we explain why this happens?

Let’s say the exchange rate is above the equilibrium price. From the graph we can see there is a surplus of S\$. This exerts a downward pressure on the external value of the S\$ in relation to the USD causing the S\$ to depreciate against the USD. This happens because people are willing to give up more S\$ in exchange for each USD due to the surplus. As the exchange rate of S\$ falls, the quantity demanded for S\$ increases while the quantity supplied falls. This continues until the equilibrium exchange rate is reached.

If the exchange rate was below the equilibrium price, the adjustment process is highly similar. Here we see a shortage of S\$, thus people who are short on S\$ will be willing to give up more USD in exchange for S\$. 1S\$ thus exchanges for more USD, and the exchange rate appreciates.

You should also bear in mind the difference between the terms appreciate/depreciate and revalue/devalue. The former refer to changes in the exchange rate of a currency due to free market forces, whilst the latter describe policy actions by a central bank or monetary authority of a nation to artificially change the exchange rate. The government does not depreciate its currency, but rather devalues it. We will learn more about his in subsequent lectures.
\subsection{Give Me Money (Internationally)}
In the previous lesson we demonstrated how the equilibrium exchange rate is determined using a demand and supply model. However, the foreign exchange or forex market that this is all taking place in is quite a special market, and so the factors influencing the demand for a currency are quite different from those you might have learnt in microeconomics.

This is because the demand for currency is a derived demand. S\$ are not demanded in and of itself, but rather so they can be used to purchase other g\&s. Specifically, the demand for S\$ in the forex market arises due to the demand for Singaporean g\&S by foreigners. If a person living in Canada wishes to purchase something from Singapore, he must first exchange his Canadian dollars for S\$, thus contributing to the demand for S\$.

The relative interest rates between countries can also affect the demand for a currency through short-term capital flows. If, for example, the interest rate in Singapore is relatively high, the demand for S\$ will increase as investors are encouraged to park their money in Singaporean banks to exploit the higher returns. In order to do this they must buy S\$ in the forex market, thus increasing the demand for S\$

Another factor that affects the demand of a currency is speculative activity by currency traders due to the changes in expectations regarding the future value of a currency. If these traders expect a currency to increase in value, they are likely to purchase the currency in the present and sell it in the future for a profit.
\subsection{Take My Money (Internationally)}
Understanding what factors affect the supply of a currency simply involves a change of perspective. When talking about demand, we are concerned with foreigners who wish to buy S\$. When talking about supply, conversely, we are looking at Singaporeans who wish to buy foreign currencies. To do so, they will have to exchange S\$ for another currency, thus contributing to the supply of S\$ in the forex market.

Hence the factors that affect the supply of currency are essentially the same as those affecting demand, but just from a different perspective. The demand for foreign g\&s by Singaporeans affects the supply of S\$. The greater the demand, the more S\$ they will sell, thus increasing its supply in the forex market.

Relative interest rates also affect S\$, but the explanation has to be tweaked.  When the i/r in Singapore falls relative to foreign countries, the supply of SGD will increase as Singaporeans start depositing funds overseas.

Speculative activity also affects the supply of currency. If currency traders expect the value of a currency to fall, they are likely to sell off that currency in the present, thus increasing its supply in the forex market.
\subsection{Exchange Rate Determination in the Long Run}
In the previous lesson, we’ve explained how relative interest rates as well as speculative activity can affect the value of a currency. These two factors are essential in explaining exchange rate movements in the short run, but over the long term there are other more important factors. In this lesson we will explain what these factors are.

The first factor is the relative price levels of similar g\&S between countries. To illustrate this, let’s look at two countries, Canada and the United States. We start with the premise that prices are rising faster in Canada compared to the US. From a Canadian’s perspective, this means it would be relatively cheaper for him to buy imported goods from the US rather than purchasing domestic goods. Assuming the demand for US exports is price elastic, this leads to an increase in the import expenditure on US goods, which increases the demand for USD. In order to purchase these US imports, Canadians would have to sell their currency in the forex market in exchange for USD, which causes an increase in the supply of C\$. The result is that the C\$ depreciates against the USD.

Now let’s look at things from an American’s perspective. Since prices are rising faster in Canada, this means that Canada’s exports relatively more expensive in the US. Assuming demand for Canada’s exports is price elastic, this leads to a more than proportionate fall in qty demanded of Canada’s exports. Hence the total export revenue of Canada falls, causing the total demand for Canadian dollars to fall. Once again, the Canadian dollar depreciates against USD.

Another factor that affects exchange rate in the long run is relative rates of productivity growth. Let’s say Japan’s productivity increases relative to US. This means its average COP falls, causing the price of its products to fall. The rest of the explanation is basically the same as what we have already explained, and the result is that the yen appreciates against USD.

Other than price, another factor that affects the demand for a good is preference. If people are suddenly enamoured with everything that comes out of Japan, then regardless of the relative price level of the goods it produces, the demand for its products will increase. Correspondingly, the demand for its currency will also rise, leading to an appreciation in its exchange rate.
\newpage
\section{Exchange Rate Systems}
\subsection{Free, Fixed, Dirty?}
In a free market, the exchange rate of a country would simply be determined by the demand and supply factors we’ve explained in the previous lesson. When a nation allows its currency to depreciate or appreciate freely, we say it has a free floating or flexible exchange rate.

In the real world, however, countries often have various reasons for intervening in the forex market, as with any other market. There are two main ways of doing this, which we will examine in this lesson.

Firstly, a country can adopt what is known as a fixed or pegged exchange rate system. The main reason this is done is to minimize uncertainty and prevent excessive or destabilizing fluctuations in a country’s exchange rate. Recall that a country’s exchange rate is the value of its currency in terms of another currency. This means that a nation’s currency can be measured in reference any other currency in the world. Rather than attempting to fix a currency at a specific rate relative to every other currency, which would be unfeasible, in the real world currencies are usually pegged at a specific rate of exchange to one other currency (usually that of a major trading partner).  In order to maintain the exchange rate at a certain value, the central bank must intervene to buy or sell currencies. We will learn more about this in the next lesson

Prior to 2005, both the Chinese yuan and Malaysian ringgit were fixed to the USD. Currently pegged currencies include the Hong Kong dollar, which is pegged at 7.80 per USD.

Secondly, a country can adopt a managed float or dirty float exchange rate. This system lies in between the extremes formed by the fixed and free float exchange rate regimes. Naturally, almost all economies today adopt some form a managed float system.  Under this policy, the exchange rate of a country is allowed to fluctuate, but the central bank intervenes to prevent wild and excessive fluctuations. This essentially means that there is an acceptable bank within which the currency is allowed to freely float, but beyond which the government will intervene. The size of this band varies greatly between nations.

We will learn more about how these interventionist systems are maintained in the following lessons.
\subsection{Fixing Games}
When the currency is fixed at a particular rate, there will be greater certainty among foreign investors and consumers, which promotes growth and external stability. Despite the attractiveness of such a scenario, not many countries adopt this system in reality due to the difficulties with maintaining one’s currency at a pegged rate.

In order influence the price at which one currency trades for another, the central bank has to affect the demand or supply of its domestic currency so as to counteract the free market forces that would otherwise have caused it to change. To increase the value of its currency, the central bank must buy up the currency in the forex market (increasing demand) by selling the foreign currency in its reserves. To decrease the value of its currency, the central bank must sell its domestic currency (increasing supply) by buying foreign currency.

 Let’s illustrate this graphically using the HK\$ as an example. Say the HK govt is committed to keep exchange rate at E0 in terms of USD. An increase in demand for HK\$ causes an upward pressure on HKD against USD. To counteract this increase, the government releases more HKD into the forex market by buying foreign currency (to which the HKD is pegged) in exchange for HKD. This causes a leftward shift in the supply curve, which pushes the equilibrium price back to E0. Additionally, the Hong Kong government now has more foreign reserves.

We can thus see that one of the big drawbacks of having a fixed exchange rate system is that it requires governments to amass large foreign reserves. Additionally, if a country’s fixed exchange rate is perceived as far below its free market value, as in the case of China, attempts to keep the currency devalued can be taken to be an unfair trading practice, inviting retaliation from other nations.
\subsection{Managing Games}
Singapore currently adopts a managed float exchange rate regime, where the S\$ is allowed to fluctuate within an undisclosed band. Other countries choose to disclose the size of the allowed ban. Should the exchange rate move out of the band, MAS intervenes by selling or buying currencies. This gives us the flexibility to cope with periods of uncertainty. We will learn more about how this policy works in Singapore in later lectures.

By setting an upper and lower limit for the exchange rate, the government allows for rapid adjustments in the short term via free market forces, but also retains the ability to set long term goals regarding the exchange rate. In Singapore’s case, before 2008, the MAS had a policy allowing “gradual and modest appreciation of SGD” After Oct 2008, the MAS set a new target of “zero \% appreciation of the nominal effective exchange rate policy band” amidst easing external and domestic inflationary pressures. We will learn more about the nominal effective exchange rate, or S\$NEER, in later lectures.

Additionally, the managed float system requires less foreign reserves to maintain, since some fluctuations are allowed. These reasons explain why a managed float policy, in some form or other, is so popular in today’s globalized world.
\newpage
\section{Exchange Rate and the Balance of Payments}
\subsection{Devaluation Of Currency To Correct A BOP Deficit}
Changing the exchange rate of a currency in terms of another is bound to affect the trade between the two countries involved. By altering the price of imports and exports between nations, the management of exchange rates can affect a country’s export competitiveness and thus balance of payments position. In this lesson we will look more closely at how this policy tool can be used.

Let’s say a nation is facing a persistent balance of payment deficit. One way it can correct this is by devaluing its currency relative to another. This will cause the price of the country’s exports in terms of the foreign currency to fall. Assuming the demand for its exports are price elastic, this will cause the quantity demanded for the exports to increase more than proportionately, resulting in a rise in export revenue. Conversely, the price of imports from the foreign nation in terms of domestic currency will rise. Assuming the demand for imports is price elastic, the quantity demanded for the imports will fall more than proportionately, resulting in a fall in import expenditure. This helps improve a nation’s BOT, and in turn its BOP.

Explaining how a revaluation of currency can correct a balance of payment surplus essentially uses the same explanation, so you can try to work this out for yourself.
\subsection{Marshall-Lerner Condition}
In the previous lesson, we’ve showed how exchange rate policy can be used to ensure BOP equilibrium. Yet we’ve assumed two very stringent criteria: that the PED for both imports and exports are elastic. In the real world this may not hold. In Singapore’s case for example, our lack of natural resources and small domestic market mean most of our necessities come from overseas. This means our imports tend to be price inelastic. Despite this, Singapore still adopts exchange rate policy as our preferred monetary policy. This is because the BOP will improve following devaluation so long as the Marshall-Lerner condition holds.

The M-L condition states that as long as the sum of price elasticities of demand for imports and exports is greater than 1, a devaluation of a country’s currency can lead to an improvement in BOP, ceteris paribus. This definition is very important, since you are required to state it at A levels every time you mention the M-L condition, which you must do every time you try to explain the impact of exchange rate on BOP. Thankfully, in the current syllabus you are not required to explain just why this particular condition must hold.
\subsection{Limitations Of Using Exchange Rate Policy For BOP}
In the previous lesson, we’ve explained that in order to improve BOP by devaluing one’s currency, the M-L condition must hold. This leads us quite nicely to one of the main limitations of exchange rate policy w.r.t. BOP, the J-Curve effect. This problem arises because the PED for exports and imports are likely to be low, or inelastic, in the short run. Consumers take time to change their consumption patterns and preferences away from imported goods. It takes time to find domestic substitutes for those imports. Furthermore, producers may have to fulfil the terms of prevailing contracts, and thus might not be able to change the quantity and price of imports/exports.

Because of the low PED for imports and exports in the very short run, the M-L condition does not hold. A devaluation of the currency is likely to worsen BOP in the short term. When illustrated on a diagram, this creates a J-like curve, hence the name.

Another limitation with using exchange rate policy to alter BOP is that it does not tackle the root causes of a BOP deficit. One of these reasons could be a loss in comparative advantage due to less efficient practices or increasing labour costs. Altering the currency merely serves to mask these problems, which might cause them to persist making the policy unsustainable in the long term.
\newpage
\section{Exchange Rate Policy in Singapore}
\subsection{Objectives Of Exchange Rate Policy In Singapore}
In previous lessons, we’ve learnt that Singapore chooses to use a special form of monetary policy, which involves exchange rate manipulations. This is carried out by the monetary authority of Singapore (MAS), which is our central bank. Whilst monetary policy has impacts on various macroeconomic goals, in Singapore we are primarily concerned with using exchange rate centred monetary policy to promote price stability, in other words a low and stable inflation rate.

Our government believes price stability serves as a sound basis for sustained economic growth as it promotes consumer and investor certainty. Given our reliance on the external sector, managing our inflation has a great impact on our economy as it enhances our export competitiveness. Such a situation also encourages FDI, which has positive impacts on our BOP.
\subsection{Exchange Rate Policy Instead Of Monetary Policy}
As an A Level econs student, you will probably be very familiar with the idea that Singapore is a small and open economy. Our unique economy is one of the main reasons why we adopt such a unique form of monetary policy. This lesson will explore why this is the case.
Firstly, as a small and open economy, our external sector is roughly 4 times our GDP. We are highly dependent on the external sector, which means altering our exchange rate has significant effects on economic activity.

Additionally, since we import so much relative to our GDP, including many raw materials, altering the exchange rate also has a significant impact on the price levels in our economy. Exchange rate policy is very effective in managing imported inflation, which we will learn more about in the next lesson, and thus is very useful in achieving our goal of price stability.

In contrast, our small domestic sector coupled with a small domestic multiplier mean that traditional monetary policy using interest rates is rather ineffective in Singapore.

This brings us on to the open economy trilemma, which is a hypothesis which states that it is impossible for a nation to achieve the following three policy variables at the same time:
\begin{enumerate}
\item A fixed exchange rate
\item Free capital movement
\item An independent monetary policy
\end{enumerate}
A country can only choose two of the three polices above. In Singapore, due to the aforementioned reasons, as well our desire to be a financial hub (which requires a high degree of capital mobility), we have chosen 1 and 2, thus abandoning the ability to have a sovereign monetary policy. Singapore is thus a world interest rate taker.

But what happens when we try to fix our interest rates above or below the global rate?  You are not required to explain why this trilemma in H2. A brief explanation would be that, due to our openness to capital flows, the resulting inflow or outflow of hot money would serve to push our i/r back to its original level.
\subsection{S\$NEER (No It's Not Money Sneering At You)}
In Singapore, instead of managing our exchange rate in reference to a particular currency, the MAS manages the S\$ against a trade-weighted basket of currencies of our major trading partners and competitors. The various currencies are given different degrees of importance, or weights, depending on our volume of trade with that country.
The value of the S\$ against this trade-weighted basket of currencies, with the effects of inflation not eliminated, is known as the Nominal Effective Exchange rate (S\$NEER). Recall that this has been mentioned before in previous lectures. Singapore does this because, given the large and diverse nature of our trading patterns, such a policy allows us to more precisely determine and manage the S\$. This is all the more important given our reliance on exchange rate policy to ensure price stability.
Since it is so difficult to take into account hourly or even daily changes in the foreign exchange market, the MAS manages the S\$NEER by allowing these short term fluctuations to occur and intervenes only when they become too drastic. As we have learnt in previous lectures, this is known as a managed float system.  
\subsection{Role Of Exchange Rate Policy In Managing Inflation}
In previous lessons, we’ve learnt that the primary objective of monetary policy in Singapore is price stability. In this lecture we will learn how it uses exchange rate management to achieve this goal.

Exchange rate policy affects inflation through two mechanisms: import and export prices. Let’s start with import prices. We’ve already mentioned this briefly in checkpoint 4.2. When the MAS allows the Singapore dollar to appreciate against foreign currencies, the prices of imported raw materials and intermediate goods in terms of S\$ will fall. Given our lack of natural resources and raw materials, this effect is quite significant. This helps to reduce cost-push inflation by shifting the AS curve downwards and insulates the domestic economy from price hikes overseas.

Moving on to export prices, we already know that Singapore is highly dependent on our export sector for economic growth. If the Singapore dollar were to appreciate against foreign currencies, the price of our exports will increase in terms of another currency, which decreases our export competitiveness. Since Singapore exports mostly capital-intensive, high value-added products including biotech, advanced medical tech and digital media, which tend to have price elastic demands, this increase in price will lead to a more than proportionate decrease in quantity demanded. Thus, our export revenue falls, ceteris paribus, resulting in a leftward shift in AD that reduces inflationary effects.

Do note, however, that this policy should only be used when the economy is operating close to full employment. Otherwise, the contractionary impact would severely increase unemployment levels.
\subsection{Limitations Of Exchange Rate Policy}
No macroeconomic policy is a panacea. As you will learn over the course of A level econs, the interconnectedness of the macroeconomy means that there will inevitably be trade-offs between macroeconomic goals for every policy. Some of these conflicts have already been covered in previous lessons. In the short term, for example, using exchange rate policy to improve BOP will lead to a worsening of the current account in the BOP, ceteris paribus. This is known as the J-curve effect.

Another trade-off was also mentioned in the previous lesson. In order to control inflation using exchange rate policy, we need to appreciate the currency, which hurts export competitiveness and leads to a fall in AD and national income via the multiplier process. This causes the economy to contract, which can worsen unemployment and slow economic growth. Hence, policymakers need to weigh the benefits of inflation reduction against the costs of a loss of output.

Other than its conflict with other macro goals, another limitation of exchange rate policy is the time lag. Government will never be a synonym for efficiency, as in any bureaucratic office there will be some kind of recognition, implementation and impact lag. You should be wary, however, of using this limitation in an exam, since all government policies carry it, which diminishes the value of such an evaluation. Hence, in order to gain more credit, you might mention that of all the macro policies in the syllabus, monetary policy (which means both traditional and ex/r policy) tends to have the least time lag. This is because the central bank or monetary authority of a nation typically acts independently, meaning it doesn’t need legislative or executive approval to act unlike the case in fiscal and supply-side policy.

Lastly, a limitation specific to exchange rate policy is that it is highly dependent on the availability of foreign reserves. Should a country run out of foreign currency, it would lose its ability to support the external value of its currency. Speculators, anticipating this, are likely to sell off this currency, causing a large depreciation in its value.

A final note on monetary policy in general: it is important to understand that monetary policy does not work alone in the real world. Where it is less effective, prudent fiscal or supply side policies can make up for it. In exams, you might get some credit for linking this to the Tinbergen-Theil theory of economic policy, which states that one policy instrument should be optimally used to achieve one marco aim at one time. If an economy wishes to achieve multiple macro goals at the same time, it should employ as many policy tools. 
\end{document}