\RequirePackage[l2tabu,orthodox]{nag}
\documentclass[DIV=classic,11pt,numbers=noenddot,listof=totoc,bibliography=totoc,parskip]{scrartcl}
\usepackage{fixltx2e,multicol,graphicx,url,caption,csquotes,amsmath,paralist,ellipsis,subfig,array,microtype,flafter,siunitx,cleveref,booktabs,textcomp}
\usepackage[hypertexnames=false,plainpages=false]{hyperref}
\usepackage{fontspec}
\setsansfont[BoldFont={* Bold}]{Miso}
\setmainfont[BoldFont={* Bold},ItalicFont={* Italic},BoldItalicFont={* Bold Italic}]{Courier Prime}
\title{Elasticities}
\author{Linan Qiu}
\date{}
\begin{document}
\maketitle
\tableofcontents
\newpage
\section{Concept of Elasticity}
\subsection{What is Elasticity}
Elasticity is the complex term that answers the simple question “if A changes, by how many times does B change by?”. So if I were to have a graph that has Y on the Y axis, and X on the X axis and a line like this, the elasticity of the line will simply be how much X changes if Y changes. It is that simple. So imagine we replace the Y and X axes with something that makes a little more sense. Y axis now becomes price. X axis is now quantity. And we draw a demand curve. Then again, the price elasticity of the demand curve will simply be how much the quantity changes given a change in price. So if the elasticity of the curve is 5, then when the price changes by 10\%, the quantity will change by 5 times as much, which is 50\%. Similarly, if the price changes by 20\%, then the quantity will change by 5 times as much, which is 100\%. It is important to note that the elasticity is not an absolute term. What do I mean by that? Well let’s say the elasticity of the curve is 4. The price increases by 30\%. It doesn’t mean that the quantity increases by 4. The quantity increases by 12, because 3 times 4. Again, it answers the question “if A changes, BY HOW MANY TIMES does B change by”. It is a matter of proportion, percentage, not absolute. 
\subsection{Is elasticity the gradient?}
The elasticity of a graph is not the gradient. In fact, the mathematical formulation of elasticity is change in B over change in A. With some mathematical manipulation, taking limits of change in A to be approximating zero, then you’ll realize that it is actually dB/dA times A/B. dB/dA will be the inverse of your gradient, and A/B is simply a number. Again, it is not gradient. Gradient is only the slope of your graph. But elasticity measures how much something changes if something else changes. For the more mathematically inclined ones among you, you’ll realize that gradient and elasticity are the same if the curve passes through the origin, which makes A/B a constant no matter where B is. However, don’t worry about that.
\subsection{Framework for Elasticity}
There are two things you look out for when you talk about elasticity. The sign and the value. Elasticities come as a number. Let’s start with the sign. Again, let’s bring up a demand curve. Since we say that elasticity answers the question “when price changes, by what proportion does the quantity change?”, then if the number is positive, when prices increase, quantity increases. If the number is negative, when price increases, quantity decreases. Or when price decreases, quantity increases. This is because when we have a positive sign, this means that when price changes, the quantity changes in the same direction, ie a positive proportion. So for a negative sign, you get opposite directions. In the case of a demand curve, when Price increases, quantity demanded decreases. Since this is the typical shape for most demand curves, the typical price elasticity of demand is negative. So for example, we have a demand curve with PED -5. This means that when price increases by 10\%, quantity drops by 50\%. This fits the shape of the demand curve! However, if you want a positive sign, you’ll have to look at the supply curve. This time, when price increases by 20\%, quantity increases by 100\%. The elasticity is then positive 5. You’ll get the hang of it once we move on to Price Elasticity of Demand.
\newpage
\section{Price Elasticity of Demand}
\subsection{What is PED}
PED is defined as the responsiveness of quantity demanded to a change in price, ceteris paribus. What does that mean? Simply, when price changes, by what proportion does your quantity demanded change? Phrased another way, the responsiveness of quantity demanded to a change in price. Simple. So let’s look here. Here we have a demand curve. When price increases by 20\%, quantity demanded decreases by 200\%. So what is the PED? 10? Wrong! It is negative 10, because the quantity demanded and the price moves in opposite directions! Ceteris Paribus simply means keeping all other factors constant. 
\subsection{Values of PED}
Remember we were talking about the two things to look out for? The sign and the value. Demand curves are usually downward sloping. This means that price and quantity changes in opposite direction. An increase in price will cause a fall in quantity demanded. When your Mars Bars go for cheaper, you’ll buy more of them. Thus when working with PED, you’re usually certainly working with negative figures. In fact, that is so common that sometimes we leave out the negative sign when talking about PED because everyone understands that PEDs are by default, negative. Let’s now concentrate on the value of the number. This tells us whether the demand is elastic or inelastic. Elastic is when a change in price causes a larger than proportionate change in quantity demanded. For example, if I have a PED of 5 for Mars Bars, this simply means that when price increases by 10\%, quantity demanded decreases by 50\%. This is a larger than proportionate change! 50\% larger than 10\%. This happens when the PED is more than 1. However, when your PED is less than 1, this means that, say your price increases by 40\%, then the resulting change in quantity demanded will be less than 40\%. There will be a less than proportionate change in quantity demanded. This means that the demand is inelastic. Unit elasticity happens when PED = 1. So again, when PED less than 1, inelastic. When PED more than 1, elastic. When PED = 1, unit. Simple!
\subsection{Determinants of PED}
PED varies a lot from one product to another. Why? Well here are some reasons. The number of substitutes. This is the most important determinant. The more substitutes there are for a good, the closer they are, and the more people will want to switch to these alternatives if the price of the original good rises. Say you are not a chocolate enthusiast, and you find Snickers and Mars Bars pretty similar. In that case, if, one day, I increase the price of Mars Bars, then you’ll switch to Snickers immediately. Then, the increase in price will cause a huge drop in the quantity demanded of Mars Bars. Hence, the demand for Mars Bars is relatively elastic. You can work out the example for inelastic yourself. The second factor is the proportion of income spend on the good. The more of your income you spend on a good, the more you will be forced to cut consumption when its price rises. Hence, the more elastic is the good. So let’s talk about rice. To most of us in Asia, we need rice. So it’s a necessity and there are very few substitutes. But also, it takes up a relatively little portion of your income. When the price of rice increases, it is unlikely to cause a huge dent to your wallet. You can continue buying rice despite the price increase. So you go on eating fried rice anyway. However, if the price of houses increase, then you might want to reconsider buying a huge mansion and settle for an apartment instead. The third reason is that of time period. The logic is simple. When you have more time to adjust your spending habits, you adjust them. Let’s take crude oil for example. In the short run, you’ll find that there are very few substitutes for crude oil. All your machinery still runs on crude oil. But in the long run, with research and other discoveries, you find substitutes for crude oil! Like how natural gas industry boomed with major shale gas discoveries in 2010 and 2011. Then in the long run, you will be able to change your machinery, reduce your dependence on crude oil and switch to other sources of energy. Hence, the longer the time period, the more elastic the PED. So again, here are your three reasons Substitutes, proportion of income and time period.
\subsection{Consumer Expenditure}
Consumer Expenditure is Price times Quantity. Work it out. If your Mars Bars costs \$2, and you buy 10 of them, you’ll spend \$20. Elasticity has implications for consumer expenditure. As price rises, quantity demanded falls. When demand is elastic, quantity demanded changes proportionately more than price. Thus, change in quantity has a bigger effect on total consumer expenditure than does change in price. Look at this diagram. The green rectangle shows the original expenditure, which is the price times the quantity. When we increase the price, we get a new total expenditure, which is the red rectangle. Since price increased, quantity demanded decreased. But quantity demanded increased more than proportionately, so though between P and Q, one increased and one decreased, we can easily conclude that total expenditure dropped because Q decreased more than proportionately. And that is because the PED is elastic. Again, when we look at something with inelastic demand. Let’s say the green box is our original expenditure. When we increase the price, quantity decrease less than proportionately. So what do you think will happen? The new total expenditure will be more than the original, because when we increase the price, quantity decreases, but less than proportionately. So P times Q actually increases! Notice how this differs from the case when PED is elastic.
\subsection{Taxation}
PED affects taxation as well! But before we go into something rather technical, let’s always use our common sense first. You should always do that in economics. Let’s think about cigarettes. They are highly addictive, and despite all the campaigns that our government has been doing, there is still a consistent population of smokers. If I increase the tax on cigarettes, do you think many smokers are going to give up smoking? Of course not. I think you can feel how elasticity comes into play here. Let’s draw a demand and supply diagram. This is a very price inelastic demand curve. Let’s draw in a supply curve as well so that we know where the original equilibrium price and quantity is. Let’s say this is the demand curve of cigarettes. If I add taxation to cigarettes, say a fixed tax of \$2 to every packet of cigarette sold, then the price at every quantity will increase by a fixed amount of \$2. This is different from adding a percentage tax of say 10\%, where the price will actually increase more when the original price is higher. But let’s ignore the difference first. Let’s just say that there is a fixed tax. Here we have the post tax supply curve, because, again, the price is \$2 higher at every quantity. What happens by the time we hit the new equilibrium price and quantity? We realize that the price is a lot higher, but the quantity didn’t move much at all! And this is consistent with our common sensical predictions - that cigarettes, being price inelastic because of addiction, will not have much reduction in quantity demanded due to a price increase. I’m sure you can work out the counter example for a price elastic good. Go ahead and try!
\subsection{Infinites}
Now we have several curious cases, when the demand curve kind of looks funny. There is a totally inelastic demand, i.e. PED = 0. This simply means that when price changes, quantity doesn’t change at all! Hence 0. There’s also infinitely elastic demand, when any minute change in price will cause an infinite change in quantity. PED will be equals to infinity. You’ll realize the significance of this graph when you go to market structure.
\newpage
\section{Cross Elasticity of Demand}
\subsection{What is XED}
XED is the responsiveness of demand of one product to a change in the price of another. What does that mean? It answers the question “If the price of B changes, how much does the demand of A change?” Be careful here. We are talking about demand, not quantity demanded. This definition is important for your essays. Why is this important? Well let’s say you’re the boss of Adidas. You want to know influential the pricing policy of Nike is on your products. Then, XED will be very very useful. It shows if your product is a substitute, or a complement. We’ll explore these concepts.
\subsection{Values of XED}
Again, we talk about sign and value. If the sign is positive, it means that when the price of B increases, the demand of B increases. Let’s visualize this. <2.1> If the price of Nike increases, will there be more or less people buying Nike? Less! <2.2> These people will go buy Adidas (which hasn’t changed its price). The demand of Adidas increases, and this only happens because Adidas is a substitute for Nike. So when the sign is positive, <2.3> the two goods are substitutes.  Now you must be thinking, then if the sign is negative, what happens? Well, let’s say if the price of printers decrease, what will happen to the demand of ink cartridges? Well, more people buy printers when price of printers decrease. They need more ink cartridges, so the demand of cartridges increase! So the demand of cartridges moves in the opposite direction of the price of printers, and the XED is negative! When the price of printers decrease, demand for ink cartridges increase! They are hence called complements! <3.1> They complement each other. 
Now let’s look at the value. If the value is between 0 and 1, it means that the relationship is weak. For example, in the case of a substitute, let’s take a look at a familiar example. Xbox and Playstations. Enthusiasts tell me that there’s actually quite a bit of difference between the two. So let’s assume there is. Either way, game CDs made for Xbox can’t be used for Playstation. So when the price of Xboxes increase, there might be an increase in the demand for Playstations, but not too significantly due to the differences like Kinect and some other lame windows feature. So the demand for Playstations will not increase more than proportionately, hence the value is somewhere around 0.5. Let’s take another example. Let’s say that we are Coke and Pepsi. Minute differences. So if the price of Cola decreases, then the demand for Pepsi will decrease more than proportionately because they are substitutes. <6.1> In this case, the XED will be more than 1. In the case of complements, the same thing applies. Weak complements like oil and car, because you won’t just buy a car because of a drop in the price of oil, but there is still a certain amount of influence. Strong complements like Razors and Shavers. Try and think of your own examples! It’s always good to have one example each for weak complements, strong complements, weak substitutes, strong substitutes.
If XED is zero, it means that when the price of one good changes, the demand for the other good doesn’t change at all. For example? If the price of shoes increase, what happens to the demand of TVs? Are you asking me WTF? Yes! It simply means the goods are not related at all!
\subsection{Determinants of XED}
The single largest determinant of XED is the closeness of substitute or complement. Why is this important? Firms need to know the XED for their product when considering the effect on the demand for their product of a change in the price of a rival’s product or of a complementary product. These are vital information that CEOs will love to have.
\newpage
\section{Income Elasticity of Demand}
\subsection{What is YED?}
YED is the responsiveness of demand to a change in consumer income, ceteris paribus. Hence YED answers the question “how much will the demand of A change if income changes?”. Let’s say something has a YED of 10. This means that when income increases by 5\%, the demand for the good will increase by 50\%. The value and sign of YED is actually quite important in determining the type of goods. Again, we are talking about demand here, not quantity demanded.
\subsection{Value of YED}
Do you buy more branded clothing when you are richer? You probably do. Do you buy more leftover bread from the bakery when you get richer? Probably not! You can buy better bread from exotic bakeries like Paul in Takashimaya. So here’s the difference. There are normal goods, and there are inferior goods. For normal goods, you buy more of them when you get richer. For inferior goods, you buy less of them. So for normal goods, you have a positive YED! When your income increases, demand for the good increases. Same direction! For inferior goods, negative YED! Within normal goods, there’s a further subdivision. There’s luxury goods and necessities. Think. If you got a lot richer, will you consume a lot more luxury stuff like Hi-fi systems, apple computers, branded bags and stuff? Well if you say no, I love you for being frugal, but most people do buy more of those crap. So luxury goods like those have a high and positive YED! When income increases, the demand for these goods increase more than proportionately. Same thing, when your income decreases, the demand for these goods are the first to drop, hence decrease more than proportionately. However, if you get richer, you won’t eat a lot more rice! You probably already have enough rice to eat. Same for butter. Or bread. You’ll consume maybe more, but not a lot more. Hence the YED will still be positive, but only a small value between 0 and 1. The demand for those goods will increase less than proportionately if your income increases. Similarly, if you suffer a drop in income, you’re probably not going to consume A LOT less rice. So the demand will drop less than proportionately. To summarize, under normal goods, there are luxury goods and necessities. Luxury goods have a high elastic YED, necessities highly inelastic YED.
\subsection{Determinants of YED}
The major determinant of YED is first, if the good is a normal good or an inferior good, and second, if it is a normal good (usually), then how necessary is it. Income elasticity of demand is an important concept to firms considering the future size of the market for their product. If the product has a high income elasticity of demand, sales are likely to expand rapidly as national income rises, but may also fall significantly if the economy moves into recession.
\newpage
\section{Price Elasticity of Supply}
\subsection{What is PES}
PES answers the question “how much does the quantity supplied change if price changes?”. The formal definition is the responsiveness of quantity supplied to a change in price, ceteris paribus. So let’s say we have a supply curve. Price increases by 10\%. If the quantity supplied increases by 20\%, then the PES is 2. PES is usually positive.
\subsection{Different Values of PES}
Again, we look at our two items - sign and value. Supply curves are usually upward sloping, hence the sign is usually positive. The value however says a lot. If the value is 0, it is a perfectly inelastic supply curve because no matter how the price changes, the quantity supplied stays the same. The quantity supplied does not change because of the PES being 0. If the PES is between 0 and 1, then the supply is inelastic. A change in price will cause a less than proportionate change in quantity supplied. If PES is 1, it is unit elastic, just like your PED. If PES is more than 1, then it is elastic! Any change in price will cause a larger than proportionate change in quantity supplied. But what causes these differences?
\subsection{Determinants of PES}
Just like PED, there are determinants for PES. Imagine you have a farm. If I ask you to produce 100,000 tonnes of oranges when you can only produce 10,000, you’ll find it hard even if I increase the price. That’s because you don’t have a lot of spare equipments, you don’t have extra supplies of land, and your orange trees simply cannot grow in time. You simply cannot expand. However, if you owned a factory producing iPad. I ask you to produce 100,000 iPads instead of 10,000 for this month. Simple! You just have to churn your machinery more, and since the spare parts are ready and your workers can just work overtime, you can easily meet the target. So the first reason is your supply capabilities. So for the farmer, I’ll have to give you very huge price increases to reap little increase in quantity supplied. For iPad, I don’t even need to offer you a large amount of money for you to produce a lot more iPads. The second is again Time Period. In the short run, production capacity is nearly fixed because the amount of raw materials and equipment you have is rather limited. In the long run however, you can acquire new technology, have new equipments and import new raw materials. Your supply will be a lot more price elastic.
\end{document}