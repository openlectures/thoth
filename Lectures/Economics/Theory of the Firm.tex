\RequirePackage{../../dominatrix}

\title{Theory of the Firm}
\author{\large by Linan Qiu}
\date{\small correct as of \today}
\begin{document}
\maketitle
\tableofcontents
\section{Factors of Production and Time}
\subsection{Fixed Factors And Variable Factors}
So in 2010, Apple got into a lot of trouble for violating human rights, particularly at one of its factories - Foxconn. But let's ignore that first. Let's say you're the boss of Foxconn. If you want to increase production, it will take time to acquire a great quantity of raw materials. You need to use more electricity as well, so you need to turn on more switches. But you might need to increase the number of machinery as well! But if that happens, it might take longer! It will take longer still to build a second or third factory. Hence if the firm wants to increase output in a hurry, it will only be able to increase the quantity of certain inputs. If can use more raw materials, more fuel, more labor. In this case, Foxconn chose to overwork its workers, hence causing the human right issues. But it will have to make do with its existing building and machinery. Here, there's an important distinction between fixed factors and variable factors. Fixed factors is an input that cannot be increased in a given time period. A variable factor is one that can. So back to Foxconn. Your fixed factor is probably your machinery, your building and perhaps your technology. Your variable will be the amount of raw material you use, your labor and perhaps the fuel or electricity. This links us to our next concept.
\subsection{Short Run And Long Run (Not Skirt And Jackets)}
Short run and long run in economics are very precise terms, unlike how we use them in every day life. The short run is the time period during which at least one factor of production is fixed. This well, no surprise, happens to be your fixed factor. The long run is when all your factors of production can be changed, even including the technology, machinery. So you will realize by now that short run and long run isn't specific. Every firm has a different short run and long run. If you're a hawker centre uncle, your long run will be the amount of time it takes for you to acquire another shop space. That'll be around 2 months or so. However, for Foxconn, to get another piece of land from the Chinese government takes at least a year. And then to move in all the machinery and stuff, it takes at least another half a year. So short run and long run differs for every firm.
\subsection{Law Of Diminishing Returns}
In the next few lessons, we'll look at production in the short run, long run, and costs in the short run and long run. Then we will look at how that translates into revenue and eventually profits. One of the most important laws you'll learn in economics is that of diminishing returns. Let's say that you're the boss of Foxconn again. Your fixed factor will be your factory space and your equipments, and the variable factor is your labor. Since land is fixed in supply, output can only be increased in the short run by increasing the amount of workers employed. So let's imagine what happens when you add workers to the factory. At first, you have an empty factory, so when you add workers, the total product increases by a lot! Why? Because when you only have one worker, he does his own stuff. When you have two workers, you might be able to produce more than what 2 times what they individually produce! They can help each other, do things that is impossible for one of them to do alone. That's good right? So you continue adding workers. But eventually you realize that when you add more workers, there's less of this synergistic effect. When you add your first specialized worker in, he might be able to impart all his skills to non specialized workers. But when you add more specialized workers, it might not have the same extent of effect. As you keep adding more and more people, the workers start getting in the way of each other! Efficiency slows down. If beyond this point, you keep adding workers, you might produce a negative effect! Your factory is way too crowded, and no one can do anything! Hence this is the Law of Diminishing Returns - when one of more factors are held fixed, in this case the size of your factory, there will come a point beyond which the extra output from additional units of the variable factor, in this case labor, will diminish.
\subsection{Total Cookies}
So let's try and plot a graph out of what we have from the Law of Diminishing Returns. When you have zero workers, you produce nothing! When you have 1 worker, you produce 3 iPads. When you have 2 workers, you produce 10, not just 6, because remember there are efficiency effects! When you have 3 workers, you produce 24 iPads! When you have 4 workers, you produce 36. Now slowly you see that the effect is diminishing. When you have 5, 40. 6, 42. Now, when you have 7 workers, the 7th fella doesn't actually add anything much! So Total product remains at 42 units. When you add the 8th guy in, he actually crowds up the factory so much he causes a negative impact on the total product, bringing it back down to 40. When you plot a graph out of this, this is your total product graph given the influences of the Law of Diminishing Returns.
\subsection{Average Cookies And Marginal Cookies}
Now let me introduce two concepts - that of average and marginal. Average is pretty simple to understand. Let's bring this graph back again. Take for example, when you have 5 workers, you produce 40 units of iPads. So what is the average amount that each worker produces? That's 8. When you do this for every worker number, you'll see something interesting. Your Average Product peaks at 4 workers, but your total product peaks at 7 workers. What's going on here? Well, simple. At the 5th worker, the total number of products you produce still increases. But the 5th worker doesn't increase the total number by a lot, so he actually brings DOWN the average, even though he adds on to the total. Think about this on the toilet bowl and you'll realize how this makes sense. Let's move on to the second concept - that of marginal. It answers the question "how much will my total product increase if I increase the number of workers by 1?". So you can see here, from 0 to 1 worker, by that addition of 1 worker, I'm adding 3 units to the produce. So the marginal product is 3. From 1 to 2, I'm adding 7 units, bringing the total product to 10. So the marginal product is 7. you'll see that the marginal product peaks at 3 workers. Why is that? Because it's the 3rd worker that brings in the most ADDITIONAL units of iPads. the 4th worker brings an additional 12 units, much less than what the 3rd worker brought in. However, do take note that the 4th worker, even though he brings in an additional amount that is less than the 3rd worker, still brings in something positive, unlike the 8th worker. I'd suggest you take some time to work this out on a piece of paper and watch this checkpoint again if you are still scratching your head.
\subsection{Classroom of Average and Marginals}
So we have just encountered our first relationship between average and marginal. This is recurring concept that will occur many many times in economics. And there are some general rules that we can explore. There are only 3 of them. Let's use an example. Imagine a room with ten people in it, and that the AVERAGE age of the people is 20. You are smart, and you know that if the average age is 20, it doesn't mean that everyone is 20 years old. Some might be thirty. Some might be forty. Some might be ten. But the AVERAGE is 20. Now if a 20 year old enters the room, making the room 11 people, this will not affect the average age. So in this case, the marginal age is 20. But the average age does not increase. So now the room has 11 people. If a 56 year old walks into the room however, the average age will rise. Not to 56, but to 23. This is found by adding everyone's ages up, which is 276, and dividing by the number of people in the room, which is 12. If not, a child of age 10 is to walk into the room, this will pull the average down. So we can deduce these rules. If the marginal equals the average, the average will not change. If the marginal is above the average, the average will rise. If the marginal is below the average, the average will fall. Think about this for a while. Continuously for 5 days at least. You'll get the hang of it.
\section{Costs in the Short Run}
\subsection{Revisiting Opportunities}
When economists measure cost, they always use opportunity cost. Opportunity cost is the cost of any activity measured in terms of the sacrifice made in doing it: in other words, it is the second best alternative which is forgone. Let's use an example. To you, going for a jog for 2 hours might be free! You don't even have to spend any money to go for a run. But in those two hours, you could have done something else, perhaps work for 2 hours and earn 10 dollars! Then in this case, by going for a run, you missed out on the next best alternative which is to work for 10 dollars. You are missing out on the ten dollars, hence you incur an opportunity cost of \$10! How does this come in useful when we investigate the costs incurred by firms? Well let's again use the example of Foxconn. Let's say that you have to pay \$100 for electricity. So when you use \$100 worth of electricity, you pay \$100. Very simple. The opportunity cost is \$100, because those \$100 could have been spent somewhere else. This is what we call explicit costs. You are buying something that is not owned by you. There's a separate idea, called implicit costs. It is for something already owned by you. Now you said "wait a minute, how come I'm incurring costs for something that is already owned by me? It just sit there what." Correct! But when it sits there, it\ldots you know how we like to say "sit there and rot"? That's exactly what the equipment does. There is opportunity cost to the equipment lying around. Let's say you own a building. You pay for rental in a yearly block. Let's say it's only May now, and you still have 7 months left. Even if you don't operate your factory, there is an opportunity cost of 7 months worth of rental because you already paid for it. There is opportunity cost for not renting it out to another firm to use. Let's use another example. You're the owner of a firm. You sit there and do nothing. Are you incurring opportunity cost? Yes! Because you could have been working for someone else instead of starting your own firm and doing nothing. In this case, if you had been working for someone else, and you get \$5000 a month, then the opportunity cost of starting a firm instead and earning nothing is \$5000 a month! If something has no second best alternative, then it incurs zero opportunity cost. For example, if you have a machine that you leave idle. You should be incurring opportunity cost right? But if the machine can't do anything else other than be left idle (i.e. it's a broken machine), then it doesn't actually incur opportunity cost! The idea of opportunity cost is one of the most powerful ideas in economics, and easily confuses people who do accounting rather than economics. 
\subsection{Total Cost}
There are two kinds of costs - Fixed Costs and Variable Costs. It's a very obvious difference. Fixed costs are costs that do not vary with the amount of output produced. For example, again, you're the owner of Foxconn. You have a factory. The rental on the land is fixed no matter how many iPads you produce! You produce 10,000 iPads, the rental of the land is still a certain price. You produce 20,000, it's still the same price. So fixed costs don't change when you vary production. Compare this to variable costs. It does vary with the amount of output produced. One great example is labor! To produce 20,000 iPads instead of 10,000 you need a lot more labor! You have to pay your workers more, so labor is a variable cost. It varies with the production level. Total cost is simply Total Fixed Cost plus Total Variable Cost. To examine the shape of the Total Cost curve, we have to look at the two components, fixed and variable.

So let's assume that you're the owner of a factory. The fixed cost is assumed to be fixed at \$12 dollars. It does not change with production. The Total variable cost however changes. With zero output, no variable factors will be used. The TVC curve is hence at 0 when quantity produced is zero. The TVC curve follows the law of diminishing returns. Initially, before diminishing returns set in, more workers are taken at first. Workers can be increasingly specialized and make fuller use of equipment. This corresponds to the part of the production curve that rises rapidly - the first part of the production curve. TVC hence increases not so fast, because the average product increases. The product per worker increases. If one worker, with a fixed pay, can produce more products, then the costs do not increase as fast. However, when we expand output, what happens is that the the extra workers produce less and less extra output. Hence the extra units cost more, because on average, each worker produces less! Hence TVC rises more and more rapidly. This corresponds to the part of the Production Curve that rises less rapidly. 

 Since TC = TFC + TVC, the TC is simply the TVC shifted up vertically by the height of the TFC.
\subsection{Average And Marginal Costs}
Again, average and marginals! These concepts keep appearing in economics don't they. Average Cost is the cost per unit of production, defined as AC = TC/Q. Hence if it costs a firm \$10,000 to produce 50 units of goods, the AC is \$200. Just like TC, AC can be divided into AFC and AVC. AC = AFC + AVC. Very simple. The AFC decreases as output increases, since total fixed cost is constant. Hence when you divide it by increasingly larger quantity, AFC simply decreases. It is being spread over a larger and larger output. The AVC again reflects your Law of diminishing returns. As the average product of workers rises before diminishing returns set in, the average variable cost decreases. Think about it this way. The average product of your worker is the number of iPads he can produce given his salary of say \$4000 dollars. If he can produce 10 iPads, then the cost of one iPad is \$400, discounting other factors of course. So the more products the worker can produce, the less the average variable cost. The less products the worker produces, the more the average cost. So when the workers still produce increasingly more products before diminishing returns set in, AVC decreases. However, once diminishing return sets in, he produces less, average product decreases, and AVC increases. Hence the U shape for your AVC. Your AC is simply the addition of the AVC and the AFC. You'll realize that as the AFC decreases to nearly zero, the AVC nearly touches the AC. But it never touches! Because AFC never becomes zero. 
 Let's move on to marginal costs. Do we talk about the marginal fixed costs? Of course not! The fixed cost is a lump sum we pay at the start. Any additional unit incurs zero marginal fixed cost. Think about this. To produce 0 units of iPads, you still have to pay the rental. To produce 1 unit of iPad, you pay the same rental too. No additional fixed cost! 2 units of iPad? Same fixed cost! So what is the marginal fixed cost? Zero! Does that mean that your fixed cost is zero? No! It just means that the additional fixed cost per unit of good is zero. So when we talk about marginal costs, we only care about variable costs. The shape of the MC follows directly from the law of diminishing returns. As more variable factor, for example labor, is used, extra units of output cost less than previous units. Remember this from your AC? Beyond a certain level, diminishing returns set in. Each additional unit of labor brings in less additional product. So the marginal cost increases. Marginal cost keeps increasing because additional workers bring in less and less products, to the extent that it eventually brings in negative additional product. This is why your MC keeps increasing.

Your teacher will tell you that the MC always cuts the AC at the bottom point. You go "WHY?!" Now remember the checkpoint we had about average and marginals? So here we go. When the marginal cost, the cost of an additional unit, is below the average, the average always decreases! When the average is 20 years old, and a 10 year old boy comes in, the average decreases! When the average is 20 years old, and a 11 year old boy comes in, the average still decreases! Same for 12. So even when the marginal age increases, the average age still decreases! Same here! Even when the marginal cost increases at this portion, the average cost can still decrease. Only when the marginal cost is higher than the average cost does the average cost increase as well. This is another example of the relationship between all averages and marginals. Again, think of the age example and you will get it.
\subsection{The Math Of Averages And Marginals}
Let's get geeky here. Let's assume TC as a function of Q, hence $TC = f(Q)$. $AC = \frac{f(Q)}{Q}$ very simple. $MC = \frac{\delta TC}{\delta Q} = f\prime(Q)$. To show that the MC always cuts the AC at the bottom of the AC, here's what we do. First we find the bottom of the AC curve. We differentiate AC with respect to Q. Using the product rule, we get $\frac{\delta AC}{\delta Q} = \frac{f\prime(Q)}{Q} - \frac{f(Q)}{Q^2}$. then we set $\frac{\delta AC}{\delta Q}$ to be zero. Multiply by Q on both sides and then we shift them around, we get $f\prime(Q) = \frac{f(Q)}{Q}$. Notice that your $f\prime(Q)$ is simply your MC, and $\frac{f(Q)}{Q}$ is your AC. So this shows that when $\frac{\delta AC}{\delta Q} = 0$ i.e.\~the bottom point of your AC, MC = AC, hence MC cuts AC at the bottom part of AC! Next we show that when MC>AC, AC rises. If $f\prime(Q) > \frac{f(Q)}{Q}$, $\frac{f\prime(Q)}{Q} > \frac{f(Q)}{Q^2}$ since Q is always positive. Then $\frac{f\prime(Q)}{Q} - \frac{f(Q)}{Q^2}> 0$. $\frac{\delta AC}{\delta Q} > 0$. Hence, AC rises. We can also use this to show that when MC is less than AC, regardless of whether MC is rising or MC is decreasing, AC still decreases. to check if MC is decreasing or increasing, simply differentiate MC with respect to Q. 
\section{Long Run Theory of Production}
\subsection{Thinking Big}
In the long run now, all factors of production are variable! As the boss of Foxconn, you can vary the factory size, land size, equipments. Everything! Hence, you are making decisions in terms of scaling your entire factory. If I place my factory at this size, what happens. If I double the size, what happens. Then, we go into investigating the scale of production. We want to know that should Foxconn double its production, would it double its output? Will it more than double or less than double? We can distinguish 3 possible scenarios. Constant, increasing and decreasing returns to scale. For constant returns to scale, this just means that when you increase your firm size by a certain percentage, say 10\%, then output increases by 10\% as well. The same percentage! For increasing returns to scale, this is where a 10\% increase will cause a 20 to 30 or even more percentage increase. A given percentage increase in inputs will lead to a lrager percentage increase in output. Decreasing returns to scale simply means a given percentage increase in inputs will lead to a smaller percentage increase in output. Now does this mean I increase raw materials by 20\%, does not increase labor, and still my output will increase by 10\%? No! To scale means to increase everything by the same proportion. Decreasing returns to scale is therefore VERY different from diminishing marginal returns (where only the variable factor increases). This table very aptly sums up the differences between diminishing returns and decreasing returns to scale. Notice how in decreasing returns to scale, the inputs increase at the same rate. 
\subsection{When I Grow Up I Wanna Have Economies Of Scale}
Now that you know what increasing returns to scale is, we now investigate WHY there is increasing returns to scale. This is due to something called economies of scale. Economies of scale happens when increasing the scale of production leads to a lower cost per unit of output. But why? Here are some reasons. Specialisation and division of labour. In large-scale plants workers can do more simple, repetitive jobs. With this specialisation and division of labour less training is needed;workers can become highly efficient in their particular job,especially with long production runs; there is less time lost in workers switching from one operation to another; and supervision is easier. Workers and managers can be employed who have specific skills in specific areas. Indivisibilities. Some inputs are of a minimum size: they are indivisible. The most obvious example is machinery. Take the case of a combine harvester. A small-scale farmer could not make full use of one. They only become economical to use, therefore, on farms above a certain size. The problem of indivisibilities is made worse when different machines, each of which is part of the production process,are of a different size. For example, if there are two types of machine, one producing 6 units a day, and the other packaging 4 units a day, a minimum of 12 units would have to be produced, involving two production machines and three packaging machines, if all machines are to be fully utilised. The ‘container principle'. Any capital equipment that contains things (blast furnaces, oil tankers, pipes, vats, etc.) tends to cost less per unit of output the larger its size.The reason has to do with the relationship between a container's volume and its surface area. A container's cost depends largely on the materials used to build it and hence roughly on its surface area. Its output depends largely on its volume. Large containers have a bigger volume relative to surface area than do small containers. For example, a container with a bottom, top and four sides, with each side measuring 1 metre, has a volume of 1 cubic metre and a surface area of 6 square metres (six surfaces of 1 square metre each). If each side were now to be doubled in length to 2 metres, the volume would be 8 cubic metres and the surface area 24 square metres (six surfaces of 4 square metres each). Thus an eightfold increase in capacity has been gained at only a fourfold increase in the container's surface area, and hence an approximate fourfold increase in cost. There are also byproducts. With production on a large scale, there may be sufficient waste products to enable them to make some by-product. For example, an oil cracking company may also own certain plastics companies because plastics are usually the by product in oil fractional distillations. There are also organisational economies. With a large firm, individual plants can specialise in particular functions. There can also be centralised administration of the firm. Often, after a merger between two firms, savings can be made by rationalising their activities in this way. When you borrow money as a large firm, you might be able to get lower interest rates as well. 
\subsection{Too Big?}
When you get too big, things are not going to be all good either! You can't just keep expanding. In fact, when you get beyond a certain size, cost per unit of output may actually start to increase. This is what we call diseconomies of scale.

 When you work in a very large firm, you might feel that the boss is very distant from you. You feel like a nobody, and that feeling sucks. Hence you don't work as hard, and in multinational corporations where the hierarchy is very very significant and huge, this happens very often! Communication takes a long time as well. For the CEO of Shell to implement something in the oil refinery in Singapore might take him a long time as well due to the long chains of communication. Industrial relations may deteriorate as a result of these factors and also as a result of the more complex interrelationships between different categories of worker. This happens in National Service in Singapore as well, if you know what I mean. At the end of the day, the conditions for each individual firm decides when economies of scale happens and when diseconomies of scale starts to kick in.
\subsection{External Economies Of Scale}
You guys heard of Silicon Valley? Now think. Each of those companies inside there might vary in size. But when you congregate these many IT firms together, they might enjoy what management consultants these days call synergy, or what I'd prefer to call the advantage of being close to each other. Why? They can share best practices, technology, share common services like data centers and security, mass import raw materials and access a common talent pool like investors and venture capitalists. What we are referring to is the infrastructure - the facilities, support services that can be shared. And governments nowadays try to foster an environment like this. So what is external economies of scale? It is when the firm's costs per unit of output decreases as the size of the industry grows. So when Silicon Valley happens, instead of a IT firm stuck in the middle of say Arizona, the costs for the firms inside actually decrease! That's why Silicon Valley is such an attractive place for IT firms.
\subsection{Size Matters}
Since economies of scale matters so much, why don't all firms become big? Well here are some reasons and examples. Diseconomies of scale might just kick in very early for your firm. At the same time you might be a luxury items shop that thrives on the fact that you have a very limited supply. Look for one of those exotic shoe makers in Italy and you'll realize that they make such a limited quantity, in such awesome quality that you not only have to pay outrageous prices to get them, but you have to be the shoe maker's friend. This leads on to the third argument, which is that of quality. Your Wagyu beef is high quality, yet you can't just increase production! The climate has to be right, rearing techniques have to be right. Same for wines like Lafite. Though the quality has been going down in recent years due to climate change, it thrives on being very very high quality wine. Or simply there might be no reason for the shop to become big. Ask a food stall owner why his stall is so small. Why doesn't he open a huge restaurant instead. He'll tell you "I don't need to! I'm earning enough". And isn't that good? If everyone wants to grow into restaurants the size of McDonalds, where can we find nice little takeaways anymore. These arguments help explain the existence of tiny little firms. 
\section{Costs in the Long Run}
\subsection{Long Run Average Costs}
Now let's talk about long run costs. Since there are no fixed factors i the long run, there are no long run fixed costs. Everything is variable! You can rent more land, buy more equipments. Hence, initially, when a firm expands, he'll face economies of scale, and thus a downward sloping Long Run Average Cost Curve, or the LRAC curve. It will flatten out, showing a region of constant returns to scale. Then it will increase, showing diseconomies of scale as he gets really too big. This is rather commonsensical, but wait a minute. You should actually imagine the LRAC as many many possibilities of the firm having different sizes. For example, at this point on the graph, you should imagine the factory as 1 hectare big. At this point, everything stays the same, just amplified 2 times, at 2 hectare big. So it is basically scaling up your factory. 
\subsection{Relationship Between The LRAC And The SRAC}
So we have two ACs here. The natural question to ask is "what is the relationship between the two?" Let's say you're Foxconn, and you have 5 factory sizes to choose from, A B C D and E, A being the smallest hence it is the SRAC on the left, and E being the largest, with the SRAC on the right. B and C are lower than A and C is lower than B, because of the effect of economies of scale. As the firm grows, it faces a lower unit cost. D and E, there's diseconomies of scale. So as the boss of Foxconn, you choose what level you want to produce at when you plan. But once you chose the size, you can't back out! Because in the short run, your fixed factors are well, fixed. so if you produce very little, you'd choose the region selected here If you produce more, here and so on .The LRAC is hence these regions, because these dictates the costs you'd incur at if you can vary all your factors. Now imagine you had an infinite number of choices which is a more accurate reflection of real life because you can pretty much expand in very small increments. Then your choices will lie along the edges of these SRACs, hence forming your LRAC. 
\subsection{Minimum Efficient Scale}
The MES is a complex way of stating the simple idea of "the bottom point of the LRAC". But what is so significant about that bottom point? Well you will realize that at this point, the economies of scales runs out and diseconomies kick in. Ideally, as the owner of a firm, you should strive to expand till this point, because before this point there's still benefits to be gained. After this points, diseconomies kick in and there's no further benefit. So this point is ideal. But how big is the MES? Again, that varies for every firm, and this is actually the interesting part. For example, the MES for a food court stall might only be 10 square meters. But the MES for a petrochemicals firm might be in the scale of billions and billions of dollars. Then in this case, you'll realize when you learn the next topic that some firms, if they are to reap economies of scale, will end up being the only firm in the market because they have to be so damned big. Keep this idea at the back of your head while we move on.
\section{Revenue}
\subsection{Total Revenue}
Now give yourself a huge pat on the back because we have just finished costs. Now on to revenues. Same thing, total marginal and average. Total revenue (TR) Total revenue is the firm's total earnings per period of time from the sale of a particular amount of output (Q). For example, if a firm sells 1000 units (Q) per month at a price of \$5 each (P), then its monthly total revenue will be \$5000: in other words, \$5 × 1000 (P × Q). It's that simple!
\subsection{Average Revenue}
The Average Revenue is simply the total revenue divided by the quantity, which is also the amount the firm earns per unit sold. If the firm gets \$5 per unit sold, that means you're paying \$5 per unit. That means that the price is \$5. so AR is the price as well. You have just learnt demand and supply. So the AR curve is also your demand curve Why? Because the demand curve shows the price at which you are willing to buy things at various quantities. That price is your average revenue as well! So your average revenue curve is also your demand curve
\subsection{Marginal Revenue}
Marginal revenue is the extra total revenue gained by selling one more unit (per time period). So if a firm sells an extra 20 units this month compared with what it expected to sell, and in the process earns an extra \$100, then it is getting an extra \$5 for each extra unit sold: MR = \$5. Thus: MR = ΔTR/ΔQ. It is hence the slope of the TR curve. 
\section{Profit Maximisation}
\subsection{Bling Bling}
Cool now we combine everything together. We are interested in profit! Who isn't? So what's the most obvious way of getting profit? Simple! We take Total Revenue minus Total Cost. Revenue minus Cost equals profit! So TR - TC = Profit. How straightforward. But that's not all. In economics, there are three types of profits - subnormal, normal and supernormal. Subnormal profit is when you have total costs higher than total revenue. It also means that you are making an economic loss. Normal profit means that your Total Revenue = Total Cost. It means zero profits. But you think, "hey why will I do a business if I earn zero profits? I mean if I sell mars bars at \$2, and the cost of the mars bar is \$2, why will I sell it?" Well, that's because you forgot opportunity cost. Remember, when we talk about costs in economics, we use the concept of opportunity cost. This means that even when you're earning zero profits, you are more than well compensated for. Let's think of it this way. You are a trader. The cost of your goods is \$2000 dollars. You sell it for \$4000 dollars. Your second best alternative is to be a worker, which earns you \$2000. So this time, your total economic cost will be \$4000. Since your revenue is \$4000 dollars, you are earning zero profits! Yet, you are compensated for your time because you are earning the same amount as your second best alternative, which is to be a worker! So in economics, zero profits does not mean zero accounting profits. That's what accountants do. Instead in economics, we consider opportunity cost. Supernormal profits is when your Total Revenue exceeds your Total Cost. That means you're earning more than your opportunity cost! You're earning something extra.
\subsection{Using Total Curves}
Total Profit can be found by subtracting Total Cost from Total Revenue. That's simple right? So to find the maximum amount of profit, we should find the quantity of production where the largest difference between the total revenue and the total cost occurs. So let's draw our curves. Here we have a total revenue curve and a total cost curve. All we need to do is to calculate the distance between the two graphs. When the Total Cost curve is above the total revenue curve, there is a subnormal profit. When the total cost curve is below the total revenue curve, there is supernormal profit. We find the area that has the largest distance between the TR and the TC, and that the TR is above the TC. We find that it is at this quantity. That's how we find the maximum profit.
\subsection{Using Marginal Cost And Marginal Revenue}
However, you rarely have the TC and TR curves given to you. That is when you need to use the concept of marginals to help you determine which quantity is profit maximized. Then you use your average revenue curve to help you find the amount of profits that you earn. Confused? Follow me step by step. If profits are to be maximized, MR = MC. Why so? Well if MR was higher than MC, one additional unit adds more revenue than the amount of cost it requires. Because MR is higher than MC. So the firm should produce more to earn more profits. If MR is less than MC, then one unit less adds more to profit than if the unit was produced, because producing that additional unit will incur more cost than revenue. So only at MR = MC, profit is maximized. All you need to do then is to find the quantity at which MR = MC, let's say it's at 100 units. At 100 units, you take 100 times the AR, you get the Total Revenue. you take 100 times AC, you get total cost. At this step, it becomes really simple finding the total profit, which is simply total revenue minus total cost.
\subsection{Profiting From Calculus}
The MR = MC rule can be proved using very very simple calculus. Total Profit = Total Revenue - Total Cost right? So how do we find the maximum profit? Simple! We differentiate Total Profit with respect to quantity. We do that for the entire equation, we get $\frac{\delta TP}{\delta Q} = \frac{\delta TR}{\delta Q} - \frac{\delta TC}{\delta Q}$. Then we equate $\frac{\delta TP}{\delta Q} = 0$ because again, we want to find the maximum profit. So it should be at a turning point. Then we swap things around , we get $\frac{\delta TR}{\delta Q} = \frac{\delta TC}{\delta Q}$. What do you see here? $\frac{\delta TR}{\delta Q}$ is your marginal revenue! Remember, MR is the revenue gained by an additional quantity of output. That's the gradient of your TR graph! Which is $\frac{\delta TR}{\delta Q}$! $\frac{\delta TC}{\delta Q}$ is your MC. So when MC = MR, you get your maximum profit! It's that simple.
\subsection{Marginalist Principle}
You'll find that the MC=MR theme occurs a lot in economics. For example, when you are a consumer and you decide how many oranges to consume today, you'll consume until your marginal benefit - now we talk about benefit here because you're not earning revenue by consuming oranges. Instead you are getting benefits. You will consume until your marginal benefit equals your marginal cost. This is commonsensical? Not quite! Most of us think that we should consume until total cost equals total benefit. But that simply means you're not getting any additional benefit at all! Instead, you should consume until your marginal benefit equals your marginal cost. If you're buying apples at \$1 each, you should keep consuming apples until the amount of happiness you get from an additional apple is only \$1. The total doesn't matter! So let's think about this scenario. You are waiting for a bus. You have already waited 10 minutes. Many of you go like "Oh I've already waited 10 minutes. I shouldn't waste those 10 minutes I've already spent!" But that's actually irrational behavior! Those 10 minutes are what we call sunk costs in economics. You cannot retrieve it back! You should instead think about what an additional minute of waiting will give you in terms of benefits, and what an additional minute of your time costs. What is already spent doesn't matter! Only the additional quantity matters.
\end{document}
