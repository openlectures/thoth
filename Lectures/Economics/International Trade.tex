\RequirePackage{../../dominatrix}

\title{International Trade}
\author{\large by Hu Fangda}
\date{\small correct as of \today}
\begin{document}
\maketitle
\tableofcontents
\section{Why We Trade}
\subsection{Definition of Trade}
This is a pencil. There is not a single person who knows how to make it. Not me and not you. This pencil only exists because of a network of producers and consumers linked together by international trade. It is here only because a Kalimantan lumberjack was free to trade his lumber with a manufacturing subcontractor in Wuhan who was in turn free to trade with an office supply multinational company based in Osaka who in turn was free to trade with a US bookstore chain who was then free to sell this pencil to me, a writer based in Singapore.

But first let's define some terms. We say that international trade happens when an utility maximizing consumer in one country buys a good or a service from a profit maximizing producer in another.

Note that there is no such thing as somewhat free trade or mostly free trade or regulated free trade. Free trade is the ability for goods to travel between countries without any restriction. Thus, either you have free trade or you do not. Also, free trade is not globalization. Rather, it is a necessary but insufficient condition needed for globalization to occur. Globalization is much broader concept - it covers the flow not only of goods but also labour and capital.
\subsection{Enablers of Trade}
New technology is by the far the largest enabler of free trade. In 1956, most cargo was loaded and unloaded by hand by longshoremen. Hand loading a ship by the break-bulk method cost \$5.86 a ton at that time. Using containers today, it costs only 16 cents a ton to load a ship, a 36-fold savings. Containerization also greatly reduced the time to load and unload ships. New transport improvements like canals, jet travel and bigger supertankers and Panamax ships have accelerated the growth of trade by enable huge economies of scale in shipping.

Equally important in the rising affluence of many of the world's developing countries. In 2011, the per capita disposable income for Chinese urban dwellers rose by 14.1 percent to 21,810 yuan (3,493 U.S.D) - leading to growing tastes for foreign luxury goods. 15\% of Apple's revenue is from China and more than 2 million iPhone 5s were sold in China in just three days after its launch.

Political developments in favour of free trade have eradicated most of the barriers previously erected by governments. The Uruguay Round of trade negotiations led to a reduction of over \$100 million in tariffs. The 153 country General Agreement on Tariffs and Trades was the first attempt to reduce worldwide tariff barriers to trade and would lead to the formation of the WTO. India's import-weighted average tariff declined from 87.0 percent in 1990-1991 to 24.6 percent in 1996-1997. Liberalization of the Chinese market in 1970 saw export receipts boom from \$970 million to \$970 billion in less than 10 years.
\subsection{Definitions of AA and CA}
Economists everywhere agree - we love trade. Joesph Stiglitz loves his trade and Paul Krugman certainly loves his. Economists love it so much that we want to date trade, get hitched and have little free trade babies. Generally, we see it as being Pareto efficient - if two economic agents voluntary undergoes trade, it must necessarily mean both experiences net welfare gains from the transaction. Trade allows for Pareto gains. But it will be hard to see without understanding the principles of Comparative and Absolute Advantage.

Fundamentally, free trade exploits Adam Smith's theory of specialization - which he illustrated using his pin factory. I alone will find it very hard to make more than 10 of these a day. But if I hire some workers, one of which straightens the metal, another pounds the head flat and another sharpens the tip - we can make hundreds of these pins but only if each worker specializes in one particular task.

Likewise, at the international level, certain countries may have factors that allow them to make certain goods at a lower opportunity cost. China and Vietnam have cheap, non-unionized labour so they enjoy a cost advantage when they make manufactures like toys, gowns and even pins. New Zealand has lots of land so it can make cheap milk and pork. The US is awash with capital and skilled workers so it can make Boeing jets and microscopes cheaper than anybody else. At the H2 level, how either AA or CA arise is not important. At a H3 level, you will learn that these can be due to different factor endowments - the Heckscher-Ohlin Theorem. Or it can be due to Michael Porter's Competitive Advantage - Singapore doesn't have oil fields but it's highly trained university grads enable it to make cheap oil rigs. Jagdish Bhagwati's Kaleidoscopic CA explores how on earth can China enjoy cost saving in making US iPhones and iPads.
\subsection{The 2 Country 2 Good Model}
The theories of AA and CA explain both why trade happens and why trade is good. To do this, we use the 2 Country 2 Good model. We make the following assumptions:
\begin{itemize}
\item There are no transport costs or any other barriers to trade
\item There are constant returns to scale
\item Equal factor endowments
\item There is perfect intranational factor mobility
\item There is perfect international factor immobility
\item The world consists of two countries making two goods only
\end{itemize}
Now these assumptions are not necessarily desirable. Clearly only a sucker would turn down economies of scale and the assumption of perfect international factor immobility runs smack in the face of globalization. But we need these assumptions for now. Let's consider the case of two countries, Japan and Vietnam.
\subsection{AA}
Fair warning: Adam Smith's AA is not on the H2 syllabus. But it's not very complex stuff.

Here, Japan has the AA in making microscopes but Vietnam has the AA in making shoes. Thus, world output can be maximized if Japan makes only microscopes and Vietnam only shoes - trading with one another. Basically if you can make it better than your trading buddy, you should make it. Again it's not complicated which is why it's not on syllabus. Don't waste your time - just go in and out quickly
\subsection{CA}
This is David Ricardo's "difficult theorem" - it says that mutually beneficial trade can still happen when one country can produce both goods cheaper.

Due to differing factor conditions in reality, this is more likely a case: one country enjoys AA in both goods. In particular, this is common when we compare a developed country like Japan with a developing country like Vietnam. The DC is likely to enjoy an AA in both goods because it has more capital, better tech and more skilled workers. In Adam Smith's time he may well not have considered this as his time was when trade was mostly happening between similar countries (e.g. England and Portugal) while today's East Asia emerging economies were still getting high off the copious opium. Yet Ricardo says that even in such conditions where all the AA is held by one country, mutually beneficial trade can still happen - if the countries produce both goods with different relative opportunity costs.

Here, we see that while Japan has the AA in both goods, Vietnam enjoys the CA in making shoes. While Japan has to sacrifice 2 microscopes to make one pair of shoe, Vietnam only has to sacrifice 0.5 microscopes. Japan, however, has the CA in microscopes.

Let's introduce some trade. Due to trade, Japan specializes in microscopes and Vietnam in shoes. Because this is CA and not AA, we cannot consider complete specialization. This is the trickiest part - you must get the numbers right you'll end up disproving centuries of economic thought. Here, Japan transfers 3 factor units to microscopes and Vietnam transfers 7 to shoes - because Japan enjoys an AA in both goods, it must transfer less factor units. Yes, I am aware this is technically cooking the books to get the results I want but this is a necessity when ceteris is not paribus.

Thus we show that with CA-based specialization free trade is mutually beneficial to both countries. QED.

Unlike the Law of Demand or AA, the theory of CA is not blindingly obvious. This thought experiment can simplify the reasoning. Imagine a lawyer - with his fancy law degree and his own practice. He's considering hiring a secretary. The lawyer can actually type faster and do everything better than the secretary but it still makes sense to hire the secretary because the lawyer's comparative advantage is in higher-order tasks: writing briefs, defending clients and so on. 

To wit: It doesn't matter how efficient you are as a country in absolute terms. It does matter how efficient you are in terms of other goods that you could have been making.

In the next lesson, we'd be learning the benefits of trade as well as some of the hazards of completely free trade. We'll also be covering the justifications proposed by protectionists and why game theory proves that protectionism inevitably begets more protectionism.
\section{The Winners and Losers}
\subsection{The PPC and Trade}
We've shown in the previous lecture that trade based on CA can lead to increases in world output. Indeed, the fundamental benefit of trade is that it enables individual trading countries to consume at a level outside their Production Possibilities Curve. We can illustrate this on a diagram.

We consider the case of an LDC and a DC. The LDC has a comparative advantage in wheat production and the DC enjoys a CA in cloth. The PPC for both countries is shown above in blue. As CA theory assumes no scale economies, the PPC is a straight line.

Before trade, let's assume the LDC consumes at point $e - 200$kg of wheat and $400$m of cloth. Let's assume the DC consumes at point $k - 1600$m of cloth and $400$kg of wheat. Without trade and assuming no improvements in either the quantity or quality of factor inputs and constant technology, both countries can only consume along or within the blue line - they are bound by their PPC.

Both now, both countries trade. As the LDC has a CA in wheat, it will only make 1000kg of wheat (point a) and no cloth. The DC will make 2400m of cloth but no wheat - point m. Assuming 1m of cloth buys 1kg of wheat, the trade consumption possibility frontier, the red line, shifts out. Both countries can still only produce within their PPC but can consume beyond it. With this particular terms of trade (TOT), the slope of the CPF will be one. Let's say that the LDC decides to consume only 400kg of wheat domestically. There will now be a surplus of 600m of wheat - traded with the DC, this buys 600m of cloth. This is point x on the left-hand-side diagram. Under autarky and producing and consuming 400kg of wheat, only 300m of cloth could be consumed. A similar argument applies to the DC. Trade enables countries to consume at a level beyond their PPC and increase their material standard of living. The empirical evidence shows that free trade leads to unambiguous welfare and income gains on the aggregate: The WTO estimates that the removal of all trade barriers would deliver over 6\% of world income - an unfathomable USD 1.8 trillion.

However, there is an important caveat. Due to the diminishing marginal rate of substitution, returns are unlikely to be constant regardless of scale - as output increases more and more ill-suited resources that yield low marginal products have to be employed. This, the PPC is like to be bowed-out rather than straight. Thus unlike our example, countries rarely completely specialize. While Singapore may have a comparative advantage in, say, high value-added services - it doesn't make sense to devote all of its resources to only make services and force the aunties in the Western Digital hard drive factories to take up positions as Shenton Way bankers or Prudential insurance actuaries. Sometimes, transport costs may be so excessive that it negates any gains from trade and trade does not happen - China may be able to 'produce' a haircut cheaper than Singapore but Singaporeans aren't likely to start flying to Shenzhen to dye their hair.
\subsection{Other Benefits of Trade}
Beyond allowing countries to consume beyond their PPC, trade has many other benefits both at the microeconomic and macroeconomic levels.

Trade can allow countries to gain access to cheaper raw materials - indeed for countries like Hong Kong and Singapore with no natural factor endowments free trade is the only way they can gain access to these precious materials. In 2010, value of Oil Imports for Singapore was US\$ 78.1 billion. The US imports 58\% of all its petroleum needs. This keeps cost-push inflation under control. The increase in factor inputs also pushes the PPC outwards.

More importantly, free trade exposes domestic monopolies to tough competition. In the UK, the motorcycle market was dominated by many big, inefficient firms like British Leyland - cushioned by pre-Thatcher protections and tariffs. The CB 750, introduced by Honda in 1968 after trade liberalization, took the industry by surprise: it was bigger, faster and better than anything the British could offer and it blew the competition away. The British industry staggered along with fewer companies and more mergers - only nine firms were left by 1969. Some attempts were made to create new machines to compete against the Japanese - the Triumph Trident, for one - but they were too little, too late. The last British motorcycle manufacturer, Triumph, closed in 1983 after 100 years of operations. This boosts dynamic efficiency and encourages innovation while reducing allocative and X efficiency losses from lazy monopolists or oligopolists. At the same time, consumer welfare in enhanced as they have more choices to choose from. Trade can also boost what Michael Porter called competitive advantage - a multi-faceted source of CA we will discuss later - by reducing monopoly power and incentivizing innovation.

For LDCs in particular, exports can generate significant foreign exchange earnings that increase GDP and growth rates - it is an 'engine of growth'. Recall that (X-M) is a component of AD, such export-led booms. After scrapping tariffs that averaged 200\% in the 1990s, India's trade to GDP ratio has increased from 15 percent to 35 percent of GDP between 1990 and 2005. Since then, its GDP growth rates per annum have been one of the highest in the world. In the year 2010-11 India's total merchandise trade (counting exports and imports) stands at \$ 606.7 billion and is currently the 9th largest in the world. During 2011-12, India's foreign trade grew by an impressive 30.6\% to reach \$ 792.3 billion while GDP grew 6.5\%

Free trade can even lead to greater intranational and international equity of both wealth and income. The factor prize equalization theorem by Paul Samuelson postulates that as LDCs specialize in labour-intensive goods in which they have a CA in, the demand for labour will increase. Ceteris paribus, wages must increase in the LDC - levelling household incomes in LDCs and DCs. David Stern postulates that such trade allow world incomes to converge "one-half of one percentage point per year faster than might otherwise occur". Within in the country, jobs in export sectors employ workers and provide them an income. Trade liberalization in China has lifted 200 million people out of abject poverty since Deng Xiaoping's sweeping economic reforms in the 1980s. Per person income has climbed from USD 16 a year in 1978 to USD 2000 in 2008. In 25 years, absolute poverty plummeted by 60\%.
\subsection{Terms of Trade}
We're about to move on to protectionism and why countries may have somewhat legitimate arguments to restrict trade. But first in order to assess the benefits of trade, we have to be familiar with how to derive the terms of trade or TOT. In general:
$$\textrm{Terms of Trade}=\frac{\textrm{Export Price Index}}{\textrm{Import Price Index}}×100$$
The terms of trade measures the rate of exchange of one good or service for another when two countries trade with each other. For international trade to be mutually beneficial for each country, the terms of trade must lie within the opportunity cost ratios for both country. If Japan's opportunity cost of making 1 microscope is 0.5 shoes and Vietnam's cost is 2 shoes, a TOT of 1 microscope for 1 shoe, for instance, will benefit both countries as it lies between the opportunity costs.

Of course, the final TOT will depend on the demand and supply conditions. In this particular example, the domestic price of the good in country A is P1. However the world's price without trade is P2. Therefore, country A can make a buck by exporting goods to the world. The final price of the good will settle between P1 and P2 - say at P3. Here, AB of goods is exported - which must equal distance CD precisely. Country A's export is the world's import.

Is a low TOT necessarily bad? In general this means that each unit of export can now buy less imports - this is very common in LDCs, the reason for which we will explain later. This may well lead to a decrease in the material standard of living. But it's all about the details. A low TOT can be due to many reasons - such as a devaluation in the exchange rate (which makes exports cheaper and imports dearer). However, this devaluation may enable a country to regain competitiveness and increase the quantity of exports. For example, UK in 1992 benefited from a decline in terms of trade and China persistently enjoys low TOT due to its deliberate undervaluation of the Yuan by the monetary authorities - both however, continue to enjoy high growth as quantity demanded for exports rise with lower prices and compensate for the lower TOT. It is more worrying if the TOT declines due to structural reasons - such as an oil crisis or peak oil that increases import prices without a corresponding increase in export volume.
\subsection{Introduction to Protectionism}
I cannot stress enough that economists everywhere tend to view free trade as a good thing. However, many countries around the world can and do restrict, in one way or another, the free flow of goods and services between countries. This is known as protectionism. This can take a few forms. We can have direct tariffs - taxes and duties on imports. We may impose quotas. Foreign currency controls are also possible - non-Indian nationals travelling into the country cannot legally import any amount of Indian rupees without clearance. Embargoes are economic sanctions in which there is a total government ban on certain imports or exports to specified countries.

In the modern world, thanks to the help of the WTO whose Uruguay Round produced 22,500 pages of regulations regarding permitted and prohibited tariffs, protectionism has become creative. Procurement rules like the Buy American Act of 1933 require federal contractors to give priority to tenders by US firms. Export subsidies are another way - In recent years the US government has made annual outlays of over \$1 billion in its agricultural Export Enhancement Program (EEP) and its Dairy Export Incentive Program (DEIP). Of course the EU's Common Agricultural Policy paid €1.0 billion in export subsidies in 2008 - mostly to dump cheap milk powder, butter and sugar in LDCs. Creative administrative barriers and 'safety' requirements are another crafty way of sneaking protection past the General Agreement on Tariffs and Trades. The 1990s Beef Hormone Dispute saw the EU banning North American beef containing artificial bovine growth hormones - that's most of their beef - ostensibly to protect against mad cow disease. However, it kicked up a stink so bad it reached the WTO's Dispute Settlement Body and led the authorization of \$116.8 million in tariffs by the US on EU exports.

Whatever form they may take, tariffs are economically inefficient. Consider this diagram.

Before the tariff the world price is Wp - the tariff raises the price. For consumers, this effects a Qd-Qd1 fall in quantity demanded. The consumers lose TDBWp in consumer surplus - this is a welfare lost to society. However, this is not all a deadweight loss. The producers - who are now able to increase quantity supplied Qs-Qs1 - gain TCAWp in additional producer surplus. The government collects CDFE in tariff revenue. The remaining area - triangles ACE and DBF - represents the deadweight welfare loss to society because of the tariff.

The same diagram applies to any form of quota. If the government imposes an import quota of Qs1Qd1 the effect will be equivalent to a tariff and raises the price of the good to T. However, the tariff revenue of CDFE becomes quota rent. This explains why countries like Japan often agree to Voluntary Export Restraints (mostly from the USA) - a seemingly suicidal move that slashes export earnings. In addition to saving face and avoiding a potentially devastating trade war, such VERs mean that the quota rent accrues to Japanese firms employing Japanese workers paid in Japanese Yen that is then used to buy Japanese products in Japanese stores - triggering the multiplier effect. If the US was forced to impose a quota, the effect on output quantity would be identical - except the tariff revenue now goes directly to the US Treasury's coffers.

And these are just the economic welfare effects of a tariff. Think of all the extra costs incurred in monitoring and administering the controls - all of the border guards, customs agents, import middlemen, lobbyists and bureaucrats that protection keeps in business. The total annual budget of U.S. Customs and Border Protection alone is \$11.84 billion - 50\% of the budget of the US Department of Education.
\subsection{Why Protect?}
The thing that while we can show that free trade is good for the world in general, we can't say that its rewards are equitably distributed amongst countries and socioeconomic groups. There will be winners and losers. Generally, these arguments for protection target the assumptions of CA theory and have varying degrees of validity.

The first and most common justification for protection is the failure of the assumption of perfect intranational factor mobility. In reality such mobility almost never exists. Often politicians are pressured to impose strict import controls or generous export subsidies because they are necessary to prevent a complete collapse of a sunset industry the country no longer enjoys a CA in and the massive structural unemployment this may result.

This is particularly troubling with Bhagwati's Kaleidoscopic CA - which postulates that CA of different countries, as opposed to being static, is dynamic and changing all the time. Today, Singapore has a CA in high-order services but just 20 years ago this CA was in engineering and manufacturing. Such shifts of CA can seriously destabilize the economy and jeopardize long-term confidence - when the CA in hard disk manufacturing shifted from Singapore to Taiwan, Vietnam and the other emerging Asian Tigers in the 1990s and early 2000s, over 42,000 were laid off in less than five years. Flatted factories, once owned by Maxtor and Seagate and company and each representing huge sunk costs lay abandoned in Tuas and the heartlands. In such cases, temporary and selective import controls can allow old industries to decline gracefully without massive structural unemployment and permit supply-side re-skilling polices which often have long lags to kick in - this is known the senile industry argument.

However, such controls tend to prove to be irreversible in practice and end up being neither selective nor temporary. Section 201 relief for US steel exporters (having lost their CA to the Far East) in the 1980s alone, according to an Ernst \& Young LLP study, exceeded more than \$30 billion. As of December 2001 there were 290 outstanding antidumping and countervailing duty measures in place, of which 156 cover steel products. Why this extravagance? Because it's political suicide to repeal such relief when in 2001 alone, nearly 30,000 American steelworkers lost their jobs and when the Rust Belt reports unemployment as high as 20\% in the worst-hit areas.

And which such measures may protect jobs in the protected sectors, such import tariffs hurt other industries and businesses who have to either switch to inefficient and dearer local producers or cough up the additional import duties if they continue to import their raw materials. These firms, in an attempt to recoup losses, may retrench more workers - one worker protected, ten harmed through higher prices for inferior goods or being laid off. In the USA, steel users employ 57 workers for every one employed in steel production. The problem is that such protection invariably goes to firms who can afford to pay lobby groups to fight their cause on Capitol Hill - even though steel users account for 13.1 percent of GDP while steel producers account for only 0.5 percent, the strong steelworkers union in the industrial north-east and the well-heeled steelmaker lobby ensures that these privileged workers continue to be protected at the expense of consumers, businesses and the government.

The second targets the assumption of constant returns to scale. The fact the very significant economies of scale exist for many manufacturing firms gives rise to the infant industry argument. The theory is that new, smaller firms tend to be unable to be price-competitive versus bigger international titans of business who enjoy these scales economies. Ernesto Zedillo, in his 2000 report to the UN Secretary-General, recommended "Legitimising limited, time-bound protection for certain industries by countries in the early stages of industrialisation" for this reason. Many countries have successfully industrialized behind tariff barriers. For example, from 1816 through 1945, tariffs in the USA were among the highest in the world - peaking at 44.6\% in 1870. By 2010, rapid economic growth enabled the US to slash it to 1.3\%.

Of course, the assumption that Ernesto makes is that these controls will eventually be lifted once local producers grow big enough to reap scale economies. Unfortunately as Pringles potato chips would like to remind us - you can't stop at one. Firstly, retaliatory barriers from trading partners can potentially hurt the same infant firms that protection is intended to help and necessitate even bigger barriers to shield these infants - kicking off a vicious cycle. Secondly, once industries enjoy a little taste of protection, they tend to want more and stronger protections. If the government chooses to cushion these firms with higher tariffs, these firms may become complacent and suffer X-inefficiency. At the same time, restricting foreign competitors increase the monopoly power of domestic producers - leading to allocative inefficiency and compromised consumer sovereignty. The producer gains producer surplus but much consumer surplus is lost. Also, far from encouraging growth and efficiency, this complacency can lead to perpetual infants that never grow up. For example, during the 1980s Brazil enforced strict controls on the import of foreign computers in an effort to nurture its own "infant" computer industry. This industry never matured; the technological gap between Brazil and the rest of the world actually widened, while the protected industries merely copied low-end foreign computers and sold them at inflated prices. 

Think of infant industries like having real children. It's OK to have your kids staying with you until they're 20 or 21 but if they choose to stay until they're 45, you start telling the neighbours they've died of polio.

The problem of increasing returns to scale doesn't just lead to the infant industry argument. Economist Lester Thurow postulates that it can allow for what is known as strategic trade policy. Due to the existence of such economies, many industries have high minimum efficient scale and can only accommodate very few large firms. Let's take the example of the world airliner market, effectively a duopoly held by Airbus and Boeing. Let's further assume that they want to introduce a wide-body 500-seater jet. They can either enter or leave the market. Warning, H3 material ahead.

If both firms enter the market, economies of scale ensure that the market will be saturated and the firms must face higher operating costs - thus they will both make a loss of \$10 million. If both firms leave, then obviously nobody makes a dime. If one firm enters and the other leaves, the entrant pockets \$10 million.

In this case, whichever firm enters the market first will always emerge the winner. Let's say Airbus enters the market. Boeing can either enter or stay out - if it enters it will lose \$10 million or Box A. If it leaves, it doesn't earn anything but it doesn't lose anything either, Box B. The rational loss-minimizing move is to leave. This is known as a first mover advantage: the firm who's quicker on the draw pockets the \$10 million. This only exists because economies of scale ensure that the market cannot accommodate both players.

Thus both Europe and the US will do whatever it takes to make sure their firm is the first one into the market - this is most often done through very generous subsidies and tax breaks or preferential procurements rules for national flag airlines. Protectionism can ensure that that one country's firm is the first to market and pockets the 10 million. However, what this tends to result is a race to bottom with increasingly lavish subsidies and protection propping up loss making ventures on both sides of the Atlantic - which seems to be the case with the Airbus A380, for instance.

Another argument is that the benefits of trade are uneven. The problem is that many of the world's LDCs are heavily dependent on exporting primary products. The Nigerian economy is heavily dependent on the oil sector, which accounts for over 95\% of export earnings and about 40\% of government revenues according to the IMF. The problem is that these primaries tend to have very low income elasticity of demand. This may lead to a worsening of the TOT for these countries - which gets worse as world income grows at an accelerating rate. Here's why.

Consider the formula for TOT. Let's take the case of Nigeria.
$$\textrm{Terms of Trade}=\frac{\textrm{Export Price Index}}{\textrm{Import Price Index}}×100$$
Suppose the world's income grows. The demand for Nigerian oil will, ceteris paribus, increase. But this increase is less than proportional to the rise in income because the demand for crude oil is income-inelastic. So there will be a modest rise in the export price index. The import content of Nigeria however has largely income-elastic demand - manufactures account for 86.45\% of Nigerian imports in 2010. Thus the demand for Nigerian imports (due to rising income in Nigeria) will lead to a more than proportional rise in the demand and thus price of imports. As this a fraction, when the increase in the denominator exceeds that of the numerator, the TOT must fall, ceteris paribus. As we have established, this means that each dollar of Nigerian exports can now buy less imports and may lead to worsening standards of living. This problem is worsened by the fact that such primaries are sold in markets that approximate perfect competition and must command lower prices. The prices for such primaries tend to be very volatile. Thus, some protection of secondary and tertiary industries can be a first step in moving countries away from primary exports towards tertiarisation.

However, an equally valid concern is that protectionism may actually lead to more inequity. The removal of all of the rich countries' barriers to the merchandise exports of developing countries—including agriculture, textiles and other manufactured goods- would add 0.6 \% to the GDP of low- and middle-income countries, according to World Bank data. The benefits to poor people in developing countries of removing rich countries' trade barriers would be more than twice the \$50 billion in annual development aid that rich countries now provide. Especially in today's economic climate, a very pressing and alluring reason for protection is an emergency measure to increase GDP growth and decrease a BOT deficit.

This argument is very tempting because ceteris paribus, economic theory predicts that such protection will lead to unambiguous gains in national income. This is because protection will either reduce import leakages from the circular flow or increase export injections or both - which will trigger the multiplier effect and lead to an export-led boom. Remember that (X-M) is a component of AD. With a BOT deficit, a cut in import expenditure or increase in export receipts must, by definition, improve the BOT position.

However, the problem is that ceteris is not paribus. Such polices will tend to lead to a whole slew of retaliatory barriers to trade from major trading partners. This is because your improvement in BOT and your export-led boom comes at the expense of their trade position and growth. Because of the dismal state of funding of our space exploration programs, the Earth doesn't trade with other planets; thus the balance of trade for all countries on the planet must sum to zero - one nation's BOT surplus must be funded by another's deficit. This is exactly what happened with the 1930 Smooth-Hawley Tariff Act in the US that raised tariffs on over 20,000 imported goods to record levels in an attempt to stave off the Great Depression. In May 1930, USA's greatest trading partner, Canada, retaliated via new tariffs on 16 products that accounted altogether for around 30\% of U.S. exports to Canada. Total U.S. imports decreased 66\% from US\$4.4 billion (1929) to US\$1.5 billion (1933) but because of the retaliation, US exports also decreased 61\% from US\$5.4 billion to US\$2.1 billion. Exports minus imports declined from 1 billion to 600 million, while GDP was \$58.9 billion, resulting in a trivial effect on GDP of about two-thirds of 1\% - the retaliation will cancel out any short-term gains in GDP or BOT. Protectionism is a fallacy of aggregation. On the other hand, free trade is proven to increase incomes - an increase in the share of trade in GDP of one percentage point raises income levels by between 0.9\% and 3\%.

In this global recession especially, everybody's in the same sinking boat. Now imagine there's an idiot who's trying to climb onto the bow in order to keep himself dry while causing the boat to dip even lower. If you slap on tariffs to protect your domestic export earnings in this shared economic fiasco, you are that idiot. You are virtually inviting retaliation. During these troubled times, beggar-thy-neighbour policies will just make us all beggars.

For the BOT in particular, the problem is that such indicators are, for the man on the street, just numbers. A BOT deficit doesn't mean a worse standard of living. Think of it on a personal level. I trade all the time - I run a persistent BOT deficit with Jurong West because, unemployed, I produce no services or goods yet continue to buy Jurong West goods and services. But does this mean I should stop trading with Jurong West; should I impose tariffs on the produce I buy at FairPrice? Should I sign a VER with Kinokuniya? I don't really want to grow my own lettuce. Unlike the BOT, the gains from trade are real and are enjoyed by most in the form of cheaper and better goods. Trying to pursue a BOT surplus for its own sake is like blowing up Bedok Reservoir because you can't get your tap to stop leaking: it's overkill and may even lead to a depression of current standards of living.

Human rights are often invoked to justify protections - such as the New York State ordinances that require police departments and fire departments to buy uniforms from sweat-free, unionized US garment makers. Now we're not denying that sweatshops are deplorable. In 2003, Honduran garment factory workers were paid US\$0.24 for each \$50 Sean John sweatshirt and \$0.15 for each long-sleeved t-shirt - less than one-half of one percent of the retail price. Even comparing international costs of living, the \$0.15 that a Honduran worker earned for the T-shirt was equal in purchasing power to \$0.50 in the United States.

But these arguments outright reject the theory of CA and may actually hurt these LDCs while benefitting DC workers. These LDCs should be allowed to exploit their CA due to cheap labour in order to make export earnings. Also, the sad fact is that matter people in the world live in grinding poverty. Basic economic theory tells us that if an LDC worker voluntarily works for a sweatshop, it must mean his other choices were even worse. The absence of the work opportunities provided by sweatshops (17\% of formal Mexican employment comes from the border maquiladoras) can quickly lead to malnourishment or starvation. After the Child Labor Deterrence Act was introduced in the US, an estimated 50,000 children were dismissed from their garment industry jobs in Asia, leaving many to resort to jobs such as "stone-crushing, street hustling, and prostitution."
Negative externalities are a legitimate argument for restricting trade. Singapore is a free port with almost no import tariffs - but let's look at the only classes of goods that are subject to import duties in Singapore:
\begin{itemize}
\item Intoxicating liquors
\item Tobacco products
\item Motor vehicles
\item Petroleum products
\end{itemize}
These are not only goods with negative externalities but the first two are demerit goods as well. The reason that you must have an almost full tank when driving out of Singapore is not to protect SPC from big, bad Petronas but rather to impose a Pigouvian tax to internalize the negative externality.

While this is an economically valid argument for protectionism, you must be careful. You can only protect an industry if that industry exists domestically. What do I mean? Singapore doesn't have domestic tobacco makers so the tariff on tobacco products doesn't technically count as protectionism.
\subsection{Game Theory and Protection}
All this doesn't even begin to address the most fundamental problem of protectionism: that economic theory predicts that protectionism will also beget more protectionism and leave everybody worse off. Again, this is H3 material but it is in understanding why so many economists froth at the mouth when they read about a new tariff or another VER. Consider two countries, Japan and the USA, who have the option of either embracing free trade or protecting.

If you watched A Beautiful Mind with Russell Crowe, you'd know where I am going. This is the classic prisoner's dilemma, one of the fundamentals of two-player game theory. If both countries trade, CA tells us that both countries can gain \$5 million of output gains from specialization. However, if both countries protect, they lose \$5 million from the welfare losses of protection. If one country trades while the other protects, the trader loses \$10 million while the protector gains \$10 million. This is because the protection decreases the trader's export earnings while decreasing the protector's import expenditures; this is why I say one country's surplus is another's deficit. You realize that if Japan protects, the USA wants to protect as well. This is because it loses \$5 million by protecting (Box A) as opposed to \$10 million by trading (Box C). If Japan plays the sucker and trades, the US wants to protect and earn \$10 million (Box B). Regardless of what Japan does, the US will protect. As this is a symmetrical game, the same argument applies if the US makes the first move. The end result is that both countries protect and lose \$5 million (Box A) - the sub-optimal Nash equilibrium. 

This is why it's so hard to countries to agree to free trade and why the Smoot-Hawley Act failed - because every fibre of their economic being screams at them to go for the Nash equilibrium even though they that economics say that protection sucks and hurts everybody; the dominant strategy is still protectionism. In order to stay at Pareto optimal Box D we need a binding contract that prevents either country for defecting in order to pocket the \$10 million - hence a Free Trade Agreement. But when you rope in close to 200 countries and territories, the transaction costs needed to come to any such contract is daunting.
\subsection{Closing Thoughts on Protectionism}
In general, if you are ever asked to evaluate protectionism in an essay or case study you must, in your conclusion, state that protectionism is a second-best solution with concentrated gains and diffuse costs - a fallacy of aggregation and a second-best solution. While it may yield temporary benefits for one country in the short-run, protectionism will tend to lead to compromised output and incomes for countries as a whole in the long-term. A better first-best option would to be directly address the problem of dwindling comparative advantage that tends to be behind most trade deficits through supply-side policies like worker retraining or raising standards of education.
\subsection{The Iowa Car Crop Model}
A very fresh way of looking at the benefits and risks of trade is the Iowa Car Crop hypothesis:

"There are two ways for the US to produce automobiles: they can build them in Detroit or grow them in Iowa. Growing them in Iowa makes use of a special technology that turns wheat into Toyotas: simply put the wheat onto ships and send them out into the Pacific Ocean. The ships come back a short while later with Toyotas on them. The technology used to turn wheat into Toyotas out in the Pacific is called 'Japan'" (David Friedman's 'Iowa Car Crop' model)

The moral of the story is that the effects of free trade are almost indistinguishable from the effects of a major revolution in technology. The benefits of free trade - cheaper, better and more varied goods - can be brought about by technology as well. Similarly, like free trade, technology can destroy jobs: containerization cost millions of stevedores and longshoremen around the world (2.4 million in the US) their livelihoods. Yet somehow, we protest at WTO summits when a free trade agreement is brokered but not at Apple stores when a new iPhone model is pushed out. We just don't want to make the connection.

More fundamentally, trade represents a voluntary economic transaction between two consenting parties and is indistinguishable from our transactions domestically at work or at the shops. If trade is happening, it must be that both parties to this transaction derive mutual benefits. To stop such a voluntary exchange would be to deny a clear Pareto improvement and thus undesirable. If you could make the world better off without making anybody worse, why should you not?
\section{Post-Ricardian Trade Theories}
\subsection{Introduction to New Trade Theory}
The Ricardian CA revolution definitely turned the classical and mercantilist theories of days gone past on their heads. But that's the problem with doing economics as part of any standardized curriculum. While the H2 syllabus stops progressing, economic thought still marches relentlessly on. Today, Ricardian trade theory has been mostly supplanted by New Trade Theory that can better explain empirical trends and anomalies in recent years.
\subsection{Intra-industry Trade}
H2 CA lectures will tend to have one believe that trading patterns of countries are clear cut. China trades in manufactures while Singapore deals with services. Now this may have been fine 50 years ago but increasingly today, we are experiencing an increase in intra-industry trade or IIT. We can actually measure this quantitatively via the Grubel-Lloyd Index and the data tells us that countries, especially in DCs, are swapping similar or even homogenous goods - activity that cannot be explained by CA differences. Of course part of the reason may be data aggregation. The USA may have a CA in making giant Boeing 747s while Singapore has a CA in making integrated circuit boards the size of my thumbnail - both will still end up classed as 'manufactures' even though they are distinct goods.

But even if we account for methodology, over 60\% of European trade and 57\% of American trade is IIT. This is a problem for CA. Paul Krugman bagged a Nobel Prize for cracking this code. His theories suggest that IIT can be accounted for many reasons. The first is differing tastes and preferences - Germans may just prefer Volvos (made in Sweden) while the Swedes may like BMWs (made in Germany) so there's IIT in cars between Germany and Sweden. CA, a supply-side theory, never considered the demand aspect of things. Consumers are fickle and are willing to pay for variety.

The second thing is that many manufactures have very large MES - as is the case with aircrafts and cars. Thus, Krugman states that much of such specialization is not rooted in CA but rather "accidents of history". There is no real reason why Volvo should be in Sweden - it just is. The steel in Sweden and the rubber of Sweden don't really care if they end up in a Volvo or a BMW. Volvo chooses to base itself in Sweden because of purely arbitrary reasons unrelated to CA. But once Volvo's there however, the economies of scale generates higher ROIs and attracts even more investment which allows for even bigger scale economies and a virtuous cycle that's self-sustaining and ever larger profits for Volvo. So in this case, the Swedes may specialize in making Volvos while Germany may specialize in another high-MES good like BMWs - leading to high IIT of cars between the Germans and Swedes; such trade happens because, remember, we have the different tastes and preferences in both countries. Incidentally, this first-mover advantage presents a clear case for strategic trade policy, as we have discussed
\subsection{Gravity Model}
Paul Krugman also came up with the Gravity Model of international trade which factors in transport costs and the relative GDP. This is the formula, where F is the volume of trade between countries I and J.

We see that the shorter the distance, the more trade happens - 63.7\% of trade in the EU is intra-EU. This is because of a few reasons. The first is that closer countries lead to lesser transport costs and more profitable trades. The second is that countries close to one another tend to be very nice buddies for reasons of national security and often have ratified free trade agreements or common market agreements like the NAFTA, ASEAN, the African Union or the EU that lower trade barriers.

We also note that the richer the countries are (where the M is the economic 'mass' of a country expressed as total GDP) they more trade happens. This is because richer consumers tend to have a more refined and diverse tastes in exports - as well as the disposable income to buy these income-elastic-demand exports. Both GDP and transport are not factors considered in CA.
As with gravity with physics, we don't know the value of the constant G and thus cannot quantify F in any meaningful way.
\subsection{Dynamic CA}
Jagdish Bhagwati's theory of Kaleidoscopic CA is important and depends on the fact that many consumer products have limited life cycles. He says that differentiated products like consumer electronics, undergo a three-stage process of growth

Stage 1: New Product Stage - the product is produced and consumed in the origin country (most likely a DC) in order for the firm to observe the market response and because of protection from the relevant patent laws. The CA is in the DC.

Stage 2: Maturing Product Stage - As more and more firms replicate the product, mass production is utilized and we observe increased homogeneity. Think of the original Apple iPad versus all the other Samsung, Sony and HTC rip-offs that took to the market in just a few years. As the product becomes homogenized, firms start looking for low-cost locations.

Stage 3: Standardized Product Stage - Competition becomes intense and the lowest-cost producer. Offshoring of manufacturing via Foreign Direct Investment happens (we assume that capital doesn't flow internationally in Ricardian trade theory) The CA is now enjoyed by the LDC, transferred from the DC - Kaleidoscopic CA holds that CA is dynamic and will move across borders. The product reaches the end of its 3-stage cycle.

This means countries must always be on their toes and come up with new and better products in order to retain CA. It's also pretty optimistic as far as economics goes - the maturity/standardization stage allows low-cost LDCs to gain a CA in products developed in the DCs and earn export earnings to drive growth; levelling the playing field. This is seen in China, which makes more than 50\% of the world's cameras, 30\% of the air conditioners and televisions, 25\% of washing machines and 20\% of refrigerators - all of them formerly differentiated products that have reached the end of their life-cycle.

"Comparative Advantage is kind of knife-edge, where one day I have comparative advantage in X and you in Y, and tomorrow it may be the other way around, and then back again: a sort of musical chairs" (Bhagwati)
\subsection{Competitive Advantage}
CA doesn't have to come from just factor endowments. Michael Porter's Competitive Advantage holds how Singapore, a country with no oilfields, can make oil rigs. Under his model, CA can come from any source of competitive advantage. This is mostly shown via Porter's Diamond.

In Singapore, for instance, the government invests in education to build up a pool of skilled workers (Factor Conditions) while enforcing the Competition Act to break up monopolies and protect fair competition (Firm Strategy, Structure and Rivalry) while simultaneously investing heavily in support infrastructure like transport and telecommunications (Related and Supporting Industries). A wealthy and educated populace also leads to a demand for refined, high-quality goods (Demand Conditions). Thus, even though Chance may have left us with poor natural resources endowments (Factor Conditions) Singapore continues to enjoy a CA even in making oil rigs or similar goods.
\section{Trade Policy}
\subsection{Singapore's Trade Pattern}
Singapore is a free port - with no tariffs on imports except for the four categories of goods already stated. In addition to the ASEAN Free Trade Zone, Singapore has major ratified FTAs with New Zealand, the European Free Trade Association, Japan, the USA, India, Korea and Peru. FTAs enabled Singapore-based companies to enjoy more than S\$700 million in tariff savings in 2009 when exporting to these countries. Again, Singapore has almost no import tariffs so that point's moot.
\subsection{Different Types of Trade Zones and Agreements}
Generally, when we talk about trade policy, we will have to bring in the FTA - a binding diplomatic agreement to remove tariff barriers on the movement of goods between members. But the FTA isn't the only type of multilateral agreement. 

In a Free Trade Area, the members remove tariff barriers between themselves but maintain individual policies against non-member nations. A Customs Union is an FTA plus an agreement to share common tariff policies against non-members. A Common Market is a CU plus free flow of capital and labour, which are not guaranteed in an FTA; Singaporeans travelling to Malaysia still must get their passport chopped in spite of the ASEAN FTA. An Economic Union is a CM plus shared monetary, fiscal, welfare and trade policies. If they share a currency, that's a currency union.
\subsection{Economic Effects of an FTA}
FTAs are not automatically a good thing. This is because an FTA generates both trade creation and trade expansion welfare gains. The former is new trade generated by the removal of tariff barriers and the latter is an increase in quantity demanded of the good brought about by the lower price - resulting in an increase in consumer surplus.

In this diagram, the removal of the tariff barrier generates trade creation of Q2-Q1 and trade expansion of Q3-Q4. Producers lose area 1 of producer surplus and the government loses area 3 of tariff revenue. However, there is a gain of consumer surplus of (1+2+3+4) and a net welfare gain to society of area (2+4). So far this is looking good.

Yet, we have yet to consider trade diversion - the third effect of a tariff barrier. Unlike trade creation and expansion, this represents a loss of economic welfare and is undesirable. Consider the diagram below. Here, The UK was trading with the lowest-cost producer, New Zealand. But due to the EU, they start buying from the EU instead.

Now, the UK loses (3+5) in tariff revenue but it is all deadweight loss. Remember that the UK gains (1+2+3+4) in consumer surplus. But because of the fact that the FTA causes the UK to 'divert' its trade from the lowest cost producer, there is still a deadweight loss of area 5 - this is the lost tariff revenue that is not transferred to additional consumer surplus.

So is an FTA good? Well it'll depend on the net effect of both trade creation/expansion and trade diversion. If area 5 exceeds areas (2+4) as is in this case, we say that the FTA creates a net loss of welfare and shouldn't happen. However, if (2+4) was bigger than 5, then the FTA would have generated a net welfare gain. When does trade creation exceed trade diversion? It depends on how far the FTA price deviates from the lowest-cost producer price.

In the right hand side, the FTA price significantly deviates from the cheapest price - thus we expect trade diversion to exceed creation and thus a net welfare loss. However, if the FTA price is not that different from the cheapest price then trade creation will dominate, as is the case of the left. Thus theory suggests that the more similar the FTA partners are to the lowest cost producer, the likely the FTA will generate welfare gains
\end{document}
