\RequirePackage{../../dominatrix}

\title{Key Economic Indicators}
\author{\large by Jeremy Tay}
\date{\small correct as of \today}
\begin{document}
\maketitle
\tableofcontents
\section{Inflation}
\subsection{What's A Key Economic Indicator?}
Ok welcome to what is, probably, your first lesson in macroeconomics. Why is it important to learn about key economic indicators you might ask? Well, in microeconomics, there is a set of metrics by which we use to measure the performance of firms and producers. These metric are allocative efficiency, productive efficiency, as well as distributive efficiency (among others). Now, in macroeconomics, there is also a similar framework. Framework is very important in economics. In macroeconomics, the four key economic indicators you will need to know are, the GDP growth rates, the unemployment rate, the inflation rate, and the balance of payments. We shall delve into that later.
\subsection{Inflation}
The textbook understanding of inflation is defined as the SUSTAINED increase in the general price level of an economy over a period of time. Please remember that without the word sustained, your definition is meaningless. If your convenience store auntie raises the price of your milo by 10 cents it doesn't mean that there is inflation; she may just be responding to an increase in the demand for her own goods. However, when prices ACROSS THE BOARD rise; lets say the prices of all drink items start rising; then there is inflation. However, how do we measure this "general price level". That is a good question; Economists often like to use a proxy for the general price level called the consumer price index. Let's talk a little about that.
\subsection{The Consumer Price Index}
In your syllabus, you are not required to understand the construction of the CPI. However, it is good to have an understanding of what it is about. The CPI is essentially an index that is a  weighted average of commonly purchased goods purchased by a household. For example, lets say your family spends 20\% on rent; 20\% on transport; 20\% on food; and save the rest, the index will take the proportionate value of each expenditure and produce an index. Note that the CPI is just an index; it does not give any indication of absolute values; it only gives you an idea of relative change. For example, a base year is often selected, say 2012, and assigned a value of 100, values given for any other year are in relation to that base year.

The use of the CPI is that it measures the change in price of a fixed basket of goods and services commonly purchased by households in a specified time period.
\subsection{High Flyer}
Now, on to more quantitative stuff. The key reason why we learn the CPI is so we can calculate the inflation rate:

The formula is given by : $\frac{CPI\prime-CPI}{CPI} x 100$\% where CPI is the CPI of the previous year, and $CPI\prime$ is the CPI of the current year.
The formula is quite intuitive, it is essentially a "percentage increase" in the CPI from the previous year.
\section{Unemployment}
\subsection{Unemployment}
Unemployment is defined as those of working age who are without work, but who are available for work at current wage rates. For now, let's not worry about the definition. Let's understand the big picture. There are three types of unemployment: demand deficient (or cyclical) unemployment, structural unemployment, and frictional unemployment. For now let's not worry about the technicalities of each type of unemployment; let's look at how this big picture can help us understand the key concept of full employment in the economy.

With the big picture in mind, the economy is at full employment, when there is no demand deficient unemployment; and there is only structural and frictional unemployment.
\subsection{Equilibrium Vs Disequilibrium Unemployment}
At this point, I think it is good to distinguish between equilibrium and disequilibrium unemployment.

Consider Fig A, where the labour market is allowed to clear. N represents the labour force. Notice that at high wage rates, AS tends towards N, this is intuitive; given that workers will be willing to supply more labour at higher wage rates. $Q_e$ represents the equilibrium number of workers who can find a job. At $Q_e$, there is equilibrium unemployment.  Whereas ($Q_2-Q_e$) represents the amount of structural and frictional unemployment in the economy. In some sense; $Q_e$ represents what is "full employment" given our previous definition.

However lets say there is a fall in demand for labour due to a recession; AD falls, however if we assume that wages are sticky downwards ( that is they are resistant to falling), then there will be disequilibrium unemployment of ($Q_e-Q_3$). This level of disequilibrium unemployment; quite clearly, adds to the already natural level of unemployment ($Q_2-Q_e$)

This idea of sticky wages will be introduced later in fiscal policy; hence I think it is a good idea for you to get an introduction to it now.
\subsection{Philips Curve}
Up to this point; we have established that the idea of the natural rate of unemployment can be thought of as the sum of the levels of structural and frictional unemployment. However; there is an alternative viewpoint. As a caveat, what I am going to talk about here is out of syllabus but useful to know; Let's me first introduce you to the philips curve; (figure to be inserted). Now, the philips curve is essentially an empirical relationship between inflation and unemployment. When a government tries to reduce unemployment, it has to suffer higher inflation. This is a Keynesian view point; However, the neoclassical view point is that this inverse relationship will only be true in the short run; the introduction of the expectations augmented Philips curve, will see the introduction of a new philips curve in the long run where inflationary expectations are risen as a result of higher inflation rates, this brings back the level of unemployment to the original point. So, for the government to successfully lower unemployment continuously, it will have to not just introduce unemployment; but to introduce increasingly accelerating rates of inflation; thus, the steady state unemployment rate can be defined as the non-accelerating inflationary rate of unemployment.
\subsection{Demand-deficient Unemployment}
Demand deficient unemployment essentially occurs when there is fall in AD; as labour is a derived demand, this causes a fall in the demand for labour in the labour market. When demand in the labour market falls, and assuming the labour market is allowed to clear, the equilibrium level of unemployment falls from $Q_e$ to $Q\prime$
\subsection{Structural Unemployment}
Structural unemployment usually occurs when there is a shift in the comparative advantage in a country as a result of changing labour market conditions or decreasing cost competitiveness in an industries. This causes firms to shift production to another country. However at the same time, as a result of changing comparative advantages, a country may have a comparative advantage in the production of a different good (e.g.\~a shift from low-skilled to high-skilled manufacturing). However, some workers in the industry that is losing its comparative advantage may not have the skills to take up new jobs in the new industry even though there are jobs available but there is no one with the necessary skills to take up the job.
\subsection{Frictional Unemployment}
Frictional Unemployment usually occurs due to imperfect information in the labour market, where there may be job openings but due to imperfect information; people looking for openings may not be able to locate these jobs even though they possess the necessary skills required. An alternate viewpoint is that individuals believe that their offers will get better as a function of time, so they wait out for better job offers.
\section{GDP}
\subsection{Gross Domestic Production - The Value Produced In Our Borders}
The GDP is defined as the value of all final goods and services produced within a country within a given time.
 
It is easy to imagine what GDP is: value addition. The idea of adding value to something is ultimately normative. Lets say I have a cupcake, and I bought it from my supplier at \$1. But then I add a very nice box and then sell it for \$5. I have added \$4 of value to my cupcake. 
 
An important point to note is that at equilibrium, GDP = National Income = National Expenditure.
\subsection{Gross National Production - The Value Produced By Our People}
GNP is the final value of all goods and services produced in a country by domestic factors of production within a given period of time. Essentially, it is the GDP in a country plus the net factor income from aboard. This means that a country like singapore, with significant investments overseas will accrue a large net factor income from abroad. However, take note that domestic factors of production mean that both foreign and local workers are counted; regardless of their citizenship, as long as they work in singapore during the period of time the calculation is made.
\subsection{Real GDP - The Real Value Of Our Goods}
However, notice that if the prices of everything go up by 10\% as a result of inflation, that does not mean that 10\% more value has been added; thus a more useful measure of how much value is being added is the real GDP. Essentially, the real GDP is the nominal GDP (which measures the absolute GDP number) multiplied by a GDP deflator.

In order to calculate real GDP, just multiply the nominal GDP by $\frac{1}{1+\textrm{inflation rate}}$ which is quite intuitive actually; just reduce the prices of all goods to that of the previous year and we get the real GDP.
\section{Balance of Payments}
\subsection{Bill Please!}
The balance of payments is an accounting record of all monetary transactions between a country and the rest of the world. Essentially a balance of payments is like a leaky glass of water. Water leaks out, and at the same time, someone is pouring water into the cup. We are only interested in the net result: the change in the amount of water in the cup. If the amount decreases, we say the BOP is in deficit. If amount increases, the BOP is in surplus. If the BOP surplus increases, we say there is an improvement in the BOP position.

The BOP consists of the current account and the capital account. An important concept when it comes to the BOP, is that the net surpluses and deficit of all countries in the world must cancel out (quite intuitive and obvious) unless we are doing trade with Mars.
\subsection{Our Current Account - An Understanding Of Our Country's Trade}
The current account comprises of many things ( I will insert diagram), however, the main thing you should be concerned with is net exports. net exports consist of X-M. So if X-M improves, it implies that overseas payments for our exports has increased over our payments for imports; thus there is a net inflow of money. The same is true for the converse.

Another important part of the current account is income transfers such as returns from our overseas investments as well as financial aid. However for countries such as Singapore, income transfers are usually a trivial part of our current account.
\subsection{Our Capital Account - An Understanding Of Our Country's Finances}
The two main factors that affect the capital account that we are concerned about are FDI and the the central bank operations that affect the reserve account. FDI is essentially when domestic citizens invest overseas, or when overseas citizens invest locally. When FDI into a country increases, the capital account improves. Central bank operations affect the reserve account when a central bank buys or sells currency in order to influence the exchange rate. If a central bank sells local currency, the capital account worsens. the converse is true.
\subsection{A Matter Of Definitions}
Now some of you may come across the term financial account; the financial account depending on whether you are using the IMF definition or not, can either be comprised under the capital account as a separate account altogether. The financial account is essentially like FDI just that the investments are in financial instruments like shares and bonds. For simplicity sake, I recommend you classify these investments under the capital account in examinations since technically it cannot be marked wrong.
\end{document}
